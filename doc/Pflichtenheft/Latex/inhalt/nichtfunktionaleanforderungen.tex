% Kapitel 6
%-------------------------------------------------------------------------------

\chapter{Nichtfunktionale Anforderungen}

In diesem Kapitel wird festgelegt, welche Qualitätsmerkmale das zu entwickelnde
Produkt in welcher Qualitätsstufe besitzen soll. Anschließend werden die als am
wichtigsten bezeichneten Qualitätsmerkmale operationalisiert, d.h. in konkrete
Produktanforderungen detailliert, falls sie nicht als allgemeine Richtlinie (z.
B. Standard, Norm) zur Verfügung gestellt werden können.

Sofern projektspezifisch möglich, ist die Anordnung gemäß Beispiel zu nutzen.
Ggf. können aber auch andere Qualitätsanforderungen gestellt oder nicht
zutreffende weggelassen werden.\\
Beispiel:\\

%----Start der Tabelle------------------------------------------------------
\begin{tabular}{|c|c|c|c|c|}
  \hline                                              % Linie oberhalb

  %----Spaltenüberschriften------------------------------------------------
  \textbf{Produktqualität}  & \textbf{sehr gut}  &    \textbf{gut}  &
  \textbf{normal}           & \textbf{nicht relevant}  \\  %Spaltenüberschrift
  \hline                                              % Gitterlinie

  %----die nachfolgeden beiden Zeilen so oft wiederholen und die ... mit den
  %    entsprechenden Daten zu füllen wie erforderlich
  \textbf{Funktionalität}  &&&&\\        % Eintrag in Zeile
  \hline
Angemessenheit&&x&&\\
\hline
Richtigkeit&x&&&\\
\hline
Interoperabilität&&x&&\\
\hline
Ordnungsmäßgkeit&x&&&\\
\hline
\textbf{Sicherheit}&x&&&\\
\hline
Zuverlässigkeit&&&&x\\
\hline
Reife&x&&&\\
\hline
Fehlertoleranz&&&x&\\
\hline
Wiederherstellbarkeit&x&&&\\
\hline
\textbf{Benutzbarkeit}&&&&\\
\hline
Verständlichkeit&&x&&\\
\hline
Erlernbarkeit&&&x&\\
\hline
Bedienbarkeit&&x&&\\
\hline
Effizienz&&&&\\
\hline
Zeitverhalten&&x&&\\
\hline
Verbrauchsverhalten&&x&&\\
\hline
\textbf{Änderbarkeit}&&&&\\
\hline
Analysierbarkeit&&x&&\\
\hline
Modifizierbarkeit&&x&&\\
\hline
Stabilität&x&&&\\
\hline
Prüfbarkeit&&x&&\\
\hline
Übertragbarkeit&&&&\\
\hline
Anpassbarkeit&&&&x\\
\hline
Installierbarkeit&&x&&\\
\hline
Konformität&&x&&\\
\hline
Austauschbarkeit&&&x&\\
\hline
%----Ende der Tabelle------------------------------------------------------
\end{tabular}


Die oben als am wichtigsten bezeichneten Qualitätsmerkmale werden im Folgenden
operationalisiert, d.h. in konkrete Produktanforderungen detailliert oder es
wird angegeben, welche Richtlinie (z. B. Standard, Norm) einzuhalten ist. Diese
Qualitätsanforderungen werden mit /Qnn/ markiert. Zu prüfen ist, ob die
gewünschte Qualität mit den in Produktdaten genannten Datenmengen erreicht
werden kann.



Beispiele:

\begin{itemize}

\item  /Q10/ Die Funktion /F60/ darf nicht länger als 5 Sekunden Antwortzeit
benötigen.
\item  /Q20/ Alle Reaktionszeiten auf Benutzeraktionen müssen unter 2 Sekunden
liegen (außer Funktion /F60/).
\item  /Q30/ Die im Rahmen der automatischen Einlagerung /F10/ notwendige
Platzwahl für einen am Anmeldescanner gemeldeten Reifen darf aus Gründen der
Kommunikation mit der SPS nicht länger als 3 Sekunden dauern, ansonsten kann
die SPS die Lieferung des Reifens zum richtigen Modul nicht garantieren. //
\item  /Q40/ Das Produkt soll plattformunabhängig sein
\item  /Q50/ Das Produkt muss anwenderfreundlich sein (intuitive Bedienbarkeit
für Benutzer ohne EDV-Vorkenntnisse, umfangreiche Hilfefunktion)
\item  /Q60/ Die Produkt soll fehlertolerant bezüglich Bedien- und
Eingabefehler sein

\end{itemize}
