% Kapitel 4
%-------------------------------------------------------------------------------

\chapter{Produktfunktionen}

In Abhängigkeit von den gewählten Konzepten erfolgt hier eine Konkretisierung
und Detaillierung der Funktionen aus den Use-Case-Diagrammen und ggf. dem
Lastenheft.

Wurde beispielsweise im Lastenheft die Funktionalität durch verbal beschriebene
Geschäftsprozesse definiert, dann kann hier eine Detaillierung erfolgen.
Mögliche Gliederung der Geschäftsfunktionen:\\
-  Geschäftsprozesse\\
-  Listen\\
-  Reports\\
Die nachfolgende Gliederung bitte für alle Funktionen übernehmen.\\

/F100/ (/LF10/)\\
-  F100 = Funktion mit eindeutiger Nummerierung in z. B. 100er Schritten, damit
   mögliche Erweiterungen eingefügt werden können. Die Schrittweite richtet
   sich nach der Anzahl möglicher Erweiterungen.\\
-  LF10 = Anforderungsnummer aus dem Lastenheft. Sollte im Lastenheft keine
   Anforderungsnummer angegeben sein, so erscheint unter Anforderung aus dem
   Lastenheft die Beschreibung des Kriteriums.

\begin{itemize}

\item  Geschäftsprozess, der durch die Funktion unterstützt werden soll:
\item   optional: Anforderung aus dem Lastenheft:
\item   Ziel der Funktion:
\item  Vorbedingung:
\item  Nachbedingung Erfolg:
\item  Nachbedingung Fehlschlag:
\item  Akteure:
\item  Auslösendes Ereignis:
\item  Beschreibung:
\item  optional: Erweiterung:
\item  optional: Alternativen:
\end{itemize}



Beispiel (Lagerverwaltung):\\
/F10/ (/LF10/)\\
Geschäftsprozess: Automatisches Einlagern\\
Ziel: Ein Reifen erscheint am Systemeingang (Scanner), erhält einen Lagerplatz
zugewiesen und wird dort eingelagert.\\
Vorbedingung: Das Scannen des Barcode-Reifens muss erfolgreich sein, sonst kann
der Typ nicht ermittelt werden. Solche unbekannten Reifen werden direkt in den
Überlauf gefördert.\\
Nachbedingung Erfolg: Reifen ist physikalisch eingelagert und logisch in der
Datenbank verbucht.\\
Nachbedingung Fehlschlag: Der Reifen wurde infolge gestörter Fördermechanik
nicht eingelagert (liegt im Überlauf) oder produzierte aufgrund inkonsistenter
Datenbank einen "`Platz belegt"' - Fehler beim Anfahren eines irrtümlich
als frei angenommenen Platzes.\\
Akteure: Produktion\\
Auslösendes Ereignis: SPS meldet der Steuerung, dass am Eingangsscanner ein
Reifen mit Seriennummer X des Typs Y eingetroffen ist.
\\Beschreibung:
1 Reifentypinformationen ermitteln (besonders Höhe des Reifens bei Wahl
zwischen unterschiedlich hohen Lagerplätzen wichtig).\\
2 Alle Module ermitteln, die\\
-  Platz auf den Einlagerstichen haben\\
-  momentan nicht im Störungszustand sind\\
-  freie Lagerplätze in der geforderten Höhe aufweisen.\\
3 Lagerplatz nach Gleichverteilungsgrundsatz bestimmen.\\
4 Reifen auf den Einlagerstich des gewählten Moduls befördern.\\
5 Sobald er auf dem vordersten Platz des Einlagerstichs steht, dem Modul den
Befehl zur Reifenaufnahme und Einlagerung auf den gewählten Platz schicken.
\\Erweiterung:
2a Zur Effizienzsteigerung auch Module ansteuern, die momentan keinen Platz auf
den Einlagerstichen haben, aber wahrscheinlich so schnell einlagern, dass der
Reifen nach der Fahrtzeit zum Modul auf den Stich eingelagert werden kann
(Überwachung des "`Unterwegsbestandes"' an Reifen für ein bestimmtes
Modul).\\
3a Lagerplatz des Reifens möglichst nah zum Einlagerstich im RBG wählen
(kürzere RBG-Fahrtzeiten).\\
Alternativen:\\
2a Wenn kein Lagerplatz gefunden wird, Reifen zum Überlauf schicken (der
Einlagerförderer wird niemals angehalten!).\\
\newpage
Beispiel für Listen\\
/F60/ (/LF60/)\\
Reifen-Bestandsliste für das Gesamtlager mit folgenden Daten:\\
Reifentyp, Reifenanzahl, Verteilung auf Module, Datum ältester Reifen.\\

