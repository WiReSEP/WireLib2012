% Kapitel 2 mit den entsprechenden Unterkapiteln
% Die Unterkapitel können auch in separaten Dateien stehen,
% die dann mit dem \include-Befehl eingebunden werden.
%------------------------------------------------------------------------------------
\chapter{Produkteinsatz}
Das Produkt soll als neues System f"ur die Bibliotheksverwaltung in die Institut-Homepage f"ur "`Wissenschaftliches Rechnen"' (http://www.wire.tu-bs.de) integriert werden, und dabei das alte Vorhandene ersetzen. Die Zielgruppe des Produktes beschr"ankt sich dabei nicht nur auf die Mitarbeiter des Institutes, sondern schliesst aufgrund eines Webinterfaces auch Mitglieder der TU Braunschweig oder einfachen Nutzern der Institutsseite mit ein. 
In den folgenden Punkten werden die voraussichtlichen Anwendungsbereiche, die Zielgruppen und die Betriebsbedingungen explizit aufgef"uhrt. 

\section{Anwendungsbereiche}
Das Produkt dient als Managementsystem im "`Bibliothek"'-Bereich des Institutes.

\section{Zielgruppen}
Zielgruppen sind die Mitarbeiter des Institutes, Studenten der TU Braunschweig, Nutzer der \gls{UB} und diverse Nutzer der Institutshomepage.

\section{Betriebsbedingungen}
Aufgrund der Nutzerfreundlichkeit eines universell genutzten Bibliothekssystems, soll 
f"ur den Betrieb ein Notebook oder ein einfacher Desktop PC ausreichen. Das Produkt soll mit allen g"angigen Betriebssystemen (Windows (XP/Vista/7), Unix/Linux) kompatibel sein und ebenso keine Schwierigkeiten mit verbreiteten Browsern wie Mozilla Firefox, Internet Explorer, etc.\ haben. Der Webserver für das Produkt sollte mit g"angigen Linuxdistributionen (mit Linux-Kernel 3.0+) kompatibel sein. 