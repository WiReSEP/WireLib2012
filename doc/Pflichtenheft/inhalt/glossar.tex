% Kapitel 9
%-------------------------------------------------------------------------------
%Hier werden Fachbegriffe und Abkürzungen erklährt.
% Verwendet werden diese Begriffe mit \gls{name} oder \Gls{name} wenn der Anfang
% groß geschrieben sein muss.
%
%%	Beispiel unter: http://ewus.de/tipp-1029.html
%%	und natürlich Erkärung mit dem Befehl
%texdoc glossaries

\newacronym{UB}{UB}{Universitätsbibliothek}
\newacronym{GITZ}{GITZ}{Gauß-IT-Zentrum}
\newacronym{LDAP}{LDAP}{Lightweight Directory Access Protocol\protect\glsadd{glos:LDAP}}

\newglossaryentry{glos:LDAP}{name=Lightweight Directory Access Protocol,
description={}
}
\newglossaryentry{glos:https}{name=Hypertext Transfer Protocol Secure,
description={Hypertext Transfer Protocol Secure ist die verschlüsselte Variante
vom Hypertext Transfer Protocol und ist eine zertifiktasbasierende sichere
Übertragungstechnik}
}
\newglossaryentry{glos:sqlite}{name=sqlite,
description={}
}
\newglossaryentry{glos:copdes}{name=Corporate Design,
description={Das Corporate Design ist das gemeinsame Erscheinungsbild eine Unternehmens. Dies bezieht sich unter anderem auf Kommmunikationsmittel, Werbemittel und Internetauftritte}
}
\newglossaryentry{glos:unicode}{name=Unicode,
description={Der Unicode ist ein Standard, in dem jedes sinntragende Schirftzeichen einem digitalen Code zugeordnet werden soll. Dadurch sollen Kompatibilitätsproblem aufgrund verschiedener Kodierungen in verschiedenen Ländern umgangen werden.}
}
\newglossaryentry{glos:thmefi}{name=Thunderbird Message Filter,
description={Der Thunderbird Message Filter ist eine einfache Möglichkeit E-Mails in dem Programm Mozilla Thunderbird zur organisieren.}
}
\newglossaryentry{glos:BibTeX}{name=Bib\TeX,
description={}
}
\newglossaryentry{glos:Allegro}{name=Allegro,
description={}
}
\newglossaryentry{glos:regex}{name=Reguläre Ausdrücke,
description={}
}
\newglossaryentry{glos:ext}{name=Externe,
description={}
}

