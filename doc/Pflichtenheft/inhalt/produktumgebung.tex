% Kapitel 10
% Die Unterkapitel können auch in separaten Dateien stehen,
% die dann mit dem \include-Befehl eingebunden werden.
%-------------------------------------------------------------------------------

\chapter{Technische Produktumgebung}

In diesem Kapitel wird die technische Umgebung des Produktes beschrieben. Bei
Client / Server  Anwendungen ist die Umgebung jeweils für Client und Server
getrennt anzugeben.

\section{Software}
\begin{itemize}
\item Server-Betriebssystem:              Linux
\item Client-Betriebssystem:                Browser
\item Frameworkprogramm :                Django
\item Programmiersprache:                  Python 2.6+
\item Zur Erstellung der Datenbank:   SQLite
\end{itemize}



\section{Hardware}
\begin{itemize}
\item Server: PC
\item Client: Computer
\end{itemize}

\section{Orgware}

Durch Internetverbindung zum WebsServer soll man sich über die Dokumente in der Bibliothek informieren können, und 
diese ausleihen oder je nach Rolle anderweitig Verwalten. Die Verbindung zum Webserver wird per https  gesichert. 



\section{Produktschnittstellen}

Es wird folgende Schnittstellen geben:
\begin{itemize}
\item Allegro - zum importieren der Daten aus der alten Datenbank, und der Möglichkeit der Uni-Bibliothek unsere Daten zu verwerten
\item BibTex - zum importieren von BibTex Dateien
\item GoogleBooks - um Links und Bilder importieren zu können
\end{itemize}


