% Kapitel 10
% Die Unterkapitel können auch in separaten Dateien stehen,
% die dann mit dem \include-Befehl eingebunden werden.
%-------------------------------------------------------------------------------

\chapter{Technische Produktumgebung}


\section{Software}


\begin{description}
\item [Server-Betriebssystem] - Als Server-Betriebssystem wird GNU/Linux Debian Squeeze mit einem an Python anbindbaren Webserver          (z.B. Apache) verwendet.																		
\item [Client-Betriebssystem] -               Für das Client-Betriebssystem ist allerdings jedes browserfähiges Gerät  vorgesehen, das neuere Browser	 verwendet. 																	
						
\item [Programmiersprache] -                 Das Programm  ist lauffähig unter Python $>$ 2.7 . 	
\item [Framework] -               Wie gefordert, wird für das Frameswork Django 1.2  benutzen.								
\item [Datenbank] - 	 Die Datenbank ist unter  SQLite $>$ 2.8  lauffähig .                                                                                
\end{description}



\section{Hardware}


\begin{description}
\item [Server] - Als Server wird ein geeigneter PC genutzt, auf der die Datenbank und die Website gespeichert wird.
\item [Client] - Auf die Website kann von jedem browserfähigen Gerät zugriffen werden, das aktuelle Browser nutzt. 
\end{description}

\section{Orgware}

Durch Internetverbindung zum Webserver soll man sich über die Dokumente in der Bibliothek informieren können und 
diese ausleihen oder je nach Rolle anderweitig zu verwalten. Die Verbindung zum Webserver wird per https gesichert. 



\section{Produktschnittstellen}


Es werden folgende Schnittstellen benötigt: 
\begin{description}
\item [Allegro] - Diese Schnittstelle wird zum Importieren der Daten aus der alten Datenbank, und der Möglichkeit der Uni-Bibliothek unsere Daten zu verwerten genutzt.
\item [BibTex] - Zum Importieren und Ausgeben von Dokumenten wird diese Schnittstelle benötigt. 
\item [WikipediaBookLocator] - Um Links und Bilder importieren zu können ist es wichtig, dass man über eine Schnittstelle diese einbeziehen und nutzen kann.
\end{description}


