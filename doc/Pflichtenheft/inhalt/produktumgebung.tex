% Kapitel 10
% Die Unterkapitel können auch in separaten Dateien stehen,
% die dann mit dem \include-Befehl eingebunden werden.
%-------------------------------------------------------------------------------

\chapter{Technische Produktumgebung}

In diesem Kapitel wird die technische Umgebung des Produktes beschrieben. Bei
Client / Server  Anwendungen ist die Umgebung jeweils für Client und Server
getrennt anzugeben.

\section{Software}
Als Server-Betriebssystem werden wir Linux verwenden. Für das Clientbetriebssystem, ist allerdings ist jedes Browserfähiges Gerät vorgesehen.Wie gefordert  werden wir für die Framesworks wir Django benutzen und als Programmiersprache Python 2.6+. Unsere Datenbank wird mit SQLite entworfen.

\begin{itemize}
\item Server-Betriebssystem:              Linux
\item Client-Betriebssystem:                Browser
\item Frameworkprogramm :                Django
\item Programmiersprache:                  Python 2.6+
\item Zur Erstellung der Datenbank:   SQLite
\end{itemize}



\section{Hardware}
Als Server ist ein PC gedacht, da die Datenbank nicht übermäßig groß ausfallen wird zum anderen kann die Website von jedem browserfähigem Gerät aus genutzt werden.


\begin{itemize}
\item Server: PC
\item Client: Computer
\end{itemize}

\section{Orgware}

Durch Internetverbindung zum WebsServer soll man sich über die Dokumente in der Bibliothek informieren können, und 
diese ausleihen oder je nach Rolle anderweitig Verwalten. Die Verbindung zum Webserver wird per https  gesichert. 



\section{Produktschnittstellen}

Es wird drei verschiedene Schnittstellen geben, wie zum einen für Allegro um der Uni-Bibliothek zu ermöglichen unsere Daten zu verwerten.
Eine weitere zum importieren von BibTeX-Dateien und eine letzte für den WikipediaBookLocator und Links und Bilder von Bücher zu importieren.

\begin{itemize}
\item Allegro - zum importieren der Daten aus der alten Datenbank, und der Möglichkeit der Uni-Bibliothek unsere Daten zu verwerten
\item BibTex - zum importieren von BibTex Dateien
\item WikipediaBookLocator - um Links und Bilder importieren zu können
\end{itemize}


