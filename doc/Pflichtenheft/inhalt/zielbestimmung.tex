% Kapitel 1
% Die Unterkapitel können auch in separaten Dateien stehen,
% die dann mit dem \include-Befehl eingebunden werden.
%-------------------------------------------------------------------------------

\chapter{Zielbestimmung}
Das Ziel des Projektes ist es, ein System zur Verwaltung einer Bibliothek und der entsprechenden Datenbanken für eine Benutzergruppe, die sich gegenseitig vertraut, zu erstellen. Die Benutzereingaben sollen über ein gut strukturiertes und für Nicht-Informatiker komfortables Webinterface stattfinden. Außerdem soll es verschiedene Möglichkeiten zum Import und Export von Buchdaten geben. 

\section{Musskriterien}
Die primäre Funktion ist die Darstellung und Verwaltung von Buchinformationen (dazu gehört auch der momentane Status innerhalb der Bibliothek) in einer Bibliothek über ein Webinterface. Dazu ist zwingend eine gut funktionierende Suchfunktion erforderlich.

Es muss verschiedene Rollen geben:
\begin{itemize}
  \item Den Gast, der Bücher suchen und sich Informationen über sie anzeigen lassen kann.
  \item Den normalen Nutzer, der alle Funktionen, die zum allgemeinen Gebrauch nötig sind, nutzen kann, wie z.B. ein Buch ausleihen, zurückgeben, an andere übertragen, \gls{glos:BibTeX} Export von Buchinformationen erhalten, eine Liste aller momentan entliehenen Bücher anzeigen lassen etc.
  \item Die Verwaltung, die neue Benutzer anlegen kann.
  \item Den Bibliothekar, der erweiterte Editierrechte für Bücher erhält, wie Bücher importieren, als verloren markieren, anlegen von bestellten Büchern etc.
\item Den Administrator, der (technische) Systemeinstellungen ändern kann.
\end{itemize}

Das Entleihen eines Buches durch Gäste muss über einen normalen Benutzer möglich sein, der das Buch im Namen des Gastes entleiht und dabei dessen Daten angibt. 

Es sind die standardbibliographischen Informationen wie Titel, Autor etc., sowie die Kategorie eines Dokumentes, also ob es ein Buch, eine Diplom-/Masterarbeit, eine Dissertation oder ein Konferenzband ist, sowie mehrere dynamische Inhalte zu speichern. Außerdem ist für jedes Buch eine Historie anzulegen. 

Das Layout muss an das \gls{glos:copdes} der Technischen Universität Braunschweig angepasst sein, sowie eine klare und übersichtliche Struktur haben. 

Es muss intern der \gls{glos:unicode} zur Codierung von Zeichen benutzt werden. Außerdem ist der Import von \gls{glos:BibTeX} sowie ein Export zu \gls{glos:BibTeX} und der \gls{UB}, welche das Format \gls{glos:Allegro} benutzt, einzubauen. 

Die Anmeldung muss über ein sicheres Protokoll erfolgen.

\section{Wunschkriterien}
Zur Verbesserung der Funktion der Rollen kann eingebaut werden, dass Rollen erst auf Anfrage aktiviert werden und die Rechte von Rollen flexibel eingestellt werden können. 

In der Buchsuche wäre ein Filter nach dem Vorbild des \gls{glos:thmefi} wünschenswert. 

Es kann eine Funktion erstellt werden, die ab einer bestimmten Ausleihdauer automatische Erinnerungsmails verschickt. Im Falle einer Entleihung durch Gäste wird diese Nachricht sowohl an den Gast als auch an den normalen Benutzer, der den Entleihvorgang vorgenommen hat, versendet. 

Eine Authentifizierung über \gls{LDAP} wäre denkbar.

\section{Abgrenzungskriterien}
Es wird kein automatischer Datenaustausch mit der \gls{UB} eingerichtet. Daten für die Bibliothek werden lediglich in einer Datei gespeichert, die der Bibliothek auf normalen Datentransferwegen wie E-Mail übermittelt werden kann. 

Das System wird nicht darauf ausgelegt, mehr als 100 Benutzer oder mehr als 5000 Dokumente zu verwalten.

Die Systemsprache ist ausschließlich deutsch.

Bei Importfehlern, die durch andere Zeichencodierungen verursacht werden, wird lediglich ein Zeichen gesetzt, dass der Datensatz fehlerhaft ist. Die Korrektur muss manuell durch einen Benutzer durchgeführt werden.

Es wird kein spezielles Layout für mobile Endgeräte erstellt.
