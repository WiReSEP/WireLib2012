% Kapitel 4
%-------------------------------------------------------------------------------
%% TODO: Literatur durch Dokument ersetzen.

\chapter{Produktfunktionen}

\section{Allgemeine Funktionen}
\subsection{F100 (Anmeldung)}
\label{F:Anmeldung}
\begin{description}
  \item[Geschäftsprozess]Anmeldung
    %% TODO: Meldet sich der Gastnutzer überhaupt an?
  \item[Ziel]Ein Benutzer meldet sich erfolgreich am System an.
  \item[Vorbedingung]Der Benutzer ist bereits auf dem System registriert und besitzt ein Passwort.
  \item[Nachbedingung Erfolg]Der Benutzer findet sich auf seiner Benutzerseite wieder.
  \item[Nachbedingung Fehlschlag]Der Benutzer bekommt eine Fehlermeldung.
  \item[Akteure]Gast/Benutzer
  \item[Auslösendes Ereignis]Der Benutzer erhält die Möglichkeiten sich Literatur zu entleihen oder den Status geliehener Literatur zu ändern.
  \item[Beschreibung]
    \begin{enumerate}
      \item Der Benutzer wählt den Link „Login“ auf der Webseite aus.
      \item Der Benutzer gibt seinen Nutzernamen und sein Passwort ein.
      \item Der Benutzer bestätigt seine Eingaben.
      \item Das System prüft die Korrektheit der Eingaben.
      \item Das System leitet den Benutzer entsprechend der Korrektheit seiner Daten weiter.
    \end{enumerate}
\end{description}

\subsection{\emph{optional:} F101 (Anmeldung über AFS)}
\label{F:AnmeldungAFS}
\begin{description}
  \item[Geschäftsprozess]Login eines Mitgliedes der Universität.
    %% TODO: Meldet sich der Gastnutzer überhaupt an?
  \item[Ziel]Ein Benutzer meldet sich erfolgreich am System an.
  \item[Vorbedingung]Der Benutzer besitzt einen AFS-Account und besitzt sein Passwort.
  \item[Nachbedingung Erfolg]Der Benutzer findet sich auf einer persönlichen Benutzerseite wieder.
  \item[Nachbedingung Fehlschlag]Der Benutzer bekommt eine Fehlermeldung.
  \item[Akteure]Gast/Benutzer
  \item[Auslösendes Ereignis]Der Benutzer erhält die Möglichkeiten sich Literatur zu entleihen oder den Status geliehener Literatur zu ändern.
  \item[Beschreibung]
    \begin{enumerate}
      \item Der Benutzer wählt den Link „Login“ auf der Webseite aus.
      \item Der Benutzer gibt seinen Nutzernamen und sein Passwort ein.
      \item Der Benutzer bestätigt seine Eingaben.
      \item Das System prüft die Korrektheit der Eingaben indem sie über das AFS-System der TU validiert werden.
      \item Das System leitet den Benutzer entsprechend der Korrektheit seiner Daten weiter.
    \end{enumerate}
\end{description}

\subsection{F102 (Anbindung an die UB)}
\begin{description}
  \item[Geschäftsprozess]Übertragung der Institutsbibliothek an die Universitätsbibliothek.
  \item[Ziel]Angleichung der im UB-Allegro verfügbaren Daten über die Institutsbibliothek.
  \item[Vorbedingung]Keine.
  \item[Nachbedingung Erfolg]Der Benutzer erhält eine Datei mit den Daten im Allegro-Format, die er an die UB per Mail verschicken kann.
  \item[Nachbedingung Fehlschlag]Das System soll die Export-Funktion nur berechtigten Benutzern bieten, damit soll ein Fehlschlag in der Bedienung abgefangen werden. Sollte der Vorgang des Parsens einen Fehler erzeugen (z.\,B. durch eine gesperrte Datenbank), erhält der Benutzer die Bitte es später noch einmal zu versuchen.
  \item[Akteure]Benutzer
  \item[Beschreibung]
    \begin{enumerate}
      \item Der Benutzer fordert über das Webinterface die Allegro-Datei an.
      \item Das System überträgt die Literatureinträge der Datenbank in das Allegro-Format.
      \item Das System gibt dem Benutzer eine Datei zum Download, die der Benutzer manuell an die UB versenden kann.
    \end{enumerate}
  \item[Erweiterung]
    \begin{itemize}
      \item \emph{optional} Das System übernimmt das versenden der E-Mail.
      \item \emph{optional} Das System versendet die Datei automatisch nachdem ein Buch zum System hinzugefügt wurde.
    \end{itemize}
  \item[Alternativen]
\end{description}


\section{Literaturfunkionen}
Die Literaturfunkionen beschreiben jeden direkten Umgang mit der durch das System verwalteten Literatur. Darunter ist die \emph{Einpflege} (F201-F202), \emph{Suche} (F211-F219), \emph{Ausleihe} (F221-F229) und der \BibTeX -Export (F231-F239).
\subsection{F201 (\BibTeX-Import)}
\begin{description}
  \item[Geschäftsprozess]Literatur hinzufügen mit \BibTeX.
  \item[Ziel]Eine neu in die Bibliothek aufgenommene Literatur in die Verwaltung aufnehmen und zur Ausleihe verfügbar machen.
  \item[Vorbedingung]Der Benutzer hat sich bereits erfolgreich Angemeldet und hat die Rolle \emph{Bibliothekar}. Der Benutzer ist im Besitz einer \BibTeX -Datei mit mindestens einer einzutragenen Literatur.
  \item[Nachbedingung Erfolg]Die neue Literatur befindet sich im System (inkl. Eintrag in der Datenbank) und besitzt entweder den Status \emph{verfügbar} oder \emph{bestellt}
  \item[Nachbedingung Fehlschlag]Der Benutzer erhält eine Meldung zum Fehlschlag und den Hinweis die \BibTeX -Datei zu überprüfen. Es findet kein Eintrag in die Datenbank statt.
  \item[Akteure]Benutzer/Bibliothekar
  \item[Auslösendes Ereignis]Der Benutzer bekommt eine Übersicht zur neu eingetragenen Literatur.
  \item[Beschreibung]
    \begin{enumerate}
      \item Der Benutzer befindet sich auf der Seite zur \BibTeX einpflege.
      \item Der Benutzer wählt eine \BibTeX -Datei von seinem Rechner.
      \item Der Benutzer wählt den späteren Status der in der Datei enthaltenen Literatur.
      \item Der Benutzer startet einen Upload in das System.
      \item Das System prüft die Datei auf Validität. Dabei ist vor allem zu beachten, dass Umlaute und Sonderzeichen in ein UTF-8 Format übersetzt werden und die Einträge valides UTF-8 sind.
	(Validität schließt sowohl den \BibTeX -Umfang ein, wie auch die Felder: \emph{DateOfPurchase}, \emph{price}, \emph{InventarNo}, \emph{InformatikBibNo}, \emph{LibraryOfCongressNo}, \emph{CIPNo}, \emph{XtraNo}, \emph{keywords}.)
	Das System prüft ebenfalls, ob ein entsprechender Eintrag bereits vorhanden ist.
      \item Das System trägt die Literatureinträge in die Datenbank ein.
      \item Das System fügt den Literatureinträgen den vom Benutzer bestimmten Status hinzu.
      \item Das System macht die Daten verfügbar.
      \item Das System leitet den Benutzer zu einer Übersicht weiter.
    \end{enumerate}
  \item[Erweiterung]Der Benutzer hat die Möglichkeit nicht vorher spezifizierte Felder in die \BibTeX -Datei einzutragen.
  \item[Alternativen]Der Benutzer registriert die Literatur über ein rein Webinterface.
\end{description}

\subsection{F202 (Webinterface-Import)}
\label{F:Web-Import}
\begin{description}
  \item[Geschäftsprozess]Literatur hinzufügen über das Webinterface.
  \item[Ziel]Eine neu in die Bibliothek aufgenommene Literatur in die Verwaltung aufnehmen und zur Ausleihe verfügbar machen.
  \item[Vorbedingung]Der Benutzer hat sich bereits erfolgreich Angemeldet und dat die Rolle \emph{Bibliothekar}. Der Benutzer ist im Besitz aller Daten die zur Eintragung der Literatur erforderlich ist.
  \item[Nachbedingung Erfolg]Die neue Literatur befindet sich im System (inkl. Eintrag in der Datenbank) und besitzt entweder den Status \emph{verfügbar} oder \emph{bestellt}.
  \item[Nachbedingung Fehlschlag]Der Benutzer erhält eine Meldung zum Fehlschag und den Hinweis die eingegebenen Daten zu prüfen. Es findet kein Eintrag in die Datenbank statt.
  \item[Akteure]Benutzer/Bibliethekar
  \item[Auslösendes Ereignis]Der Benutzer bekommt eine Übersicht zur neu eingetragenen Literatur.
  \item[Beschreibung]
    \begin{enumerate}
      \item Der Benutzer befindet sich auf der Seite zur Einpflege über das Webinterface.
      \item Der Benutzer wählt den Typ der Literatur und erhält alle Eingabefelder die für diese Literatur notwendig und optional sind. Das System stellt die Basisfelder, die in \BibTeX\ zur Verfügung stehen wie auch die Felder: \emph{DateOfPurchase}, \emph{price}, \emph{InventarNo}, \emph{InformatikBibNo}, \emph{LibraryOfCongressNo}, \emph{CIPNo}, \emph{XtraNo}, \emph{keywords}.
      \item Der Benutzer bestätigt seine Eingaben.
      \item Das System prüft die Eingaben auf Validität und vorhandene Einträge.
      \item Das System trägt die Literatureinträge in die Datenbank ein.
      \item Das System fügt den Literatureinträgen den vom Benutzer bestimmten Status hinzu.
      \item Das System macht die Daten verfügbar.
      \item Das System leitet den Benutzer zu einer Übersicht weiter.
    \end{enumerate}
    %% TODO:
  \item[Erweiterung]Der Benutzer hat die Möglichkeit nicht vorher spezifizierte Felder selbst einzufügen.
  \item[Alternativen]Der Benutzer registriert die Literatur über den \BibTeX\ Import.
\end{description}

\subsection{F203 (Editieren von Dokumenten)}
\begin{description}
    %% TODO:
  \item[Geschäftsprozess]Der Eintrag eines Dokumentes wird verändert.
  \item[Ziel]siehe oben.
  \item[Vorbedingung]Das Dokument exestiert im System.
  \item[Nachbedingung Erfolg]Die Änderungen waren zulässig und werden in die Datenbank übernommen.
  \item[Nachbedingung Fehlschlag]Die Änderungen waren nicht zulässig und der Benutzer wird auf den Eingabefehler aufmerksam gemacht.
  \item[Akteure]Bibliothekar oder Administrator
  \item[Beschreibung]
    \begin{enumerate}
      \item Der Benutzer wählt die Option \emph{Dokument editieren}
      \item Das System leitet den Benutzer an eine Seite ähnlich zu \nameref{f202}.
      \item Der Benutzer führt seine Anpassungen durch und bestätigt.
      \item Das System übernimmt die Anpassungen.
    \end{enumerate}
  \item[Erweiterung]
  \item[Alternativen]
\end{description}

\subsection{F204 (Löschen von Dokumenten)}
\label{F:Löschen}
\begin{description}
    %% TODO:
  \item[Geschäftsprozess]Das Löschen eines Dokumentes aus dem System.
  \item[Ziel]Das Dokument ist nicht mehr im System gespeichert wie auch alle weiteren Informationen zu dem Dokument.
  \item[Vorbedingung]Das Dokument ist als \emph{vermisst} markiert.
  \item[Nachbedingung Erfolg]Keine Information zu dem Dokument exestiert mehr im System.
  \item[Nachbedingung Fehlschlag]Der Fehler wird abgefangen und die Datenbank wird nicht verändert.
  \item[Akteure]Administrator
  \item[Beschreibung]
    \begin{enumerate}
      \item Der Administrator wählt auf der Dokumentenseite die Option \emph{Dokument löschen}.
      \item Das System bittet den Administrator um Bestätigung.
      \item Der Administrator bestätigt den Löschvorgang.
      \item Das System entfernt alle Datenbankeinträge die zu dem Dokument gehören.
    \end{enumerate}
  \item[Erweiterung]
  \item[Alternativen]
\end{description}

\subsection{F211 (Generelle Suche)}
\label{F:Suche}
\begin{description}
  \item[Geschäftsprozess]Der Benutzer sucht im System einen bestimmten Literatureintrag.
  \item[Ziel]Der Benuter bekommt alle seiner Suche entsprechenden Einträge.
  \item[Vorbedingung]Der Benutzer befindet sich auf der Seite mit der Suchmaske.
  \item[Nachbedingung Erfolg]Der Benutzer bekommt eine Liste aller Suchergebnisse.
  \item[Nachbedingung Fehlschlag]Es konnte keine Literatur der Suche entsprechend gefunden werden. Es wird ein Hinweis ausgegeben.
  \item[Akteure]Benutzer
  \item[Auslösendes Ereignis]Die Suche wird gespeichert um später darauf wieder zurück greifen zu können.
  \item[Beschreibung]
    \begin{enumerate}
      \item Der Benuter gibt seine Suchwörter ein. Dabei kann der Benutzer mehrere Worte verwenden die entsprechned in den Literatureinträgen enthalten sind (Die Reihenfolge ist nicht relevant). Mehrere Wörter werden automatisch logisch mit \emph{und} verknüpft.
      \item Der Benutzer bestätigt seine Eingabe.
      \item Das System durchsucht seine Datenbank um Literatureinträge mit den eingegebenen Teilworten zu finden.
      \item Das System gibt eine Seite zurück die alle passenden Einträge anzeigt.
    \end{enumerate}
    %% TODO:
    \item[Erweiterung]Erweiterungen der Suche sind F212 und F213
\end{description}


\subsection{F212 (Suche mit erweiterten Ausdrücken)}
\label{F:extSuche}
\begin{description}
  \item[Geschäftsprozess]Der Benutzer kann bei der Suche Teilworte verwenden.
  \item[Ziel]Der Benutzer kann komplexe Mittel verwenden um nach möglichen Zeichenfolgend zu suchen.
  \item[Vorbedingung]Der Benutzer befindet sich auf der Seite mit der Suchmaske.
  \item[Nachbedingung Erfolg]Der Benutzer bekommt eine Liste aller Suchergebnisse.
  \item[Nachbedingung Fehlschlag]Es konnte keine Literatur der Suche entsprechend gefunden werden oder wurde kein valider regulärer Ausdrück angegeben. Es wird ein Hinweis ausgegeben.
  \item[Akteure]Benutzer
  \item[Auslösendes Ereignis]Die Suche wird gespeichert um später darauf wieder zurück greifen zu können.
  \item[Beschreibung]
    \begin{enumerate}
      \item Der Benuter gibt seine Ausdruck ein. 
	\begin{itemize}
	  \item Einzelne Ausdrücke werden durch ein Leerzeichen getrennt, um der Trennung zu entgehen müssen die Ausdrücke mit einfachen oder doppelten Hochkommata zu einem Ausdruck zusammen gefasst werden. 
	  \item Mehrere Ausdrücke können mit \emph{and} und \emph{or} oder mit \emph{und} und \emph{oder} verknüpft werden.
	  \item Bestimmte Felder können direkt angesprochen werden, darunter Titel und Autor. Angesprochen werden die Felder mit z.\,B. „title:``Tim und Struppi''“.
	\end{itemize}
      \item Der Benutzer bestätigt seine Eingabe.
      \item Das System durchsucht seine Datenbank um Literatureinträge mit den eingegebenen Teilworten zu finden.
      \item Das System gibt eine Seite zurück die alle passenden Einträge anzeigt.
    \end{enumerate}
\end{description}


\subsection{\emph{optional:} F213 (Suche mit regulären Ausdrücken)}
\begin{description}
  \item[Geschäftsprozess]Der Benutzer kann bei der Suche reguläre Ausdrücke verwenden.
  \item[Ziel]Der Benutzer kann komplexe Mittel verwenden um nach möglichen Zeichenfolgend zu suchen.
  \item[Vorbedingung]Der Benutzer befindet sich auf einer extra Seite für eine erweiterte Suche.
  \item[Nachbedingung Erfolg]Der Benutzer bekommt eine Liste aller Suchergebnisse.
  \item[Nachbedingung Fehlschlag]Es konnte keine Literatur der Suche entsprechend gefunden werden oder wurde kein valider regulärer Ausdrück angegeben. Es wird ein Hinweis ausgegeben.
  \item[Akteure]Benutzer
  \item[Auslösendes Ereignis]Die Suche wird gespeichert um später darauf wieder zurück greifen zu können.
  \item[Beschreibung]
    \begin{enumerate}
      \item Der Benuter gibt seine regulären Ausdrücke ein, die für die Suche verwendet werden sollen. Mehrere reguläre Ausdrücke müssen evtl. mit \emph{and} oder \emph{or} getrennt werden.
      \item Der Benutzer bestätigt seine Eingabe.
      \item Das System durchsucht seine Datenbank um Literatureinträge mit den eingegebenen Teilworten zu finden.
      \item Das System gibt eine Seite zurück die alle passenden Einträge anzeigt.
    \end{enumerate}
\end{description}

\subsection{F214 (Sortierung der Suche)}
\begin{description}
  \item[Geschäftsprozess]Der Benutzer ordnet sein Suchergebnis nach gewünschter Spalte.
  \item[Ziel]Das System ordnet das Suchergebnis.
  \item[Vorbedingung]Der Benutzer hat bereits seine Suchergebnisse.
  \item[Nachbedingung Erfolg]Das Ergebnis wird nach der vom Benutzer gewünschten Spalte sortiert.
  \item[Nachbedingung Fehlschlag]Dieser Fall sollte im System nicht möglich sein.
  \item[Aktuere]Benutzer
  \item[Auslösendes Ereignis]Aktualisierung der Seite.
  \item[Beschreibung]
    \begin{enumerate}
      \item Der Benutzer klickt auf die gekennzeichnete Spaltenüberschrift.
      \item Lokal wird ihm die Suche aktualisiert.
    \end{enumerate}
\end{description}

\subsection{F221 (Ausleihe)}
\begin{description}
  \item[Geschäftsprozess]Der Benutzer entnimmt Literatur der Bibliothek.
  \item[Ziel]Es gibt eine Übersicht für andere Benutzer welche Bücher verfügbar sind. Es gibt auch eine Übersicht, welche Literatur der Benutzer selbst ausgeliehen hat.
  \item[Vorbedingung]Der Benutzer ist im System registiert und angemeldet.
  \item[Nachbedingung Erfolg]Der Benutzer kann eine Literatur der Bibliothek entnehmen, da es im System registiert ist.
  \item[Nachbedingung Fehlschlag]Das Buch ist wahrscheinlich derzeit nicht verfügbar, der Benutzer muss es später nocheinmal versuchen.
  \item[Akteure]Benutzer
  \item[Auslösendes Ereignis]Die Literatur ändert ihren Status von \emph{verfügbar} auf \emph{verliehen}. Der Benutzer sieht bei dieser Literatur einen Hinweis, dass er sie ausgeliehen hat, zusätzlich zu einer persönlichen Ausleihliste.
  \item[Beschreibung]
    \begin{enumerate}
      \item Der Benutzer sucht die gewünschte Literatur über die Systemsuche.
      \item Der Benutzer befindet sich auf der Detail-Ansicht der Literatur.
      \item Wenn die Literatur verfügbar ist wählt der Benutzer die Option \emph{ausleihen}.
      \item Das System ändert den Status der Literatur auf \emph{verliehen} und registriert den Benutzer entsprechend.
    \end{enumerate}
\end{description}

\subsection{F222 (Ausleihe externe)}
\begin{description}
  \item[Geschäftsprozess]Ein Benutzer die nicht dem Institut angehört, leiht sich ein Dokument.
  \item[Ziel]Der Benutzer kann sich das Dokument für die Dauer der Leihfrist mitnehmen.
  \item[Vorbedingung]Der Benutzer hat einen \emph{Verwalter} als Bürgen und das Dokument ist verfügbar.
  \item[Nachbedingung Erfolg]Der Benutzer kann das Dokument der Bibliothek entnehmen und sowohl auf der Liste der geliehenen Dokumente des Bürgen wie auch auch auf der Seite des Dokumentes findet sich der Name Benutzers. Auf der Seite des Dokumentes ist weiter der Name des Bürgen zu entnehmen.
  \item[Nachbedingung Fehlschlag]Das Buch ist wahrscheinlich derzeit nicht verfügbar, der Benutzer muss es später nocheinmal versuchen.
  \item[Akteure]Institutsfremder Benutzer.und ein Verwalter
  \item[Beschreibung]
    \begin{enumerate}
      \item Der Benutzer bittet einen \emph{Verwalter} als Bürgen für ihn das Dokument zu entleihen.
      \item Der Bürge wählt das Dokument aus und wählt dort die Option \emph{leihen für externe Person}.
      \item Der Bürge trägt die Personalien (Name, E-Mail und Adresse) sowie die Dauer der Leihfrist ein.
      \item Der Bürge bestätigt und das System trägt die Daten in die Datenbank ein.
    \end{enumerate}
  \item[Erweiterung]
  \item[Alternativen]
\end{description}

\subsection{F223 (Ausleihe übertragen)}
\begin{description}
  \item[Geschäftsprozess]Ein Benutzer $A$ gibt eine Literatur an eine an einen anderen Benutzer $B$ weiter.
  \item[Ziel]Der Benutzer $A$ registiert einen anderen Benutzer $B$ als Leihender.
  \item[Vorbedingung]Der Benutzer $A$ ist derzeit Leihender der Literatur.
  \item[Nachbedingung Erfolg]Der Benutzer $B$ ist Leihender der Literatur.
  \item[Nachbedingung Fehlschlag]Der Benutzer $A$ bleibt Leihender der Literatur, es wird nichts an der Datenbank geändert.
  \item[Akteure]Leihender Benutzer und zukünftig leihender Benutzer.
  \item[Beschreibung]
    \begin{enumerate}
      \item Der Benutzer $A$ wählt die Literatur aus.
      \item Der Benutzer $A$ wählt die Option \emph{übertragen} aus.
	%%TODO oder wählt ihn über ein Dropdown aus?
      \item Der Benutzer $A$ gibt Benutzer $B$ ein.
      \item Das System ändert die Ausleihe auf Benutzer $B$.
    \end{enumerate}
  \item[Erweiterung]
  \item[Alternativen]Der Benutzer $B$  überträgt die Literatur von Benutzer $A$ auf sich.
\end{description}

\subsection{F224 (Ausleihe zurückgeben)}
\begin{description}
  \item[Geschäftsprozess]Der Benutzer bringt eine Literatur zurück in die Bibliothek.
  \item[Ziel]Die Literatur ist wieder \emph{verfügbar}.
  \item[Vorbedingung]Der Benutzer ist Leihender der Literatur.
  \item[Nachbedingung Erfolg]Der Benutzer ist nicht mehr Leihender der Literatur und die Literatur ist wieder \emph{verfügbar}
  \item[Nachbedingung Fehlschlag]Der Benutzer wird über den Fehler informiert und wird gebeten, es später erneut zu versuchen.
  \item[Akteure]Benutzer des Institutes.
  \item[Beschreibung]
    \begin{enumerate}
      \item Der Benutzer wählt die Literatur aus.
      \item Der Benutzer wählt auf der Literaturseite die Rückgabe aus.
      \item Das SysEintem ändert den Status der Literatur und verschiebt den Benutzer in die History der Leihenden.
    \end{enumerate}
  \item[Erweiterung]Externe Benutzer können ihr Buch an jede Person zurück geben und von denen Entsprechend als Zurückgegeben markieren lassen.
  \item[Alternativen]
\end{description}

\subsection{F225 (Ausleihe verlohren melden)}
\begin{description}
    %%TODO: 	- email versenden?
    %%		- Schwarzes Brett der Verlohrenen Dokumente?
  \item[Geschäftsprozess]Ein Dokument das ausgeliehen wurde oder als Verfügbar gilt, wird als verlohren gemeldet.
  \item[Ziel]Nicht mehr auffindbare Dokumente können allgemein als Vermisst gemeldet werden.
  \item[Vorbedingung]Das Dokument ist bereits im System registiert.
  \item[Nachbedingung Erfolg]Das Dokument gilt offiziell als Vermisst.
  \item[Nachbedingung Fehlschlag]Der Fehler wird vom System an den Benutzer gemeldet.
  \item[Akteure]Benutzer
  \item[Beschreibung]
    \begin{enumerate}
      \item Der Benutzer befindet sich auf der Detailseite des Dokumentes
      \item Der Benutzer wählt eine Option \emph{Vermisst melden}.
      \item Das System registiert den Vorgang.
    \end{enumerate}
  \item[Erweiterung]
    \begin{itemize}
      \item \emph{optional:} Ein allgemeines Blackboardsystem mit vermissten Dokumenten auf der Bibliotheksseite.
      \item \emph{optional:} Das System versendet einen E-Mail an eine Mailingliste, dass das Buch als vermisst gemeldet wurde.
    \end{itemize}
  \item[Alternativen]
\end{description}

\subsection{F226 (Ausleihfrist abgelaufen)}
\begin{description}
  \item[Geschäftsprozess]Die Leihfrist eines Dokumentes ist abgelaufen.
  \item[Ziel]Sowohl der Bürge als auch die Person, die das Dokument geliehen hat, werden per E-Mail noch einmal erinnert.
  \item[Vorbedingung]Die Person hat das Dokument geliehen und die Leihfrist ist abgelaufen.
  \item[Nachbedingung Erfolg]Beide Parteien sind via E-Mail nocheinmal erinnert worden.
  \item[Nachbedingung Fehlschlag]
    \begin{itemize}
      \item Die E-Mails konnten nicht verschickt werden, der Administrator wird benachrichtigt.
      \item Die E-Mail an die Person, die das Dokument geliehen hat, ist nicht korrekt oder konnte nicht zugestellt werden. Damit wird der Bürge benachrichtigt.
    \end{itemize}
  \item[Akteure]Das System
  \item[Beschreibung]
    \begin{enumerate}
      \item Das System prüft täglich ob eine Leihfrist abgelaufen ist.
      \item Das System stellt fest, dass eine Leihfrist abgelaufen ist.
      \item Das System sendet an beide Parteien eine E-Mail mit den Informationen auch zu dem Dokument selbst.
    
    \end{enumerate}
  \item[Erweiterung]
  \item[Alternativen]
\end{description}

\subsection{F227 (Ausleihhistory)}
\begin{description}
    %% TODO Sichtbarkeit optional?
  \item[Geschäftsprozess]
  \item[Ziel]
  \item[Vorbedingung]
  \item[Nachbedingung Erfolg]
  \item[Nachbedingung Fehlschlag]
  \item[Akteure]
  \item[Beschreibung]
  \item[Erweiterung]
  \item[Alternativen]
\end{description}

\subsection{F228 (Entleihliste)}
\begin{description}
    %% TODO:
  \item[Geschäftsprozess]
  \item[Ziel]
  \item[Vorbedingung]
  \item[Nachbedingung Erfolg]
  \item[Nachbedingung Fehlschlag]
  \item[Akteure]
  \item[Beschreibung]
  \item[Erweiterung]
  \item[Alternativen]
\end{description}

\subsection{F231 (\BibTeX -Export)}
\begin{description}
    %% TODO:
  \item[Geschäftsprozess]
  \item[Ziel]
  \item[Vorbedingung]
  \item[Nachbedingung Erfolg]
  \item[Nachbedingung Fehlschlag]
  \item[Akteure]
  \item[Beschreibung]
  \item[Erweiterung]
  \item[Alternativen]
\end{description}


\section{Rollen}
\begin{description}
  \item[F301-309]Rechte
  \item[F311-319]Rollenadministration
  \item[F321-329]Rollenzuweisungen
\end{description}

\subsection{F301 (Literaturimport)}
\begin{description}
    %% TODO:
  \item[Geschäftsprozess]
  \item[Ziel]
  \item[Vorbedingung]
  \item[Nachbedingung Erfolg]
  \item[Nachbedingung Fehlschlag]
  \item[Akteure]
  \item[Beschreibung]
  \item[Erweiterung]
  \item[Alternativen]
\end{description}

\subsection{F302 (Literaturausleihe externer)}
\begin{description}
    %% TODO:
  \item[Geschäftsprozess]
  \item[Ziel]
  \item[Vorbedingung]
  \item[Nachbedingung Erfolg]
  \item[Nachbedingung Fehlschlag]
  \item[Akteure]
  \item[Beschreibung]
  \item[Erweiterung]
  \item[Alternativen]
\end{description}

\subsection{F303 (Benutzerverwaltung)}
\begin{description}
    %% TODO:
  \item[Geschäftsprozess]
  \item[Ziel]
  \item[Vorbedingung]
  \item[Nachbedingung Erfolg]
  \item[Nachbedingung Fehlschlag]
  \item[Akteure]
  \item[Beschreibung]
  \item[Erweiterung]
  \item[Alternativen]
\end{description}

\subsection{F304 (Verleihstatus übertragen)}
\begin{description}
    %% TODO:
  \item[Geschäftsprozess]
  \item[Ziel]
  \item[Vorbedingung]
  \item[Nachbedingung Erfolg]
  \item[Nachbedingung Fehlschlag]
  \item[Akteure]
  \item[Beschreibung]
  \item[Erweiterung]
  \item[Alternativen]
\end{description}

\subsection{F305 (UB-Export)}
\begin{description}
    %% TODO:
  \item[Geschäftsprozess]
  \item[Ziel]
  \item[Vorbedingung]
  \item[Nachbedingung Erfolg]
  \item[Nachbedingung Fehlschlag]
  \item[Akteure]
  \item[Beschreibung]
  \item[Erweiterung]
  \item[Alternativen]
\end{description}

\subsection{F306 (Derzeitiger Leihender)}
\begin{description}
    %% TODO:
  \item[Geschäftsprozess]
  \item[Ziel]Ein Benutzer der dieses Recht besitzt kann bei verliehenen Büchern sehen, an wen das Buch verliehen ist.
  \item[Vorbedingung]
  \item[Nachbedingung Erfolg]
  \item[Nachbedingung Fehlschlag]
  \item[Akteure]
  \item[Beschreibung]
  \item[Erweiterung]
  \item[Alternativen]
\end{description}

\subsection{F211 (Rollengruppe Rechte zuwiesen)} 
\begin{description}
    %% TODO:
  \item[Geschäftsprozess]
  \item[Ziel]
  \item[Vorbedingung]
  \item[Nachbedingung Erfolg]
  \item[Nachbedingung Fehlschlag]
  \item[Akteure]
  \item[Beschreibung]
  \item[Erweiterung]
  \item[Alternativen]
\end{description}

\subsection{F212 (Benutzer Rolle(n) zuweisen)}
\begin{description}
    %% TODO:
  \item[Geschäftsprozess]
  \item[Ziel]
  \item[Vorbedingung]
  \item[Nachbedingung Erfolg]
  \item[Nachbedingung Fehlschlag]
  \item[Akteure]
  \item[Beschreibung]
  \item[Erweiterung]
  \item[Alternativen]
\end{description}
