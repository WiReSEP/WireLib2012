% Kapitel 5
%------------------------------------------------------------------------------------------

\chapter{Produktdaten}

\section{/D10/ Nutzer}
\subsection{User}
User werden von der Verwaltung angelegt. Die gespeicherten Daten über sich selbst kann jeder User ändern. Dazu gehören: 
\begin{enumerate}
	\item der Username, 
	\item gegebenenfalls ein Passwort, falls ohne \gls{LDAP} gearbeitet wird,
	\item der Vor- und Nachname, 
	\item die E-Mail-Adresse, 
	\item die Wohnanschrift und 
	\item eine oder mehrere Telefonnummern. 
\end{enumerate} 
Die jeweils zugewiesene Rolle können lediglich von der Verwaltung und vom Administrator verändert werden.

\subsection{Externe}
\gls{glos:ext} können von jedem angemeldeten Benutzer hinzugefügt werden. Ihnen steht es nicht frei ihre eigenen Daten zu ändern. Dies muss auch immer über einen Angemeldeten ablaufen. Gespeichert werden auch hier: 
\begin{enumerate}
	\item Vor- und Nachname, 
	\item die E-Mail-Adresse, 
	\item die Wohnanschrift und 
	\item gegebenenfalls Telefonnummern.
\end{enumerate}

\section{/D20/ Dokumente}
Das Löschen von Dokumenten ist allgemein nicht möglich und deren Hinzufügen erfolgt über den Bibliothekar. Dabei müssen mindestens folgende Daten vorhanden sein:
\begin{enumerate}
	\item die ID,
	\item der Titel,
	\item eventuell mehrere Autoren,
	\item die Kategorie und
	\item der Status.
\end{enumerate}
Im Status wird hierbei gemerkt, ob ein Dokument vorhanden (0), ausgeliehen (1), bestellt (2), vermisst (3) oder verloren gegangen (4) ist. 
Zudem sind weitere Eingabe wie die ISBN, das Erscheinungsjahr und selbst gewählte Informationen möglich. 

\section{/D30/ Verleihdaten}
Zu jedem Dokument wird eine eigene Verleihhistorie erstellt. Da ein Dokument nicht gelöscht werden kann, funktioniert dieses auch nicht mit der Historie. Sie wird aber ständig erweitert. Dabei sollen von jedem Verleihvorgang folgende Daten gespeichert werden:
\begin{enumerate}
	\item das Ausleihdatum,
	\item das Rückgabedatum,
	\item die Entleihfrist,
	\item der Entleiher,
	\item gegebenenfalls einen \gls{glos:ext}n und
	\item das Dokument selber.
\end{enumerate} 
