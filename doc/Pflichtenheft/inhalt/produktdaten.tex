% Kapitel 5
%------------------------------------------------------------------------------------------

\chapter{Produktdaten}
/D10/ Daten über einen Nutzer:
\begin{itemize}
  \item Daten werden von der Verwaltung eingegeben.
  \item Bestandteile:
  \begin{itemize}
    \item[*] User:
    \begin{itemize}
      \item[-] Username
      \item[-] Passwort, falls ohne AFS gearbeitet wird
      \item[-] Vorname
      \item[-] Nachname
      \item[-] E-Mail
      \item[-] Adresse
      \item[-] Telefon
      \item[-] Rolle
    \end{itemize}
    \item[*] Externe:
    \begin{itemize}
      \item[-] dasselbe wie beim User bis auf den Usernamen und die Rolle
    \end{itemize}
  \end{itemize}
\end{itemize}

\newpage

/D20/ Daten über Dokumente:
\begin{itemize}
  \item Dateneingabe erfolgt über den Bibliothekar.
  \item Löschen eines Dokumentes aus der Datenbank ist allgemein nicht möglich.
  \item Bestandteile:
  \begin{itemize}
    \item[*] ID
    \item[*] Titel
    \item[*] Autoren
    \item[*] ISBN
    \item[*] Kategorie
    \item[*] Status (0 = vorhanden, 1 = ausgeliehen, 2 = bestellt, 3 = vermisst, 4 = verloren gegangen)
    \item[*] weitere Eingaben möglich
  \end{itemize}
\end{itemize}

/D30/ Verleihdaten
\begin{itemize}
  \item Zu jedem Dokument wird eine Verleihhistorie erstellt.
  \item Die Daten können nicht gelöscht werden. Sie werden lediglich bei weiteren Verleihungen erweitert.
  \item Bestandteile:
  \begin{itemize}
    \item[*] Ausleihdatum
    \item[*] Rückgabedatum
    \item[*] Entleihfrist
    \item[*] Entleiher
    \item[*] evtl. einen Externen
    \item[*] das Dokument
  \end{itemize}
\end{itemize}