% Kapitel 6
%-------------------------------------------------------------------------------

\chapter{Nichtfunktionale Anforderungen}


%----Start der Tabelle------------------------------------------------------
\begin{tabular}{|c|c|c|c|c|}
  \hline                                              % Linie oberhalb

  %----Spaltenüberschriften------------------------------------------------
  \textbf{Produktqualität}  & \textbf{sehr gut}  &    \textbf{gut}  &
  \textbf{normal}           & \textbf{nicht relevant}  \\  %Spaltenüberschrift
  \hline                                              % Gitterlinie

  %----die nachfolgeden beiden Zeilen so oft wiederholen und die ... mit den
  %    entsprechenden Daten zu füllen wie erforderlich
Funktionalität  &x&&&\\        % Eintrag in Zeile
  \hline
Angemessenheit&&x&&\\
\hline
Richtigkeit&&x&&\\
\hline
Interoperabilität&&x&&\\
\hline
Ordnungsmäßgkeit&&x&&\\
\hline
Sicherheit&&x&&\\
\hline
Zuverlässigkeit&&x&&\\
\hline
Reife&x&&&\\
\hline
Fehlertoleranz&&&x&\\
\hline
Wiederherstellbarkeit&&x&&\\
\hline
Benutzbarkeit&x&&&\\
\hline
Verständlichkeit&x&&&\\
\hline
Erlernbarkeit&x&&&\\
\hline
Bedienbarkeit&x&&&\\
\hline
Effizienz&&&x&\\
\hline
Zeitverhalten&&&x&\\
\hline
Verbrauchsverhalten&&x&&\\
\hline
Änderbarkeit&&x&&\\
\hline
Analysierbarkeit&&x&&\\
\hline
Modifizierbarkeit&x&&&\\
\hline
Stabilität&x&&&\\
\hline
Prüfbarkeit&&x&&\\
\hline
Übertragbarkeit&&&&x\\
\hline
Anpassbarkeit&&&x&\\
\hline
Installierbarkeit&&&&x\\
\hline
Konformität&&&&\\
\hline
Austauschbarkeit&&&x&\\
\hline
%----Ende der Tabelle------------------------------------------------------
\end{tabular}


\begin{itemize}

\item  /Q10/ Der Login über die Funktionen /\nameref{F:Anmeldung}/ und /\nameref{F:AnmeldungAFS}/ sowie Cookies sind verschlüsselt.
\item  /Q11/ SQL-Injections sollen bei der Suche von Dokumenten sowie bei anderen Eingaben nicht möglich sein. Dadurch sollen Sicherheitslücken vermieden werden.
\item  /Q20/ Die Layoutstruktur und Fonts der Webseite sollen ähnlich sein denen des \Gls{glos:copdes} der TU Braunschweig. Dadurch soll die Webseite nicht aus dem einheitlichen Layout der TU Braunschweig herausfallen. Weiterhin soll damit eine weitere Anpassung an das CMS der TU Braunschweig erleichtert werden.
\item  /Q21/ Das Layout der Webseite soll aufgeräumt sein mit klarer Navigation und Schaltflächen. Links sollen als solche markiert werden und nicht mit dem restlichen Text verschmelzen. Ebenso sollen auf der Webseite dargestellte Informationen wie \mbox{z.\,B.}\xspace Autor/Titel eines Dokumentes leicht ersichtlich und übersichtlich angezeigt werden.
\item  /Q22/ Die Bedienung der Funktionen /\nameref{F:Suche}/ und /\nameref{F:extSuche}/ soll intuitiv und einfach verständlich sein. Es soll kein tiefgehendes Wissen notwendig sein, um die Suche zu bedienen.
\item  /Q30/ Intern sollen alle Zeichenketten als \Gls{glos:unicode} bearbeitet werden. Damit werden Zeichensatzinkompatiblitäten vermieden und gleichzeitig alle Sonderzeichen ermöglicht.
\item  /Q31/ Nur als \emph{bestellt} markierte Dokumente dürfen mit der Funktion /\nameref{F:Löschen}/ gelöscht werden. Bereits vorhandene Dokumente dürfen nur als \emph{vermisst} beziehungsweise \emph{verloren} markiert werden. Optional: Administratoren dürfen auch bereits vorhandene Dokumente löschen.

\end{itemize} 