% Kapitel 6
%-------------------------------------------------------------------------------

\chapter{Nichtfunktionale Anforderungen}


%----Start der Tabelle------------------------------------------------------
\begin{tabular}{|c|c|c|c|c|}
  \hline                                              % Linie oberhalb

  %----Spaltenüberschriften------------------------------------------------
  \textbf{Produktqualität}  & \textbf{sehr gut}  &    \textbf{gut}  &
  \textbf{normal}           & \textbf{nicht relevant}  \\  %Spaltenüberschrift
  \hline                                              % Gitterlinie

  %----die nachfolgeden beiden Zeilen so oft wiederholen und die ... mit den
  %    entsprechenden Daten zu füllen wie erforderlich
Funktionalität  &x&&&\\        % Eintrag in Zeile
  \hline
Angemessenheit&&x&&\\
\hline
Richtigkeit&&x&&\\
\hline
Interoperabilität&&x&&\\
\hline
Ordnungsmäßgkeit&&x&&\\
\hline
Sicherheit&&x&&\\
\hline
Zuverlässigkeit&&x&&\\
\hline
Reife&x&&&\\
\hline
Fehlertoleranz&&&x&\\
\hline
Wiederherstellbarkeit&&x&&\\
\hline
Benutzbarkeit&x&&&\\
\hline
Verständlichkeit&x&&&\\
\hline
Erlernbarkeit&x&&&\\
\hline
Bedienbarkeit&x&&&\\
\hline
Effizienz&&&x&\\
\hline
Zeitverhalten&&&x&\\
\hline
Verbrauchsverhalten&&x&&\\
\hline
Änderbarkeit&&x&&\\
\hline
Analysierbarkeit&&x&&\\
\hline
Modifizierbarkeit&&x&&\\
\hline
Stabilität&x&&&\\
\hline
Prüfbarkeit&&x&&\\
\hline
Übertragbarkeit&&&&x\\
\hline
Anpassbarkeit&&&x&\\
\hline
Installierbarkeit&&&&x\\
\hline
Konformität&&&&\\
\hline
Austauschbarkeit&&&x&\\
\hline
%----Ende der Tabelle------------------------------------------------------
\end{tabular}


\begin{itemize}

\item  /Q10/ Der Login über die Funktionen /\nameref{F:Anmeldung}/ und /\nameref{F:AnmeldungAFS}/ sowie Cookies sind verschlüsselt.
\item  /Q11/ SQL-Injections sollen bei der Suche und anderen Eingaben nicht möglich sein.
\item  /Q20/ Die Layoutstruktur und Fonts sind ähnlich denen des CMS der TU BRaunschweig.
\item  /Q21/ Das Layout soll aufgeräumt sein mit klarer Navigation und Schaltflächen.
\item  /Q22/ Informationen sollen übersichtlich dargestellt werden.
\item  /Q23/ Die Funktionen /\nameref{F:Suche}/ und /\nameref{F:extSuche}/ sollen intuitiv und einfach verständlich sein.
\item  /Q30/ Intern sollen alle Zeichenketten als Unicode bearbeitet werden.
\item  /Q31/ Nur bestellte Dokumente dürfen mit der Funktion /\nameref{F:Löschen}/ gelöscht werden. Bereits vorhandene Dokumente dürfen nur als 'fehlend' markiert werden. Optional: Administratoren dürfen auch bereits vorhandene Dokumente löschen.

\end{itemize}
