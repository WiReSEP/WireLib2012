%Diese Datei dient der Versionskontrolle. Sie ist vollst\"andig zu bearbeiten.

%----\"Uberschrift------------------------------------------------------------
{\relsize{2}\textbf{Versions\"ubersicht}}\\[2ex]

%----Start der Tabelle------------------------------------------------------
\begin{longtable}{|m{1.78cm}|m{1.59cm}|m{2.86cm}|m{1.9cm}|m{5.25cm}|}

  \hline                                              % Linie oberhalb

  %----Spalten\"uberschriften------------------------------------------------
  \textbf{Version}  &    \textbf{Datum}  &    \textbf{Autor}  &
  \textbf{Status}   &    \textbf{Kommentar}  \\       %Spalten\"uberschrift
  \hline                                              % Gitterlinie

  %----die nachfolgeden beiden Zeilen so oft wiederholen und die ... mit den
  %i   entsprechenden Daten zu f\"ullen wie erforderlich
  0.1   & 09.07.2012 & Jörn Hameyer     & in Bearbeitung & Einleitung \\
  \hline
  0.2   & 09.07.2012 & Markus Dietrich  & in Bearbeitung & Testplan \\
  \hline
  0.3   & 11.07.2012 & alle Teilnehmer  & in Bearbeitung & Testdokumentation \\
  \hline
  0.4   & 12.07.2012 & alle Teilnehmer  & in Bearbeitung & Testdurchführung \\
  \hline
  0.5   & 12.07.2012 & Johann Hong      & in Bearbeitung & Zusammenfassung \\
  \hline
  0.6   & 12.07.2012 & Stephan Sobol    & in Bearbeitung & Versionsübersicht \\
  \hline
  0.7   & 12.07.2012 & alle Teilnehmer  & in Bearbeitung & Rechtschreib- und Grammatikprüfung \\
  \hline
  1.0   & 12.07.2012 & alle Teilnehmer  & abgenommen     & \\
     % Eintrag in Zeile
  \hline                                              % Gitterlinie unten

%----Ende der Tabelle------------------------------------------------------
\end{longtable}
Status: "`in Bearbeitung"' oder "`abgenommen"'\\
Kommentar: hier eintragen, was ge\"andert bzw. erg\"anzt wurde

