% Kapitel 3 mit den entsprechenden Unterkapiteln
% Die Unterkapitel k\"onnen auch in separaten Dateien stehen,
% die dann mit dem \include-Befehl eingebunden werden.
%----------------------------------------------------------------------------

\chapter{Testdurchf\"uhrung}

%In diesem Abschnitt werden die einzelnen Testf\"alle beschrieben und deren
%Durchf\"uhrungen (=Testl\"aufe) protokolliert.\\
%Ein Testfall ist eine Kombination von Eingabedaten, Bedingungen und erwarteten
%Ausgaben, die einen bestimmten Zweck erf\"ullen. Man pr\"uft z.B., ob Vorgaben in
%einem Spezifikationsdokument eingehalten werden oder ob der Programmablauf
%tats\"achlich dem erwarteten Pfad entspricht.\\
%Dieses Kapitel enth\"alt drei Arten von Tabellen:\\
%1.  Die \"Ubersichtstabelle zeigt an, welche Testf\"alle es gibt und welcher
%Testfall welche Objekte, Methoden oder Anforderungen testet. So hat man den
%\"Uberblick, Verfolgbarkeit zwischen der Testdokumentation und anderen
%Dokumenten, und man kann sehen, ob die Testf\"alle vollst\"andig sind.\\
%2.  Der Testfall beschreibt jeden einzelnen Testfall im Detail.\\
%3.  Der Testlauf beschreibt eine Durchf\"uhrung eines Testfalls. Derselbe
%Testfall kann mit verschiedenen Eingabedaten oder auch mit verschiedenen
%Softwareversionen mehrmals durchgef\"uhrt werden.\\

\section{\"Ubersichtstabelle}
  %In der folgenden Tabelle sind entweder f\"ur alle Testf\"alle die zu testende
  %Komponente oder die zu testende Funktion (oder beides) anzugeben. Die
  %Bezeichnungen der Komponenten m\"ussen konsistent sein mit denen in Fein- und
  %Grobkonzept, um die Verfolgbarkeit zum Konzept sicherzustellen. Die IDs und
  %Bezeichnungen der Funktionen m\"ussen denen im Pflichtenheft entsprechen, um
  %die Verfolgbarkeit zu den Anforderungen sicherzustellen. \\
\begin{tabular}{|c|c|c|}
\hline
\textbf{Testfall ID und Bezeichnung} &  \textbf {Zu testende Komponente} &
\textbf {Zu testende Funktion}\\
\hline
/T212/ Erweiterte Suche &  Views und Template  & /F212/ Erweiterte Suche \\
\hline
/T228/ Derzeitiger Leihender &  Views und Template  &
/F228/ Derzeitiger Leihender\\
\hline
&&
\end{tabular}

\subsection{Testfall -- ID und Bezeichnung}
Jeder Testfall erh\"alt eine eindeutige Identifikation mit Kurzbezeichnung.\\
Beispiel: /T100/ Lager anlegen\\
Die folgende Tabelle beschreibt den Testfall. \\
\begin{longtable}{|p{5cm}|p{10cm}|}
\hline
\textbf{Testfall -- ID und Bezeichnung} &  \textit{Beispiel:
                                                        /T100/ Lager anlegen} \\
\hline
\textbf{Zu testende Objekte und Methoden} &  \textit{Hier sind alle Testobjekte
und Methoden zu beschreiben, die von diesem Testfall ausgef\"uhrt werden.
Testobjekte k\"onnen dabei z.B. auch Komponenten oder einzelne Webseiten sein.}
\\
\hline
\textbf{Kriterien f\"ur erfolgreiche bzw. fehlgeschlagene Testf\"alle} &
\textit{Es sind die Kriterien anzugeben, mit denen man feststellt, dass der
Testfall erfolgreich bzw. fehlgeschlagen ist. } \\
\hline
\textbf{Einzelschritte} &  \textit{Es ist zu beschreiben, was zu tun ist, um
einen Testlauf vorzubereiten und ihn zu starten.
Ggf. sind erforderliche Schritte w\"ahrend seiner Ausf\"uhrung anzugeben (z.B.
Benutzerinteraktion \"uber ein User-Interface). Ferner ist zu beschreiben, was zu
tun ist, um den Testlauf ordnungsgem\"aß oder im Falle unvorhergesehener
Ereignisse anzuhalten (falls er nicht von selbst terminiert).
Ggf. sind Aufr\"aumarbeiten zu beschreiben, um nach den Tests den urspr\"unglichen
Zustand wiederherzustellen (falls der Testlauf nicht seiteneffektfrei ist)
} \\
\hline
\textbf{Beobachtungen / Log} &  \textit{Es sind alle speziellen Methoden oder
Formate zu beschreiben, mit denen die Ergebnisse der Testl\"aufe, die
Zwischenf\"alle und sonstige wichtige Ereignisse aufgenommen werden sollen.
Beispiel: Logdatei eines Servers, Messung der Antwortzeit eines Remote
Terminals mittels Netzwerk Simulator, \ldots} \\
\hline
\textbf{Besonderheiten } &  \textit{optional; auszuf\"ullen, falls es
Besonderheiten in diesem Testfall gibt.
Testfallspezifische Besonderheiten, z.B. Ausf\"uhrungsvorschriften oder
Abweichungen von der Testumgebung (siehe 2.5)  werden hier aufgelistet.} \\
\hline
\textbf{Abh\"angigkeiten} &  \textit{optional; auszuf\"ullen, falls es
Abh\"angigkeiten in diesem Testfall gibt
Ist dieser Testfall von der Ausf\"uhrung anderer Testf\"alle abh\"angig, so werden
diese Testf\"alle hier aufgelistet und kurz beschrieben, worin die Abh\"angigkeit
besteht.} \\
\hline

 \end{longtable}

Die folgenden Tabellen beschreiben, wie der Testfall ausgef\"uhrt wurde und
welches Ergebnis er geliefert hat. Da es bei Korrektur von Softwarefehlern oder
anderen Gegebenheiten notwendig ist, einen Test mehrfach durchzuf\"uhren
(Testl\"aufe), ist jede Testdurchf\"uhrung zu dokumentieren. Daher ist diese
Tabelle f\"ur \textbf{jeden Testlauf }  zu erstellen und \textbf{ fortlaufend zu
nummerieren}. \\


\begin{longtable}{|p{5cm}|p{10cm}|}
\hline
\textbf{Testfall -- ID und Bezeichnung} & \textit{Beispiel: /T100/ Lager
anlegen} \\
\hline
\textbf{Testlauf Nr.} & \textit{Beispiel: 1} \\
\hline
\textbf{Eingaben} & \textit{Es sind alle Eingabedaten bzw. andere Aktionen
aufzuf\"uh-ren, die f\"ur die Ausf\"uhrung des Testfalls notwendig sind.
Diese k\"onnen sowohl als Wert angegeben werden (ggf. mit Toleranzen) als auch
als Name, falls es sich um konstante Tabellen oder um Dateien handelt. Außerdem
sind alle betroffenen Datenbanken, Dateien, Terminal Meldungen, etc. anzugeben.
Hinweis: Es sind nicht noch mal die Einzelschritte aus 3.1.3 zu wiederholen.
W\"ahrend jene allgemeiner sind (z.B. "`Ein-loggen \"uber das Login-Formular"')
sind hier die konkreten eingegebenen Testdaten aufzuf\"uhren (z.B. "`Loginname:
test; Passwort: xxxtest"'`). } \\
\hline
\textbf{Soll - Reaktion} & \textit{Hier ist anzugeben, welches Ergebnis bzw.
Ausgabe der Test haben soll.
Hinweis: Es sind nicht noch mal die Erfolgskriterien aus 3.1.2 zu wiederholen.
W\"ahrend jene allgemeiner sind (z.B. "`Testnachricht wird \"uber Netzwerkkanal
empfangen"') sind hier die konkreten erhaltenen Testdaten aufzuf\"uhren (z.B.
Konsole zeigt Meldung: "`Testnachricht 123 erhalten"').
} \\
\hline
\textbf{Ist -- Reaktion} & \textit{Hier ist anzugeben, welches Ergebnisdaten
bzw. Ausgaben dieser Testlauf geliefert hat.} \\
\hline
\textbf{Ergebnis} & \textit{F\"ur jeden Testlauf ist zu vermerken, ob der Test
erfolgreich durchgef\"uhrt werden konnte oder nicht. Einen missgl\"uckten Test
bitte begr\"unden, sofern der Grund des Fehlschlags bekannt oder offensichtlich
ist.} \\
\hline
\textbf{Unvorhergesehene Ereignisse w\"ahrend des Test-laufs } &
\textit{optional; nur anzugeben, falls es unvorhergesehene Ereig-nisse gab} \\
\hline
\textbf{Nacharbeiten } & \textit{Ist ein Testlauf nicht erfolgreich
durchgef\"uhrt worden, so werden hier die erforderlichen Nacharbeiten aufgef\"uhrt
(z.B. Bugfixes).} \\
\hline
 \end{longtable}


%\subsection{Testfall -- /T103/ Mailtexte ändern}
%Jeder Testfall erh\"alt eine eindeutige Identifikation mit Kurzbezeichnung.\\
%Beispiel: /T100/ Lager anlegen\\
%Die folgende Tabelle beschreibt den Testfall. \\
In der folgenden Tabelle wird der Test zur Funktion E-Mail Änderung protokolliert.
\begin{longtable}{|p{5cm}|p{10cm}|}
\hline
\textbf{Testfall -- ID und Bezeichnung} &  \textnormal{/T103/ Mailtexte ändern} \\
\hline
\textbf{Zu testende Objekte und Methoden} &  \textnormal{
\begin{itemize}
\item In Komponente \textit{Admin} der Bereich E-Mail 
\end{itemize} }
\\
\hline
\textbf{Kriterien f\"ur erfolgreiche bzw. fehlgeschlagene Testf\"alle} &
\textnormal{Erfolgreich: E-Mail Inhalt steht, wie gewünscht, in der Datenbank ;
        Fehlgeschlagen: E-Mail Inhalt steht nicht, oder nicht wie gewünscht in der DB   } \\
\hline
\textbf{Einzelschritte} &  \textnormal{Zunächst wird über das Django-Testing-Framework getestet,
ob der Benutzer angemeldet ist (angemeldet wirft einen Http-Status-Code 200 zurück, sonst 
einen Http-Status-Code 400). Falls dieser angemeldete Benutzer genügend Rechte besitzt,
wiederum zu testen über eine spezielle Django-Testing-Authentification Methode, wird der interne View
für das E-Mail ändern angesteuert (Http-Code=200 erfolgreich, Http-Code=404 gescheitert.).
Falls dies alles erfolgreich sein sollte, kann man nun E-Mail Texte verändern.
} \\
\hline
\textbf{Beobachtungen / Log} &  \textnormal{Ergebnisse werden haptsächlich über 
die Kommandozeile per \textbf{stdout}(gibt Text der E-Mail aus) und \textbf{Django-Testing-
Framework} kontrolliert  } \\
\hline


 \end{longtable}

%Die folgenden Tabellen beschreiben, wie der Testfall ausgef\"uhrt wurde und
%welches Ergebnis er geliefert hat. Da es bei Korrektur von Softwarefehlern oder
%anderen Gegebenheiten notwendig ist, einen Test mehrfach durchzuf\"uhren
%(Testl\"aufe), ist jede Testdurchf\"uhrung zu dokumentieren. Daher ist diese
%Tabelle f\"ur \textbf{jeden Testlauf }  zu erstellen und \textbf{ fortlaufend zu
%nummerieren}. \\

Folgend werden die benötigten Testdurchläufe näher beschrieben
\begin{longtable}{|p{5cm}|p{10cm}|}
\hline
\textbf{Testfall -- ID und Bezeichnung} & \textnormal{/T103/ Mailtexte ändern} \\
\hline
\textbf{Testlauf Nr.} & \textnormal{1} \\
\hline
\textbf{Eingaben} & \textnormal{Ein eingeloggter Nutzer mit entsprechenden Rechten, E-Mail 
Einträge in der Datenbank, Klick auf E-Mail ändern in dem Admin Backend } \\
\hline
\textbf{Soll - Reaktion} & \textnormal{Ausgabe sollte eingegebenen E-Mail
Text enthalten: "`Sehr geehrte Damen und Herren, "').
} \\
\hline
\textbf{Ist -- Reaktion} & \textnormal{stdout lieferte: "`Sehr geehrte Damen und Herren, "'} \\
\hline
\textbf{Ergebnis} & \textnormal{Testlauf Nr.1 gab den erwarteten E-Mail Text zurück, somit erfolgreich.} \\
\hline
 \end{longtable}
 
Dieser Testdurchlauf testet, ob das Text Feld der E-Mail mögliche Sonderzeichen beachtet.
\begin{longtable}{|p{5cm}|p{10cm}|}
\hline
\textbf{Testfall -- ID und Bezeichnung} & \textnormal{/T103/ Mailtexte ändern} \\
\hline
\textbf{Testlauf Nr.} & \textnormal{2} \\
\hline
\textbf{Eingaben} & \textnormal{Ein eingeloggter Nutzer mit entsprechenden Rechten, E-Mail 
Einträge in der Datenbank, Klick auf E-Mail ändern in dem Admin Backend } \\
\hline
\textbf{Soll - Reaktion} & \textnormal{Ausgabe sollte eingegebenen E-Mail
Text enthalten: "`Sehr geehrte Damen und Herren,///!123 
                    ..--\_\_??==))88\&/\& "').
} \\
\hline
\textbf{Ist -- Reaktion} & \textnormal{stdout lieferte: "`Sehr geehrte Damen und Herren,///!123 
                    ..--\_\_??==))88\&/\& "'} \\
\hline
\textbf{Ergebnis} & \textnormal{Testlauf Nr.2 gab den erwarteten E-Mail Text zurück, somit erfolgreich.} \\
\hline
 \end{longtable} 
 


\subsection{Testfall -- /T200/ Bib\TeX -Import}
\label{t200}
Im Pflichtenheft wurde bestimmt, dass der hier beschriebene Vorgang dem
Benutzer ermöglicht über eine Seite eine \BibTeX -Datei hochzuladen. Diese wird
dann an das System weiter gegeben und durch einen \BibTeX -Parser zerlegt und
in die Datenbank übertragen.

\begin{longtable}{|p{5cm}|p{10cm}|}
  \hline
  \textbf{Testfall -- ID und Bezeichnung} &  T200 -- Bib\TeX -Import \\
  \hline
  \textbf{Zu testende Objekte und Methoden} & 
  \textnormal{
  \begin{itemize}
	\item In Komponente \textit{views} die Funktion
	  \lstinline{import_bibtex()}
	\item In Komponente \textit{Server (App: Documents)} die Funktion
	  \lstinline{Bibtex.do_import()}, \lstinline{insert_doc()} und
	  \lstinline{is_valid()} 
  \end{itemize} }
  \\
  %\hline
  \textbf{Kriterien f\"ur erfolgreiche bzw. fehlgeschlagene Testf\"alle} &
  Alle in der Datei enthaltenen validen Dokumente sind nach Abschluss
  des Testes in der Datenbank vorhanden bzw.\ im Fehlerfall nicht vorhanden.\\
  \hline
  \textbf{Einzelschritte} &  Über das Testing Framework wird zuerst
  getestet, dass bei einer fehlenden Anmeldung ein \textbf{HTTP Status Code 404}
  zurück gegeben wird und ein angemeldeter Benutzer mit entsprechneden Rechten
  bekommt einen \textbf{HTTP Status Code 200} zurück. Danach wird die Datei über
  die Seite hochgeladen und damit getestet ob die weiteren Funktionen durch
  diesen Upload ausgelöst werden, ob diese Erfolgreich laufen und am Ende die
  gewünschten Daten in der Datenbank sind. Der Test kommt vollständig ohne
  Benutzerinteraktion aus, muss aber für den Fall von Dead-Locks vom Benutzer
  überwacht und ggf.\ abgebrochen werden.\\
  \hline
  \textbf{Beobachtungen / Log} & 
  Fehler die im Test passieren sorgen dafür, dass der Test abbricht und die
  entsprechnde Bedingung ausgegeben wird. Bei Fehlern in der Verarbeitung der
  \BibTeX -Datei werden diese in eine entsprechende Fehler-Datei geschrieben.
  Sonst läuft der Test still ab.
  \\
  \hline
  \textbf{Besonderheiten } &  Für diesen Test wird als Eingabe Satz die
  \textit{bib2000.bib} mit entsprechenden Daten verwendet und dafür nur eine
  Grundlage Datenbank mit Benutzern, Rechten und E-Mails aber ohne eingetragene
  Dokumente. Weiter wird dieser Test auf den zwei Datenbanksystemen
  \Gls{glos:mysql} und \Gls{glos:sqlite} gefahren\\
  \hline
\end{longtable}

Im folgenden wird das Testszenario mit der \textit{bib2000.bib}-Datei
beschrieben. Das Szenario ist erfolgreich, wenn auch alle Daten aus der Datei
in die Datenbank übernommen wurden. Einmal wird das Szenario mit einer
\Gls{glos:mysql} und einmal mit einer \Gls{glos:sqlite} durchgeführt.

\begin{longtable}{|p{5cm}|p{10cm}|}
  \hline
  \textbf{Testfall -- ID und Bezeichnung} & T200 -- Bib\TeX -Import \\
  \hline
  \textbf{Testlauf Nr.} & 1 (SQLite) \\
  \hline
  \textbf{Eingaben} &  Die Datei \textit{bib2000.bib} und die Userdaten des
  \textit{admin}-Benutzers mit dem Password \textit{sep2012} im Request für den
  \textit{HTTP Status Code 200}.  Der Test läuft automatisch ab und bedarf keiner
  Interaktion, nur die Error-Datei der \textit{bib2000.bib} muss nach Abschluss
  geprüft werden.\\
  \hline
  \textbf{Soll - Reaktion} & Die Datei \textit{bib2000.bib.err} ist nach
  Abschluss des Importes vorhanden aber leer und das Django Testing Framework
  meldet auf der Konsole „0 Errors“ auch als Folge der leeren Fehlerdatei.
  \\
  \hline
  \textbf{Ist -- Reaktion} & Der Test läuft entsprechend der Soll-Reaktion durch.\\
  \hline
  \textbf{Ergebnis} & Erfolgreich \\
  \hline
\end{longtable}

Im folgenden nun der Import der Datei in eine \Gls{glos:mysql}-Datenbank.
\begin{longtable}{|p{5cm}|p{10cm}|}
  \hline
  \textbf{Testfall -- ID und Bezeichnung} & T200 -- Bib\TeX -Import \\
  \hline
  \textbf{Testlauf Nr.} & 2 (MySQL) \\
  \hline
  \textbf{Eingaben} &  Die Datei \textit{bib2000.bib} und die Userdaten des
  \textit{admin}-Benutzers mit dem Password \textit{sep2012} im Request für den
  \textit{HTTP Status Code 200}.  Der Test läuft automatisch ab und bedarf keiner
  Interaktion, nur die Error-Datei der \textit{bib2000.bib} muss nach Abschluss
  geprüft werden.\\
  \hline
  \textbf{Soll - Reaktion} & Die Datei \textit{bib2000.bib.err} ist nach
  Abschluss des Importes vorhanden aber leer und das Django Testing Framework
  meldet auf der Konsole „0 Errors“ auch als Folge der leeren Fehlerdatei.
  \\
  \hline
  \textbf{Ist -- Reaktion} & Der Test wird von einer Exception abgebrochen, die
  von \Gls{glos:mysql} wegen eines „Truncate“ geworfen wird. Ein Keyword-Feld ist
  größer als die definierte Tabellenspalte.\\
  \hline
  \textbf{Ergebnis} & Nicht Erfolgreich: Die „Truncate“-Exception wird durch
  einen Fehler in der {\sffamily import\_bibtex()} ausgelöst, die den
  Keyword-Eintrag nicht korrekt ausplitten kann. Entsprechend muss dieses
  Splitten der Keywords überarbeitet werden.\\ \hline
\end{longtable}

\subsection{T212 - Erweiterte Suche}

Dieser Testfall testet die Erweiterte Suche mit verschiedenen mögliche Eingaben.

\begin{longtable}{|p{5cm}|p{10cm}|}
\hline
\textbf{Testfall -- ID und Bezeichnung} &  \textit{/T212 Erweiterte Suche} \\
\hline
\textbf{Zu testende Objekte und Methoden} & \textnormal{ 
\begin{itemize}
\item In Komponente \emph{Template} die Datei \lstinline{search_pro.html}
\item In Komponente \emph{Views} die Datei \lstinline{search_pro}
\end{itemize}
}\\
\hline
\textbf{Kriterien f\"ur erfolgreiche bzw. fehlgeschlagene Testf\"alle} &
\textit{Ein Test ist erfolgreich, wenn als Ausgabe die in der Datenbank
vorhanden auf die Suchanfrage passenden Dokumente ausgegeben werden.} \\
\hline
\textbf{Einzelschritte} &  \textit{Zuerst muss per Browser die Webseite  
der erweiterten Suche aufgerufen werden. Danach werden je nach Testlauf die 
Felder der erweiterten Suche ausgefüllt und die Suche per Klick auf den 
entsprechenden Button gestartet. Dies wird mit allen im Testplan angegebenen
Browsern durchgeführt. Der Prozess sollte in jedem Fall terminieren und entweder
eine Fehlermeldung oder eine Erfolgsmeldung mitteilen. Sollte sich wieder
erwarten der Prozess nicht terminieren ist der Django Testserver zu beenden.
Aufräumschritte sind in keinem Fall nötig.} \\
\hline
\textbf{Beobachtungen / Log} &  \textit{Zur Beobachtung werden die Ausgabe des
Browsers und des Testservers auf der Konsole benutzt.}\\
\hline
 \end{longtable}

%Die folgenden Tabellen beschreiben, wie der Testfall ausgef\"uhrt wurde und
%welches Ergebnis er geliefert hat. Da es bei Korrektur von Softwarefehlern oder
%anderen Gegebenheiten notwendig ist, einen Test mehrfach durchzuf\"uhren
%(Testl\"aufe), ist jede Testdurchf\"uhrung zu dokumentieren. Daher ist diese
%Tabelle f\"ur \textbf{jeden Testlauf }  zu erstellen und \textbf{ fortlaufend zu
%nummerieren}. \\

Dieser Testlauf testet den einfachsten Fall eines einzelnen Wortes als
Suchanfrage.

\begin{longtable}{|p{5cm}|p{10cm}|}
\hline
\textbf{Testfall -- ID und Bezeichnung} & \textit{/T212/ Erweiterte
Suche} \\
\hline
\textbf{Testlauf Nr.} & \textit{1} \\
\hline
\textbf{Eingaben} & \textit{Es wird der String \lstinline{Analysis} in das
das Eingabefeld des Titels eingetragen.} \\
\hline
\textbf{Soll - Reaktion} & \textit{Es werden alle Bücher mit
\lstinline{Analysis} im Titel angzeigt.
} \\
\hline
\textbf{Ist -- Reaktion} & \textit{Entspricht der Soll-Reaktion} \\
\hline
\textbf{Ergebnis} & \textit{Der Test ist erfolgreich verlaufen} \\
\hline
\textbf{Unvorhergesehene Ereignisse w\"ahrend des Test-laufs } &
\textit{Opera zeigt die Buttons nicht mit transparentem Hintergrund an. Dies
behindert allerdings nicht die Funktionalität.} \\
\hline
\end{longtable}

Dieser Testlauf testen den Fall, dass ein Suchwort eingegeben, welches aus
mehreren Wortfragmenten besteht, welche in nicht richtiger Reihenfolge
eingegeben werden.

\begin{longtable}{|p{5cm}|p{10cm}|}
\hline
\textbf{Testfall -- ID und Bezeichnung} & \textit{/T212/ Erweiterte
Suche} \\
\hline
\textbf{Testlauf Nr.} & \textit{2} \\
\hline
\textbf{Eingaben} & \textit{Es wird der String \lstinline{Ana Func} in das
das Eingabefeld des Titels eingetragen.} \\
\hline
\textbf{Soll - Reaktion} & \textit{Es werden alle Bücher mit
\lstinline{Functional Analysis} im Titel angzeigt.
} \\
\hline
\textbf{Ist -- Reaktion} & \textit{Entspricht der Soll-Reaktion} \\
\hline
\textbf{Ergebnis} & \textit{Der Test ist erfolgreich verlaufen} \\
\hline
\textbf{Unvorhergesehene Ereignisse w\"ahrend des Test-laufs } &
\textit{Opera zeigt die Buttons nicht mit transparentem Hintergrund an. Dies
behindert allerdings nicht die Funktionalität.} \\
\hline
\end{longtable}

Dieser Testlauf testet den Fall das viele Eingaben gemacht werden, sodass
nurnoch ein einzelnes Dokumet als Ergebnis ausgegeben wird.

\begin{longtable}{|p{5cm}|p{10cm}|}
\hline
\textbf{Testfall -- ID und Bezeichnung} & \textit{/T212/ Erweiterte
Suche} \\
\hline
\textbf{Testlauf Nr.} & \textit{3} \\
\hline
\textbf{Eingaben} & \textit{Es wird 
der String \lstinline{Analysis} in das das Eingabefeld des Titels eingetragen,
der String \lstinline{Conway} in das das Eingabefeld des Authors eingetragen,
der String \lstinline{1995} in das das Eingabefeld des Erscheinungsjahrs
eingetragen,
der String \lstinline{Springer} in das das Eingabefeld des Herausgebers
eingetragen,
der String \lstinline{K056031} in das das Eingabefeld der Bibliotheksnummer eingetragen
.} \\
\hline
\textbf{Soll - Reaktion} & \textit{Es wird die Detailansicht des Buches \emph{A
Course of Functional Analysis} angezeigt.} \\
\hline
\textbf{Ist -- Reaktion} & \textit{Entspricht der Soll-Reaktion} \\
\hline
\textbf{Ergebnis} & \textit{Der Test ist erfolgreich verlaufen} \\
\hline
\textbf{Unvorhergesehene Ereignisse w\"ahrend des Test-laufs } &
\textit{Opera zeigt die Buttons nicht mit transparentem Hintergrund an. Dies
behindert allerdings nicht die Funktionalität.} \\
\hline
\end{longtable}

\subsection{Testfall -- /T223/ Ausleihe zurückgeben}

In der folgenden Tabelle wird der Test der Ausleihe zurückgeben beschrieben.
\begin{longtable}{|p{5cm}|p{10cm}|}
\hline
\textbf{Testfall -- ID und Bezeichnung} &  \textnormal{/T223/ Ausleihe zurückgeben} \\
\hline
\textbf{Zu testende Objekte und Funktionen} &  
\textnormal{\begin{itemize}
    \item die Webseite \uline{doc\_detail.html},
    \item in Komponente \textit{Models} die Funktion \lstinline{document.unlend()}, 
    \item in Komponente \textit{Models} die Funktion \lstinline{document.set_status()},
    \item in Komponente \textit{Views} die Funktion \lstinline{doc_detail()}
\end{itemize}}
\\
\hline
\textbf{Kriterien f\"ur erfolgreiche bzw. fehlgeschlagene Testf\"alle} &
\textnormal{Zum Einen muss der letzte Eintrag in der Tabelle \glqq doc\_status\grqq\ 
        von diesem Dokument aktualisiert werden.
        Außerdem sollte ein neuer Eintrag in dieser Tabelle existieren.
        Auch sollte sich die Webseite aktualisieren.}  
\\
\hline
\textbf{Einzelschritte} &  
\textnormal{Zuerst sollte gewährleistet sein, dass man angemeldet ist. Danach muss 
        man die \uline{doc\_detail.html} von einem Dokument öffnen, welches man 
        ausgeliehen hat. Schlussendlich wird nur noch auf den Button 
        \uline{Rückgabe} geklickt.} 
\\
\hline
\textbf{Beobachtungen / Log} &  
\textnormal{Alle Fehler beziehungsweise Logs werden über die Konsole des Servers 
        ausgegeben. }
\\
\hline
\textbf{Besonderheiten } &  
\textnormal{Falls der vorherige Eintrag in \glqq doc\_status \grqq dem neuen bis auf
        den Timestamp gleicht, muss natürlich kein neuer Eintrag eingefügt
        werden. Die Funktionen sollten also komplett ignoriert werden.} 
\\
\hline
 \end{longtable}

Die folgende Tabelle zeigt den Testfall, wenn man ein ausgeliehenes Buch 
zurückgeben möchte.
\begin{longtable}{|p{5cm}|p{10cm}|}
\hline
\textbf{Testfall -- ID und Bezeichnung} & \textnormal{/T223/ Ausleihe zurückgeben} \\
\hline
\textbf{Testlauf Nr.} & \textnormal{1} \\
\hline
\textbf{Eingaben} & 
\textnormal{Erforderliche Eingaben für das Zurückgeben eines Dokumentes ist der
        Datensatz eines Buches, welches an den eingeloggten User ausgeliehen 
        ist, und ein Klick auf den Button \uline{Rückgabe}.}
\\
\hline
\textbf{Soll - Reaktion} & 
\textnormal{Der \glq return\_lend \grq -Wert des letzten Eintrages in \glqq 
        doc\_status \grqq zu diesem Dokument wird auf TRUE gesetzt, ein neuer 
        Eintrag mit dem neuen Status sollte in selbiger Tabelle erstellt werden 
        und der Button \uline{Ausleihen} müsste anstelle von \uline{Rückgabe} 
        und \uline{Übertragen} auf der Webseite dargestellt sein.} 
\\
\hline
\textbf{Ist -- Reaktion} & 
\textnormal{Der Test gibt die gewünschten Soll-Reaktionen zurück.} 
\\
\hline
\textbf{Ergebnis} & 
\textnormal{Der Test ist erfolgreich abgelaufen.} \\
\hline
 \end{longtable}
 
Die folgende Tabelle zeigt den Testfall, wenn man ein Buch zurückgeben möchte,
obwohl man dieses bereits hat (z.B. durch Seitenaktualisierung). 
\begin{longtable}{|p{5cm}|p{10cm}|}
\hline
\textbf{Testfall -- ID und Bezeichnung} & \textnormal{/T223/ Ausleihe zurückgeben} \\
\hline
\textbf{Testlauf Nr.} & \textnormal{1} \\
\hline
\textbf{Eingaben} & 
\textnormal{Erforderliche Eingaben für das Zurückgeben eines Dokumentes ist der
        Datensatz eines Buches, welches an den eingeloggten User ausgeliehen 
        ist, und ein Klick auf den Button \uline{Rückgabe}.}
\\
\hline
\textbf{Soll - Reaktion} & 
\textnormal{Die Seite wird genauso zurückgegeben, wie sie bereits ist. Es wird der
        letzte \glqq doc\_status \grqq -Eintrag nicht verändert und auch kein 
        neuer hinzugefügt.} 
\\
\hline
\textbf{Ist -- Reaktion} & 
\textnormal{Der Test gibt die gewünschten Soll-Reaktionen zurück.} 
\\
\hline
\textbf{Ergebnis} & 
\textnormal{Der Test ist erfolgreich abgelaufen.} \\
\hline
 \end{longtable}


\subsection{T228 Derzeitiger Leihender}

Dieser Testfall testet die Profilseite eines User der nicht der angemeldete
ist. Auf diese Weise werden die Informationen eines derzeitig Leihenden
dargestellt.

\begin{longtable}{|p{5cm}|p{10cm}|}
\hline
\textbf{Testfall -- ID und Bezeichnung} &  \textit{T228 Derzeitiger Leihender} \\
\hline
\textbf{Zu testende Objekte und Methoden} &  \textnormal{
\begin{itemize}
\item In Komponente \emph{Views} die Datei
  \lstinline{stranger_profile.html}
\item In Komponente \emph{Templates} die Funktion
  \lstinline{profile}
\end{itemize}
}
\\
\hline
\textbf{Kriterien f\"ur erfolgreiche bzw. fehlgeschlagene Testf\"alle} &
\textit{Der Test ist erfolgreich, falls das Profil richtig angezeigt wird.} \\
\hline
\textbf{Einzelschritte} &  \textit{Zuerst logt man sich mit einem User
ein und ruft dann die Seite eines anderen Users auf. Dies wird mit
jedem im Testplan aufgeführten Browser durchgeführt.
} \\
\hline
\textbf{Beobachtungen / Log} &  \textit{Beobachtungen werden über das Terminal
und den Browser vorgenommen.} \\
\hline

\end{longtable}

Der User \emph{admin} für diesen Testlauf hat alle vorhandenen Rechte. 

\begin{longtable}{|p{5cm}|p{10cm}|}
\hline
\textbf{Testfall -- ID und Bezeichnung} & \textit{/T228/ Derzeitiger
Leihender} \\
\hline
\textbf{Testlauf Nr.} & \textit{1} \\
\hline
\textbf{Eingaben} & \textit{Es wird der User \emph{admin} güt den Login verwendet} \\
\hline
\textbf{Soll - Reaktion} & \textit{Das komplette Profil mit Informationen über
Gruppenzugehörigkeit des Profilinhabers werden dargestellt.
} \\
\hline
\textbf{Ist -- Reaktion} & \textit{Entspricht der Soll-Reaktion} \\
\hline
\textbf{Ergebnis} & \textit{Der Test war erfolgreich} \\
\hline
 \end{longtable}

Der User \emph{foo} für diesen Testlauf hat die Rechte eines normalen Anwenders
des Programmes. 

\begin{longtable}{|p{5cm}|p{10cm}|}
\hline
\textbf{Testfall -- ID und Bezeichnung} & \textit{/T228/ Derzeitiger
Leihender} \\
\hline
\textbf{Testlauf Nr.} & \textit{2} \\
\hline
\textbf{Eingaben} & \textit{Es wird der User \emph{foo} für den Login verwendet} \\
\hline
\textbf{Soll - Reaktion} & \textit{Das komplette Profil ohne Informationen über
Gruppenzugehörigkeit des Profilinhabers werden dargestellt.
} \\
\hline
\textbf{Ist -- Reaktion} & \textit{Entspricht der Soll-Reaktion} \\
\hline
\textbf{Ergebnis} & \textit{Der Test war erfolgreich} \\
\hline
 \end{longtable}

Der User \emph{gast} für diesen Testlauf modelliert einen unregistrierten
Benutzer und deshalb keinerlei Rechte. 

\begin{longtable}{|p{5cm}|p{10cm}|}
\hline
\textbf{Testfall -- ID und Bezeichnung} & \textit{/T228/ Derzeitiger
Leihender} \\
\hline
\textbf{Testlauf Nr.} & \textit{3} \\
\hline
\textbf{Eingaben} & \textit{Es wird der User \emph{gast} für den Login verwendet} \\
\hline
\textbf{Soll - Reaktion} & \textit{Der Zugriff auf die Seite wird verweigert
} \\
\hline
\textbf{Ist -- Reaktion} & \textit{Entspricht der Soll-Reaktion} \\
\hline
\textbf{Ergebnis} & \textit{Der Test war erfolgreich} \\
\hline
 \end{longtable}

\subsection{Testfall -- /T231/ Universitätsbibliotheks - Export}
In der folgenden Tabelle wird der Test des Exports in das
Universitätsbibliotheks-Format ADT beschrieben.
\begin{longtable}{|p{5cm}|p{10cm}|}
\hline
\textbf{Testfall -- ID und Bezeichnung} &  \textit{/T102/ Universitätsbibliotheks - Export} \\
\hline
\textbf{Zu testende Objekte und Funktionen} & 
\textit{
\begin{itemize}
  \item In Komponente \emph{Server (App: Documents)} die Funktion
	\lstinline{extras_allegro.export_allegro()}
\end{itemize} } \\
\hline
\textbf{Kriterien f\"ur erfolgreiche bzw. fehlgeschlagene Testf\"alle} &
\textit{Der Test ist erfolgreich, wenn eine Datei im ADT-Format ausgegeben
wird, ausser es existieren keine Dokumente in der Datenbank, die exportiert
werden müssten. Weiterhin müssen alle Einträge in der Datei auf Korrektheit
überprüft werden. Vorbedingung ist, dass der eingeloggte Benutzer über
ausreichend Rechte verfügt, um den Export durchzuführen. Falls zu exportierende
Dokumente in der Datenbank vorhanden sind, aber keine neue .ADT-Datei angelegt
wurde, oder der eingeloggte Benutzer verfügt nicht über ausreichend Rechte, so
ist der Test fehlgeschlagen. } \\
\hline
\textbf{Einzelschritte} & 
\textit{Zuerst wird die Webseite in einem Browser aufgerufen. Falls der
eingeloggte Benutzer nicht ausreichend Rechte besitzt, so wird der Export
abgebrochen. Durch den Klick auf \uline{Allegro-Export starten} wird die Datei angelegt und
zum Download angeboten. Es wird eine Meldung ausgegeben, falls es aktuell keine
Dokumente zu exportieren gibt. Anschließend wird auf Serverseite die Datei auf
Korrektheit überprüft. } \\
\hline
\textbf{Beobachtungen / Log} &  \textit{Alle Fehler beziehungsweise Logs werden
über die Konsole des Servers ausgegeben. } \\
\hline

 \end{longtable}

Im Folgenden werden die beiden möglichen erfolgreichen Testläufe für den
Universitätsbibliotheks - Export beschrieben. Angenommen wird dabei, dass
Django, der Server, sowie alle benötigten Komponenten korrekt konfiguriert
sind, da somit Fehler durch die Umgebung ausgeschlossen sind.

\begin{longtable}{|p{5cm}|p{10cm}|}
\hline
\textbf{Testfall -- ID und Bezeichnung} &
\textit{/T231/ Universitätsbibliotheks - Export} \\
\hline 
\textbf{Testlauf Nr.} & \textit{1} \\
\hline
\textbf{Eingaben} & \textit{Erforderliche Eingaben für den Export sind Einträge in der
Datenbank, ein eingeloggter Benutzer und ein Klick auf \uline{Allegro-Export
starten}. Weitere Eingaben sind nicht notwendig. } \\
\hline
\textbf{Soll - Reaktion} & \textit{Die neu erstellte .ADT-Datei wird erstellt
und über den Browser zum Download angeboten, oder es wird die Meldung gegeben,
dass keine Dokumente exportiert werden müssen. } \\
\hline
\textbf{Ist -- Reaktion} & \textit{Die .ADT-Datei wird erstellt und zum
Download angeboten. } \\
\hline
\textbf{Ergebnis} & \textit{Der Test ist erfolgreich abgelaufen. } \\
\hline
\end{longtable}

\begin{longtable}{|p{5cm}|p{10cm}|}
\hline
\textbf{Testfall -- ID und Bezeichnung} &
\textit{/T231/ Universitätsbibliotheks - Export} \\
\hline 
\textbf{Testlauf Nr.} & \textit{2} \\
\hline
\textbf{Eingaben} & \textit{Erforderliche Eingaben für den Export sind Einträge in der
Datenbank, ein eingeloggter Benutzer und ein Klick auf \uline{Allegro-Export
starten}. Weitere Eingaben sind nicht notwendig. } \\
\hline
\textbf{Soll - Reaktion} & \textit{Die neu erstellte .ADT-Datei wird erstellt
und über den Browser zum Download angeboten, oder es wird die Meldung gegeben,
dass keine Dokumente exportiert werden müssen. } \\
\hline
\textbf{Ist -- Reaktion} & \textit{Es wird die Meldung gegeben, dass keine
Dokumente exportiert werden müssen. } \\
\hline
\textbf{Ergebnis} & \textit{Der Test ist erfolgreich abgelaufen. } \\
\hline
\end{longtable}



