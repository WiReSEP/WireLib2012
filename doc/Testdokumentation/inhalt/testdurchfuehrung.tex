% Kapitel 3 mit den entsprechenden Unterkapiteln
% Die Unterkapitel k\"onnen auch in separaten Dateien stehen,
% die dann mit dem \include-Befehl eingebunden werden.
%----------------------------------------------------------------------------

\chapter{Testdurchf\"uhrung}

%In diesem Abschnitt werden die einzelnen Testf\"alle beschrieben und deren
%Durchf\"uhrungen (=Testl\"aufe) protokolliert.\\
%Ein Testfall ist eine Kombination von Eingabedaten, Bedingungen und erwarteten
%Ausgaben, die einen bestimmten Zweck erf\"ullen. Man pr\"uft z.B., ob Vorgaben in
%einem Spezifikationsdokument eingehalten werden oder ob der Programmablauf
%tats\"achlich dem erwarteten Pfad entspricht.\\
%Dieses Kapitel enth\"alt drei Arten von Tabellen:\\
%1.  Die \"Ubersichtstabelle zeigt an, welche Testf\"alle es gibt und welcher
%Testfall welche Objekte, Methoden oder Anforderungen testet. So hat man den
%\"Uberblick, Verfolgbarkeit zwischen der Testdokumentation und anderen
%Dokumenten, und man kann sehen, ob die Testf\"alle vollst\"andig sind.\\
%2.  Der Testfall beschreibt jeden einzelnen Testfall im Detail.\\
%3.  Der Testlauf beschreibt eine Durchf\"uhrung eines Testfalls. Derselbe
%Testfall kann mit verschiedenen Eingabedaten oder auch mit verschiedenen
%Softwareversionen mehrmals durchgef\"uhrt werden.\\

\section{\"Ubersichtstabelle}
  %In der folgenden Tabelle sind entweder f\"ur alle Testf\"alle die zu testende
  %Komponente oder die zu testende Funktion (oder beides) anzugeben. Die
  %Bezeichnungen der Komponenten m\"ussen konsistent sein mit denen in Fein- und
  %Grobkonzept, um die Verfolgbarkeit zum Konzept sicherzustellen. Die IDs und
  %Bezeichnungen der Funktionen m\"ussen denen im Pflichtenheft entsprechen, um
  %die Verfolgbarkeit zu den Anforderungen sicherzustellen. \\
\begin{tabular}{|c|c|c|}
\hline
\textbf{Testfall ID und Bezeichnung} &  \textbf {Zu testende Komponente} &
\textbf {Zu testende Funktion}\\
\hline
/T212/ Erweiterte Suche &  Views und Template  & /F212/ Erweiterte Suche \\
\hline
/T228/ Derzeitiger Leihender &  Views und Template  &
/F228/ Derzeitiger Leihender\\
\hline
&&
\end{tabular}

\subsection{Testfall -- /T103/ Mailtexte ändern}
%Jeder Testfall erh\"alt eine eindeutige Identifikation mit Kurzbezeichnung.\\
%Beispiel: /T100/ Lager anlegen\\
%Die folgende Tabelle beschreibt den Testfall. \\
In der folgenden Tabelle wird der Test zur Funktion E-Mail Änderung protokolliert.
\begin{longtable}{|p{5cm}|p{10cm}|}
\hline
\textbf{Testfall -- ID und Bezeichnung} &  \textnormal{/T103/ Mailtexte ändern} \\
\hline
\textbf{Zu testende Objekte und Methoden} &  \textnormal{
\begin{itemize}
\item In Komponente \textit{Admin} der Bereich E-Mail 
\end{itemize} }
\\
\hline
\textbf{Kriterien f\"ur erfolgreiche bzw. fehlgeschlagene Testf\"alle} &
\textnormal{Erfolgreich: E-Mail Inhalt steht, wie gewünscht, in der Datenbank ;
        Fehlgeschlagen: E-Mail Inhalt steht nicht, oder nicht wie gewünscht in der DB   } \\
\hline
\textbf{Einzelschritte} &  \textnormal{Zunächst wird über das Django-Testing-Framework getestet,
ob der Benutzer angemeldet ist (angemeldet wirft einen Http-Status-Code 200 zurück, sonst 
einen Http-Status-Code 400). Falls dieser angemeldete Benutzer genügend Rechte besitzt,
wiederum zu testen über eine spezielle Django-Testing-Authentification Methode, wird der interne View
für das E-Mail ändern angesteuert (Http-Code=200 erfolgreich, Http-Code=404 gescheitert.).
Falls dies alles erfolgreich sein sollte, kann man nun E-Mail Texte verändern.
} \\
\hline
\textbf{Beobachtungen / Log} &  \textnormal{Ergebnisse werden haptsächlich über 
die Kommandozeile per \textbf{stdout}(gibt Text der E-Mail aus) und \textbf{Django-Testing-
Framework} kontrolliert  } \\
\hline


 \end{longtable}

%Die folgenden Tabellen beschreiben, wie der Testfall ausgef\"uhrt wurde und
%welches Ergebnis er geliefert hat. Da es bei Korrektur von Softwarefehlern oder
%anderen Gegebenheiten notwendig ist, einen Test mehrfach durchzuf\"uhren
%(Testl\"aufe), ist jede Testdurchf\"uhrung zu dokumentieren. Daher ist diese
%Tabelle f\"ur \textbf{jeden Testlauf }  zu erstellen und \textbf{ fortlaufend zu
%nummerieren}. \\

Folgend werden die benötigten Testdurchläufe näher beschrieben
\begin{longtable}{|p{5cm}|p{10cm}|}
\hline
\textbf{Testfall -- ID und Bezeichnung} & \textnormal{/T103/ Mailtexte ändern} \\
\hline
\textbf{Testlauf Nr.} & \textnormal{1} \\
\hline
\textbf{Eingaben} & \textnormal{Ein eingeloggter Nutzer mit entsprechenden Rechten, E-Mail 
Einträge in der Datenbank, Klick auf E-Mail ändern in dem Admin Backend } \\
\hline
\textbf{Soll - Reaktion} & \textnormal{Ausgabe sollte eingegebenen E-Mail
Text enthalten: "`Sehr geehrte Damen und Herren, "').
} \\
\hline
\textbf{Ist -- Reaktion} & \textnormal{stdout lieferte: "`Sehr geehrte Damen und Herren, "'} \\
\hline
\textbf{Ergebnis} & \textnormal{Testlauf Nr.1 gab den erwarteten E-Mail Text zurück, somit erfolgreich.} \\
\hline
 \end{longtable}
 
Dieser Testdurchlauf testet, ob das Text Feld der E-Mail mögliche Sonderzeichen beachtet.
\begin{longtable}{|p{5cm}|p{10cm}|}
\hline
\textbf{Testfall -- ID und Bezeichnung} & \textnormal{/T103/ Mailtexte ändern} \\
\hline
\textbf{Testlauf Nr.} & \textnormal{2} \\
\hline
\textbf{Eingaben} & \textnormal{Ein eingeloggter Nutzer mit entsprechenden Rechten, E-Mail 
Einträge in der Datenbank, Klick auf E-Mail ändern in dem Admin Backend } \\
\hline
\textbf{Soll - Reaktion} & \textnormal{Ausgabe sollte eingegebenen E-Mail
Text enthalten: "`Sehr geehrte Damen und Herren,///!123 
                    ..--\_\_??==))88\&/\& "').
} \\
\hline
\textbf{Ist -- Reaktion} & \textnormal{stdout lieferte: "`Sehr geehrte Damen und Herren,///!123 
                    ..--\_\_??==))88\&/\& "'} \\
\hline
\textbf{Ergebnis} & \textnormal{Testlauf Nr.2 gab den erwarteten E-Mail Text zurück, somit erfolgreich.} \\
\hline
 \end{longtable} 
 


\subsection{Testfall -- /T200/ Bib\TeX -Import}
\label{t200}
Im Pflichtenheft wurde bestimmt, dass der hier beschriebene Vorgang dem
Benutzer ermöglicht über eine Seite eine \BibTeX -Datei hochzuladen. Diese wird
dann an das System weiter gegeben und durch einen \BibTeX -Parser zerlegt und
in die Datenbank übertragen.

\begin{longtable}{|p{5cm}|p{10cm}|}
  \hline
  \textbf{Testfall -- ID und Bezeichnung} &  T200 -- Bib\TeX -Import \\
  \hline
  \textbf{Zu testende Objekte und Methoden} & 
  \textnormal{
  \begin{itemize}
	\item In Komponente \textit{views} die Funktion
	  \lstinline{import_bibtex()}
	\item In Komponente \textit{Server (App: Documents)} die Funktion
	  \lstinline{Bibtex.do_import()}, \lstinline{insert_doc()} und
	  \lstinline{is_valid()} 
  \end{itemize} }
  \\
  %\hline
  \textbf{Kriterien f\"ur erfolgreiche bzw. fehlgeschlagene Testf\"alle} &
  Alle in der Datei enthaltenen validen Dokumente sind nach Abschluss
  des Testes in der Datenbank vorhanden bzw.\ im Fehlerfall nicht vorhanden.\\
  \hline
  \textbf{Einzelschritte} &  Über das Testing Framework wird zuerst
  getestet, dass bei einer fehlenden Anmeldung ein \textbf{HTTP Status Code 404}
  zurück gegeben wird und ein angemeldeter Benutzer mit entsprechneden Rechten
  bekommt einen \textbf{HTTP Status Code 200} zurück. Danach wird die Datei über
  die Seite hochgeladen und damit getestet ob die weiteren Funktionen durch
  diesen Upload ausgelöst werden, ob diese Erfolgreich laufen und am Ende die
  gewünschten Daten in der Datenbank sind. Der Test kommt vollständig ohne
  Benutzerinteraktion aus, muss aber für den Fall von Dead-Locks vom Benutzer
  überwacht und ggf.\ abgebrochen werden.\\
  \hline
  \textbf{Beobachtungen / Log} & 
  Fehler die im Test passieren sorgen dafür, dass der Test abbricht und die
  entsprechnde Bedingung ausgegeben wird. Bei Fehlern in der Verarbeitung der
  \BibTeX -Datei werden diese in eine entsprechende Fehler-Datei geschrieben.
  Sonst läuft der Test still ab.
  \\
  \hline
  \textbf{Besonderheiten } &  Für diesen Test wird als Eingabe Satz die
  \textit{bib2000.bib} mit entsprechenden Daten verwendet und dafür nur eine
  Grundlage Datenbank mit Benutzern, Rechten und E-Mails aber ohne eingetragene
  Dokumente. Weiter wird dieser Test auf den zwei Datenbanksystemen
  \Gls{glos:mysql} und \Gls{glos:sqlite} gefahren\\
  \hline
\end{longtable}

Im folgenden wird das Testszenario mit der \textit{bib2000.bib}-Datei
beschrieben. Das Szenario ist erfolgreich, wenn auch alle Daten aus der Datei
in die Datenbank übernommen wurden. Einmal wird das Szenario mit einer
\Gls{glos:mysql} und einmal mit einer \Gls{glos:sqlite} durchgeführt.

\begin{longtable}{|p{5cm}|p{10cm}|}
  \hline
  \textbf{Testfall -- ID und Bezeichnung} & T200 -- Bib\TeX -Import \\
  \hline
  \textbf{Testlauf Nr.} & 1 (SQLite) \\
  \hline
  \textbf{Eingaben} &  Die Datei \textit{bib2000.bib} und die Userdaten des
  \textit{admin}-Benutzers mit dem Password \textit{sep2012} im Request für den
  \textit{HTTP Status Code 200}.  Der Test läuft automatisch ab und bedarf keiner
  Interaktion, nur die Error-Datei der \textit{bib2000.bib} muss nach Abschluss
  geprüft werden.\\
  \hline
  \textbf{Soll - Reaktion} & Die Datei \textit{bib2000.bib.err} ist nach
  Abschluss des Importes vorhanden aber leer und das Django Testing Framework
  meldet auf der Konsole „0 Errors“ auch als Folge der leeren Fehlerdatei.
  \\
  \hline
  \textbf{Ist -- Reaktion} & Der Test läuft entsprechend der Soll-Reaktion durch.\\
  \hline
  \textbf{Ergebnis} & Erfolgreich \\
  \hline
\end{longtable}

Im folgenden nun der Import der Datei in eine \Gls{glos:mysql}-Datenbank.
\begin{longtable}{|p{5cm}|p{10cm}|}
  \hline
  \textbf{Testfall -- ID und Bezeichnung} & T200 -- Bib\TeX -Import \\
  \hline
  \textbf{Testlauf Nr.} & 2 (MySQL) \\
  \hline
  \textbf{Eingaben} &  Die Datei \textit{bib2000.bib} und die Userdaten des
  \textit{admin}-Benutzers mit dem Password \textit{sep2012} im Request für den
  \textit{HTTP Status Code 200}.  Der Test läuft automatisch ab und bedarf keiner
  Interaktion, nur die Error-Datei der \textit{bib2000.bib} muss nach Abschluss
  geprüft werden.\\
  \hline
  \textbf{Soll - Reaktion} & Die Datei \textit{bib2000.bib.err} ist nach
  Abschluss des Importes vorhanden aber leer und das Django Testing Framework
  meldet auf der Konsole „0 Errors“ auch als Folge der leeren Fehlerdatei.
  \\
  \hline
  \textbf{Ist -- Reaktion} & Der Test wird von einer Exception abgebrochen, die
  von \Gls{glos:mysql} wegen eines „Truncate“ geworfen wird. Ein Keyword-Feld ist
  größer als die definierte Tabellenspalte.\\
  \hline
  \textbf{Ergebnis} & Nicht Erfolgreich: Die „Truncate“-Exception wird durch
  einen Fehler in der {\sffamily import\_bibtex()} ausgelöst, die den
  Keyword-Eintrag nicht korrekt ausplitten kann. Entsprechend muss dieses
  Splitten der Keywords überarbeitet werden.\\ \hline
\end{longtable}

\subsection{Testfall -- /T201/ Webinterface - Import}
In der folgenden Tabelle wird der Test des Imports über das Webinterface
beschrieben.
\begin{longtable}{|p{5cm}|p{10cm}|}
\hline
\textbf{Testfall -- ID und Bezeichnung} &  \textnormal{/T201/ Webinterface - Import} \\
\hline
\textbf{Zu testende Objekte und Methoden} &  
\textnormal{
\begin{itemize}
  \item In Komponente \textit{views} die Funktion \lstinline{doc_add}
  \item In Komponente \textit{Template} die Datei \textit{doc\_add.html}
\end{itemize}}
\\
\hline
\textbf{Kriterien f\"ur erfolgreiche bzw. fehlgeschlagene Testf\"alle} &
\textnormal{Wenn die eingegebenen Daten korrekt sind, so wird das Dokument in die
Datenbank übernommen. Bei unvollständigen Daten wird die Webseite mit einer
entprechenden Meldung erneut aufgerufen. } \\
\hline
\textbf{Einzelschritte} &  \textnormal{Zuerst wird die Webseite in einem Browser
aufgerufen. Falls der eingeloggte Benutzer nicht über ausreichend Rechte
verfügt, so wird das Laden der Webseite abgebrochen. Andernfalls wird die
Eingabeform für ein Dokument angezeigt und beim Absenden in die Datenabnk
übernommen. } \\
\hline
\textbf{Beobachtungen / Log} &  \textnormal{Der Server loggt in der Konsole, dass
doc\_add angefordert wurde. Informationen über das Hinzufügen des Dokumentes
werden nicht ausgegeben. } \\
\hline
\textbf{Abh\"angigkeiten} &  \textnormal{Dieser Test ist von dem \BibTeX - Import
abhängig, da es auch anstelle von Texteingabemasken auch ein \BibTeX - Uploader
zur Verfügung steht. } \\
\hline

 \end{longtable}

Die folgende Tabelle zeigt den Test für ein vollständig ausgefüllte Eingabemaske ohne \BibTeX-Upload.
\begin{longtable}{|p{5cm}|p{10cm}|}
\hline
\textbf{Testfall -- ID und Bezeichnung} & \textnormal{/T201/ Webinterface - Import} \\
\hline
\textbf{Testlauf Nr.} & \textnormal{1} \\
\hline
\textbf{Eingaben} & \textnormal{Eingaben sind Einträge in der Datenbank, sowie alle
Eingaben in der Eingabemaske. In der Eingabemaske sind alle benötigten Felder
bedient und der \BibTeX - Upload ist leer. } \\
\hline
\textbf{Soll - Reaktion} & \textnormal{Die eingegebenen Daten werden als Dokument
in die Datenbank übernommen. } \\
\hline
\textbf{Ist -- Reaktion} & \textnormal{Wie Soll - Reaktion.} \\
\hline
\textbf{Ergebnis} & \textnormal{Der Test verlief erfolgreich. } \\
\hline
 \end{longtable}

Diese nächste Tabelle beschreibt den Test, wenn nicht alle Muss-Felder ausgefüllt sind oder bereits in der Datenbank vorhanden sind. Der \BibTeX-Upload ist leer.
\begin{longtable}{|p{5cm}|p{10cm}|}
\hline
\textbf{Testfall -- ID und Bezeichnung} & \textnormal{/T201/ Webinterface - Import} \\
\hline
\textbf{Testlauf Nr.} & \textnormal{2} \\
\hline
\textbf{Eingaben} & \textnormal{Eingaben sind Einträge in der Datenbank, sowie alle
Eingaben in der Eingabemaske. In der Eingabemaske sind nicht alle benötigten Felder
bedient und der \BibTeX - Upload ist leer. } \\
\hline
\textbf{Soll - Reaktion} & \textnormal{Die eingegebenen Daten werden nicht als Dokument
in die Datenbank übernommen und die Webseite wird erneut angezeigt. } \\
\hline
\textbf{Ist -- Reaktion} & \textnormal{Wie Soll - Reaktion.} \\
\hline
\textbf{Ergebnis} & \textnormal{Der Test verlief erfolgreich. } \\
\hline
 \end{longtable}

Die Tabelle stellt den Test für den \BibTeX-Upload dar.
\begin{longtable}{|p{5cm}|p{10cm}|}
\hline
\textbf{Testfall -- ID und Bezeichnung} & \textnormal{/T201/ Webinterface - Import} \\
\hline
\textbf{Testlauf Nr.} & \textnormal{3} \\
\hline
\textbf{Eingaben} & \textnormal{Eingaben sind Einträge in der Datenbank, sowie alle
Eingaben in der Eingabemaske. In der Eingabemaske sind nicht alle benötigten Felder
bedient und der \BibTeX - Upload ist befüllt. } \\
\hline
\textbf{Soll - Reaktion} & \textnormal{Für die hochgeladene Datei wird der \BibTeX
- Importer aufgerufen. Alle Fehler werden dem Nutzer dabei auf der Webseite
angezeigt.  } \\
\hline
\textbf{Ist -- Reaktion} & \textnormal{Wie Soll - Reaktion.} \\
\hline
\textbf{Ergebnis} & \textnormal{Der Test verlief erfolgreich. } \\
\hline
 \end{longtable}

\subsection{Testfall -- /T202/ Editieren von Dokumenten}
Die folgende Tabelle beschreibt den Testfall: /T202/ Editieren von Dokumenten \\
\begin{longtable}{|p{5cm}|p{10cm}|}
\hline
\textbf{Testfall -- ID und Bezeichnung} &  \textnormal{/T202/ Editieren von
Dokumenten} \\
\hline
\textbf{Zu testende Objekte und Methoden} &  \textnormal{
\begin{itemize}
  \item In Komponente \textit{Models} die Datei \lstinline{doc\_add.html}
  \item In Komponente \textit{Models} die Datei \lstinline{forms.py}
\end{itemize}
} \\
\hline
\textbf{Kriterien f\"ur erfolgreiche bzw. fehlgeschlagene Testf\"alle} &
\textnormal{Test erfolgreich: Dokument wurde erstellt und im Table documents
erzeugt, oder eine Meldung wirft das die Importdatei fehlerhaft ist.
Test fehlgeschlagen: Dokument ist nicht im Table documents vorhanden.} \\
\hline
\textbf{Einzelschritte} &  \textnormal{
Über das Userinterface können Bibliothekare und Administratoren Dokumente,
über den Link \uline{Import}, der Datenbank hinzufügen.
Dabei gelangen sie auf eine Form mit der Möglichkeit des Imports einer
\BibTeX Datei oder der direkten Eingabe der Informationen.
} \\
\hline
\textbf{Beobachtungen / Log} &  \textnormal{
Es werden über das Terminal Ladezeiten sowie Schreibzugriffe beobachtet.} \\
\hline

 \end{longtable}

Die folgende Tabelle beschreibt, wie der Testfall /T202/ Editieren von
Dokumenten ausgef\"uhrt wurde und welches Ergebnis er bei einer Bib-Tex Eingabe geliefert hat.


\begin{longtable}{|p{5cm}|p{10cm}|}
\hline
\textbf{Testfall -- ID und Bezeichnung} & \textnormal{/T202/ Editieren von
Dokumenten} \\
\hline
\textbf{Testlauf Nr.} & \textnormal{1} \\
\hline
\textbf{Eingaben} & \textnormal{
Es wird eine \BibTex Datei eingegeben} \\
\hline
\textbf{Soll - Reaktion} & \textnormal{
Datei wird eingelesen und ausgelesen. Die Informationen werden in der Datenbank
gespeichert.
} \\
\hline
\textbf{Ist -- Reaktion} & \textnormal{Datei wurde eingelesen und Informationen
gespeichert.} \\
\hline
\textbf{Ergebnis} & \textnormal{Test war erfolgreich.} \\
\hline
\end{longtable}
Die folgende Tabelle beschreibt, wie der Testfall /T202/ Editieren von
Dokumenten ausgef\"uhrt wurde und welches Ergebnis er bei Eingabe der Informationen über Felder geliefert hat.
\begin{longtable}{|p{5cm}|p{10cm}|}
\hline
\textbf{Testfall -- ID und Bezeichnung} & \textnormal{/T202/ Editieren von
Dokumenten} \\
\hline
\textbf{Testlauf Nr.} & \textnormal{2} \\
\hline
\textbf{Eingaben} & \textnormal{
Eingabe der Informationen über die Felder.} \\
\hline
\textbf{Soll - Reaktion} & \textnormal{
Die Felder werden ausgelesen und die Informationen in der Datenbank gespeichert.
Bei notwendigen Feldern wird eine Meldung angezeigt, diese Felder noch zu
füllen.
} \\
\hline
\textbf{Ist -- Reaktion} & \textnormal{Daten wurden eingelesen und Informationen 
gespeichert.} \\
\hline
\textbf{Ergebnis} & \textnormal{Test war erfolgreich.} \\
\hline
\end{longtable}


\subsection{T212 - Erweiterte Suche}

Dieser Testfall testet die Erweiterte Suche mit verschiedenen mögliche Eingaben.

\begin{longtable}{|p{5cm}|p{10cm}|}
\hline
\textbf{Testfall -- ID und Bezeichnung} &  \textit{/T212 Erweiterte Suche} \\
\hline
\textbf{Zu testende Objekte und Methoden} & \textnormal{ 
\begin{itemize}
\item In Komponente \emph{Template} die Datei \lstinline{search_pro.html}
\item In Komponente \emph{Views} die Datei \lstinline{search_pro}
\end{itemize}
}\\
\hline
\textbf{Kriterien f\"ur erfolgreiche bzw. fehlgeschlagene Testf\"alle} &
\textit{Ein Test ist erfolgreich, wenn als Ausgabe die in der Datenbank
vorhanden auf die Suchanfrage passenden Dokumente ausgegeben werden.} \\
\hline
\textbf{Einzelschritte} &  \textit{Zuerst muss per Browser die Webseite  
der erweiterten Suche aufgerufen werden. Danach werden je nach Testlauf die 
Felder der erweiterten Suche ausgefüllt und die Suche per Klick auf den 
entsprechenden Button gestartet. Dies wird mit allen im Testplan angegebenen
Browsern durchgeführt. Der Prozess sollte in jedem Fall terminieren und entweder
eine Fehlermeldung oder eine Erfolgsmeldung mitteilen. Sollte sich wieder
erwarten der Prozess nicht terminieren ist der Django Testserver zu beenden.
Aufräumschritte sind in keinem Fall nötig.} \\
\hline
\textbf{Beobachtungen / Log} &  \textit{Zur Beobachtung werden die Ausgabe des
Browsers und des Testservers auf der Konsole benutzt.}\\
\hline
 \end{longtable}

%Die folgenden Tabellen beschreiben, wie der Testfall ausgef\"uhrt wurde und
%welches Ergebnis er geliefert hat. Da es bei Korrektur von Softwarefehlern oder
%anderen Gegebenheiten notwendig ist, einen Test mehrfach durchzuf\"uhren
%(Testl\"aufe), ist jede Testdurchf\"uhrung zu dokumentieren. Daher ist diese
%Tabelle f\"ur \textbf{jeden Testlauf }  zu erstellen und \textbf{ fortlaufend zu
%nummerieren}. \\

Dieser Testlauf testet den einfachsten Fall eines einzelnen Wortes als
Suchanfrage.

\begin{longtable}{|p{5cm}|p{10cm}|}
\hline
\textbf{Testfall -- ID und Bezeichnung} & \textit{/T212/ Erweiterte
Suche} \\
\hline
\textbf{Testlauf Nr.} & \textit{1} \\
\hline
\textbf{Eingaben} & \textit{Es wird der String \lstinline{Analysis} in das
das Eingabefeld des Titels eingetragen.} \\
\hline
\textbf{Soll - Reaktion} & \textit{Es werden alle Bücher mit
\lstinline{Analysis} im Titel angzeigt.
} \\
\hline
\textbf{Ist -- Reaktion} & \textit{Entspricht der Soll-Reaktion} \\
\hline
\textbf{Ergebnis} & \textit{Der Test ist erfolgreich verlaufen} \\
\hline
\textbf{Unvorhergesehene Ereignisse w\"ahrend des Test-laufs } &
\textit{Opera zeigt die Buttons nicht mit transparentem Hintergrund an. Dies
behindert allerdings nicht die Funktionalität.} \\
\hline
\end{longtable}

Dieser Testlauf testen den Fall, dass ein Suchwort eingegeben, welches aus
mehreren Wortfragmenten besteht, welche in nicht richtiger Reihenfolge
eingegeben werden.

\begin{longtable}{|p{5cm}|p{10cm}|}
\hline
\textbf{Testfall -- ID und Bezeichnung} & \textit{/T212/ Erweiterte
Suche} \\
\hline
\textbf{Testlauf Nr.} & \textit{2} \\
\hline
\textbf{Eingaben} & \textit{Es wird der String \lstinline{Ana Func} in das
das Eingabefeld des Titels eingetragen.} \\
\hline
\textbf{Soll - Reaktion} & \textit{Es werden alle Bücher mit
\lstinline{Functional Analysis} im Titel angzeigt.
} \\
\hline
\textbf{Ist -- Reaktion} & \textit{Entspricht der Soll-Reaktion} \\
\hline
\textbf{Ergebnis} & \textit{Der Test ist erfolgreich verlaufen} \\
\hline
\textbf{Unvorhergesehene Ereignisse w\"ahrend des Test-laufs } &
\textit{Opera zeigt die Buttons nicht mit transparentem Hintergrund an. Dies
behindert allerdings nicht die Funktionalität.} \\
\hline
\end{longtable}

Dieser Testlauf testet den Fall das viele Eingaben gemacht werden, sodass
nurnoch ein einzelnes Dokumet als Ergebnis ausgegeben wird.

\begin{longtable}{|p{5cm}|p{10cm}|}
\hline
\textbf{Testfall -- ID und Bezeichnung} & \textit{/T212/ Erweiterte
Suche} \\
\hline
\textbf{Testlauf Nr.} & \textit{3} \\
\hline
\textbf{Eingaben} & \textit{Es wird 
der String \lstinline{Analysis} in das das Eingabefeld des Titels eingetragen,
der String \lstinline{Conway} in das das Eingabefeld des Authors eingetragen,
der String \lstinline{1995} in das das Eingabefeld des Erscheinungsjahrs
eingetragen,
der String \lstinline{Springer} in das das Eingabefeld des Herausgebers
eingetragen,
der String \lstinline{K056031} in das das Eingabefeld der Bibliotheksnummer eingetragen
.} \\
\hline
\textbf{Soll - Reaktion} & \textit{Es wird die Detailansicht des Buches \emph{A
Course of Functional Analysis} angezeigt.} \\
\hline
\textbf{Ist -- Reaktion} & \textit{Entspricht der Soll-Reaktion} \\
\hline
\textbf{Ergebnis} & \textit{Der Test ist erfolgreich verlaufen} \\
\hline
\textbf{Unvorhergesehene Ereignisse w\"ahrend des Test-laufs } &
\textit{Opera zeigt die Buttons nicht mit transparentem Hintergrund an. Dies
behindert allerdings nicht die Funktionalität.} \\
\hline
\end{longtable}

\subsection{T214 - Sortierung der Suche  }
In der folgenden Tabelle wird der Testverlauf für die "Sortierung der Suche" beschrieben.\\
\begin{longtable}{|p{5cm}|p{10cm}|}
\hline
\textbf{Testfall -- ID und Bezeichnung} &  \textnormal{                                                        /T214/ Sortierung der Suche} \\
\hline
\textbf {Zu testende Objekte und Methoden.}  &  
\textnormal {
\begin{itemize}
    \item In Komponenten \textit{Views}  die Datei \lstinline {search, search_pro,} und \lstinline {doc_list}
    \item In Komponente \textit{Template} die Datei \lstinline {doc_list}
\end{itemize}}
\\
\hline
\textbf{Kriterien f\"ur erfolgreiche bzw. fehlgeschlagene Testf\"alle. } &
\textnormal{Im Fall, der Sortierung nach Titel ist der Vorgang erfolgreich, wenn die gefundenen Dokumente alphabetisch und absteigend aufgelistet werden. Die Sortierung nach Authoren ist dann erfolgreich abgelaufen, wenn anschließend die Dokumente alphabetisch absteigend nach ihren Autoren augelistet werden. Die Sortierung nach Veröffentlichung ist erfolgreich abgelaufen, wenn anschließend die Documente absteigend nach ihrem Veröffentlichungsdatum aufelistet werden. } \\
\hline
\textbf{Einzelschritte} &  \textnormal{Zum Anfang muss man eine Suche, erweiterte Suche oder das Literaturverzeichnis aufrufen. Je nach Auswahl werden alle oder nur ein Teil der Dokumente aufgelistet. Es stehen dann einige Sortier-Links zur Verfügung. \uline{Dokumententitel}, \uline{Veröffentlichung}, \uline{Autoren}. Ein einfacher Klick reicht um die Sortierungsfunktion nach den Kriterien anzuordnen.} \\
\hline
\textbf{Beobachtungen / Log} &  \textnormal{ Die Fehler können über den Browser visuell fest gestellt werden, oder über die Konsole gefunden werden. \ldots} \\
\hline
\textbf{Abh\"angigkeiten} &  \textnormal{Diese Funktion ist teilweise abhängig von der Funktionsfähigkeit der erweiterten Suche. Man kommt auch über das Literaturverzeichnis an sortierbare Documente, aber um durch eine erweiterete Suche an jene zu kommen muss diese funktionieren und und welche liefern. } \\
\hline

 \end{longtable}



Testfall für Sortierung der Suche nach Dokumententitel\\
\begin{longtable}{|p{5cm}|p{10cm}|}
\hline
\textbf{Testfall -- ID und Bezeichnung} & \textnormal{/T214/ Sortierung der Suche} \\
\hline
\textbf{Testlauf Nr.} & \textnormal{1} \\
\hline
\textbf{Eingaben} & \textnormal{In der Navigationsliste links wird wird das Literaturverzeichnis aufgerufen um zu sortierende Dokumente anzuzeigen. Anschließend wird der link \uline{Dokumententitel} einfach angeklickt.} \\
\hline
\textbf{Soll - Reaktion} & \textnormal{Es sollten nun immernoch die gleichen Dokumente angezeigt werden, aber in alphabetischer Reihenfolge der Titel absteigend geordnet aufgelistet sein .} \\
\hline
\textbf{Ist -- Reaktion} & \textnormal{Die vor liegenden Dokumente sind alphabetisch nach ihren Titeln absteigend sortiert.} \\
\hline
\textbf{Ergebnis} & \textnormal{Da die Soll- und Ist- Reaktion übereinstimmen, kann man den Test als erfolgreich betrachten.} \\
\hline
\textbf{Nacharbeiten } & \textnormal{Ist ein Testlauf nicht erfolgreich
durchgef\"uhrt worden, so werden hier die erforderlichen Nacharbeiten aufgef\"uhrt
(z.B. Bugfixes).} \\
\hline
 \end{longtable}
Testfall für Sortierung der Suche nach Veröffentlichung\\ 
\begin{longtable}{|p{5cm}|p{10cm}|}
\hline
\textbf{Testfall -- ID und Bezeichnung} & \textnormal{/T214/ Sortieren der Suche} \\
\hline
\textbf{Testlauf Nr.} & \textnormal{2} \\
\hline
\textbf{Eingaben} & \textnormal{In der Navigationsliste links wird wird das Literaturverzeichnis aufgerufen um zu sortierende Dokumente anzuzeigen. Anschließend wird der link \uline{Veröffentlichung} einfach angeklickt.} \\
\hline
\textbf{Soll - Reaktion} & \textnormal{Es sollten nun immernoch die gleichen Dokumente angezeigt werden, aber in numerischer Reihenfolge des Veröffentlichungsjahr absteigend geordnet aufgelistet sein .} \\
\hline
\textbf{Ist -- Reaktion} & \textnormal{Die vor liegenden Dokumente sind numerisch nach ihrem Veröffentlichungsjahr absteigend sortiert.} \\
\hline
\textbf{Ergebnis} & \textnormal{Da die Soll- und Ist- Reaktion übereinstimmen, kann man den Test als erfolgreich betrachten.} \\
\hline
\textbf{Nacharbeiten } & \textnormal{Ist ein Testlauf nicht erfolgreich
durchgef\"uhrt worden, so werden hier die erforderlichen Nacharbeiten aufgef\"uhrt
(z.B. Bugfixes).} \\
\hline
 \end{longtable}
Testfall für Sortierung der Suche nach Autoren\\ 
\begin{longtable}{|p{5cm}|p{10cm}|}
\hline
\textbf{Testfall -- ID und Bezeichnung} & \textnormal{/T214/ Sortieren der Suche} \\
\hline
\textbf{Testlauf Nr.} & \textnormal{3} \\
\hline
\textbf{Eingaben} & \textnormal{In der Navigationsliste links wird wird das Literaturverzeichnis aufgerufen um zu sortierende Dokumente anzuzeigen. Anschließend wird der link \uline{Autoren} einfach angeklickt.} \\
\hline
\textbf{Soll - Reaktion} & \textnormal{Es sollten nun immernoch die gleichen Dokumente angezeigt werden, aber in alphabetischer Reihenfolge des Autors absteigend geordnet aufgelistet sein .} \\
\hline
\textbf{Ist -- Reaktion} & \textnormal{Die vor liegenden Dokumente sind alphabetisch nach ihrem Autor absteigend sortiert.} \\
\hline
\textbf{Ergebnis} & \textnormal{Da die Soll- und Ist- Reaktion übereinstimmen, kann man den Test als erfolgreich betrachten.} \\
\hline
\textbf{Nacharbeiten } & \textnormal{Ist ein Testlauf nicht erfolgreich
durchgef\"uhrt worden, so werden hier die erforderlichen Nacharbeiten aufgef\"uhrt
(z.B. Bugfixes).} \\
\hline
 \end{longtable}

\subsection{Testfall -- T221 Ausleihe an Externe}
Die folgende Tabelle beschreibt den Testfall: T221 Ausleihe an Externe. \\
\begin{longtable}{|p{5cm}|p{10cm}|}
\hline
\textbf{Testfall -- ID und Bezeichnung} &  \textit{/T221/ Ausleihe an Externe} \\
\hline
\textbf{Zu testende Objekte und Methoden} & \textnormal{
\begin{itemize}
  \item In Komponente \emph{Models} die Datei \lstinline{doc\_assign.html}
  \item In Komponente \emph{Models} die Datei \lstinline{views.py}
\end{itemize}
} \\
\hline
\textbf{Kriterien f\"ur erfolgreiche bzw. fehlgeschlagene Testf\"alle} &
\textit{Der Test ist erfolgreich, wenn ein neuer Eintrag im Table non\_user
erstellt wird und das Dokument diesem zugewiesen wird, oder einem bereits
existierendem dieses zugewiesen wird.} \\
\hline
\textbf{Einzelschritte} &  \textit{
Über das Userinterface wird auf den Testserver zugegriffen.
Es muss ein Dokument auf den derzeit angemeldeten User ausgeliehen sein.
Über die Ausgeliehenen Dokumente des Users kann man die Dokumentinformationen
einsehen wo der Button \uline{Übertragen} zu der benötigten Form führt.
Dort kann ein Externer ausgewählt werden, oder durch Eingabe der benötigten
Informationen neu hinzugefügt werden.} \\
\hline
\textbf{Beobachtungen / Log} &  \textit{
Über den Browser wird die Veränderung der Website beobachtet.
Über das Terminal werden Ladezeiten sowie Schreib-/Lesevorgänge verfolgt.
}\\
\hline
\end{longtable}

In der folgenden Tabelle werden die Testergebnisse zum Testfall: T221 Ausleihe an
Externe beschrieben.

\begin{longtable}{|p{5cm}|p{10cm}|}
\hline
\textbf{Testfall -- ID und Bezeichnung} & \textit{/T221/ Ausleihe an Externe} \\
\hline
\textbf{Testlauf Nr.} & \textit{1} \\
\hline
\textbf{Eingaben} & \textit{
Es wird ein Externer User ausgewählt oder Daten über die Textfelder eingegeben
um einen neuen Externen Nutzer zu erstellen.
} \\
\hline
\textbf{Soll - Reaktion} & \textit{
Der User wird auf die Seite mit den Dokumentinformationen zurückgeleitet,
welches nun an den Externen entliehen ist.
Im Table non\_user wurde der Externe erstellt und/oder ihm das Dokument
zugewiesen.
} \\
\hline
\textbf{Ist -- Reaktion} & \textit{
Seite wird nur neu geladen und nicht auf die Dokumentinformationen geleitet.
Es wurden keine neuen Einträge erstellt oder aktualisiert.
} \\
\hline
\textbf{Ergebnis} & \textit{
Test war nicht erfolgreich aufgrund fehlender Einträge und fehlender
weiterleitung.
} \\
\hline
\textbf{Unvorhergesehene Ereignisse w\"ahrend des Test-laufs } &
\textit{Kein Zurückleiten auf die Dokumentseite} \\
\hline
\textbf{Nacharbeiten } & \textit{Ändern des Links wenn auf den Button \uline{An
Externen übertragen} geklickt wurde} \\
\hline
 \end{longtable}


\subsection{Testfall -- T222 Ausleihe übertragen}
Die folgende Tabelle beschreibt den Testfall: T222 Ausleihe übertragen. \\
\begin{longtable}{|p{5cm}|p{10cm}|}
\hline
\textbf{Testfall -- ID und Bezeichnung} &  \textnormal{/T222/ Ausleihe übertragen} \\
\hline
\textbf{Zu testende Objekte und Methoden} & \textnormal{
\begin{itemize}
  \item In der Komponente \emph{Models} die Datei \lstinline{doc\_assign.html}
  \item In der Komponente \emph{Models} die Datei \lstinline{views.py}
\end{itemize}
} \\
\hline
\textbf{Kriterien f\"ur erfolgreiche bzw. fehlgeschlagene Testf\"alle} &
\textnormal{Der Test ist erfolgreich, wenn das Dokument dem User zugeordent wurde
und nicht mehr in der Liste ausgeliehener Dokumente des aktuellen Users
auftaucht.
Der Test schlägt fehl, wenn das Dokument immernoch dem ursprünglichem User
zugewiesen ist.} \\
\hline
\textbf{Einzelschritte} &  \textnormal{
Über das Userinterface wird auf den Testserver zugegriffen.
Es muss ein Dokument auf einen beliebigen User ausgeliehen sein.
Über die Dokumentinformationen gelangt man über den Button
\uline{Übertragen} zu der benötigten Form.
Dort kann ein User ausgewählt werden, oder das Dokument auf den angemeldeten
Nutzer übertragen werden.} \\
\hline
\textbf{Beobachtungen / Log} &  \textnormal{
Über den Browser wird eine Veränderung der Website beobachtet.
Über das Terminal werden Ladezeiten sowie Schreib-/Lesevorgänge verfolgt.
} \\
\hline
\end{longtable}

In der folgenden Tabelle werden die Testergebnisse zum Testfall: T222
Ausleihe übertragen beschrieben.

\begin{longtable}{|p{5cm}|p{10cm}|}
\hline
\textbf{Testfall -- ID und Bezeichnung} & \textnormal{/T222/ Ausleihe übertragen} \\
\hline
\textbf{Testlauf Nr.} & \textnormal{1} \\
\hline
\textbf{Eingaben} & \textnormal{
Der angemeldete User ist nicht der User, welcher das Dokument
entliehen hat. Eingabe: Angemeldeter User.} \\
\hline
\textbf{Soll - Reaktion} & \textnormal{
Der User wird auf die Seite mit den Dokumentinformationen
zurückgeleitet, welches nun an den bestimmten User entliehen ist.
Das Dokument ist jetzt in der Ausleihliste des Users.
} \\
\hline
\textbf{Ist -- Reaktion} & \textnormal{
Der User befindet sich auf der Seite mit den Dokumentinformationen.
Das Dokument kann von ihm mittels des Buttons \uline{Rückgabe}
zurückgegeben werden. Somit ist es in seiner Ausleihliste.} \\
\hline
\textbf{Ergebnis} & \textnormal{
Test war erfolgreich.} \\
\hline
\end{longtable}

\begin{longtable}{|p{5cm}|p{10cm}|}
\hline
\textbf{Testfall -- ID und Bezeichnung} & \textnormal{/T222/ Ausleihe
übertragen} \\
\hline
\textbf{Testlauf Nr.} & \textnormal{2} \\
\hline
\textbf{Eingaben} & \textnormal{
Der angemeldete User ist der User, welcher das Dokument entliehen hat. Der
User wählt einen anderen aus einer Liste existierender User aus.} \\
\hline
\textbf{Soll - Reaktion} & \textnormal{
Der User wird auf die Seite mit den Dokumentinformationen
zurückgeleitet, welches nun an den bestimmten User entliehen ist.
Das Dokument ist jetzt nicht mehr in der Ausleihliste des Users.
} \\
\hline
\textbf{Ist -- Reaktion} & \textnormal{
Der User befindet sich auf der Seite mit den Dokumentinformationen.
Das Dokument kann nicht mehr von ihm mittels des Buttons \uline{Rückgabe}
zurückgegeben werden. Er kann es aber wieder mittels des Buttons
\uline{Übertragen} auf sich übertragen.} \\
\hline
\textbf{Ergebnis} & \textnormal{
Test war erfolgreich.} \\
\hline
\end{longtable}


\subsection{Testfall -- /T223/ Ausleihe zurückgeben}

In der folgenden Tabelle wird der Test der Ausleihe zurückgeben beschrieben.
\begin{longtable}{|p{5cm}|p{10cm}|}
\hline
\textbf{Testfall -- ID und Bezeichnung} &  \textnormal{/T223/ Ausleihe zurückgeben} \\
\hline
\textbf{Zu testende Objekte und Funktionen} &  
\textnormal{\begin{itemize}
    \item die Webseite \uline{doc\_detail.html},
    \item in Komponente \textit{Models} die Funktion \lstinline{document.unlend()}, 
    \item in Komponente \textit{Models} die Funktion \lstinline{document.set_status()},
    \item in Komponente \textit{Views} die Funktion \lstinline{doc_detail()}
\end{itemize}}
\\
\hline
\textbf{Kriterien f\"ur erfolgreiche bzw. fehlgeschlagene Testf\"alle} &
\textnormal{Zum Einen muss der letzte Eintrag in der Tabelle \glqq doc\_status\grqq\ 
        von diesem Dokument aktualisiert werden.
        Außerdem sollte ein neuer Eintrag in dieser Tabelle existieren.
        Auch sollte sich die Webseite aktualisieren.}  
\\
\hline
\textbf{Einzelschritte} &  
\textnormal{Zuerst sollte gewährleistet sein, dass man angemeldet ist. Danach muss 
        man die \uline{doc\_detail.html} von einem Dokument öffnen, welches man 
        ausgeliehen hat. Schlussendlich wird nur noch auf den Button 
        \uline{Rückgabe} geklickt.} 
\\
\hline
\textbf{Beobachtungen / Log} &  
\textnormal{Alle Fehler beziehungsweise Logs werden über die Konsole des Servers 
        ausgegeben. }
\\
\hline
\textbf{Besonderheiten } &  
\textnormal{Falls der vorherige Eintrag in \glqq doc\_status \grqq dem neuen bis auf
        den Timestamp gleicht, muss natürlich kein neuer Eintrag eingefügt
        werden. Die Funktionen sollten also komplett ignoriert werden.} 
\\
\hline
 \end{longtable}

Die folgende Tabelle zeigt den Testfall, wenn man ein ausgeliehenes Buch 
zurückgeben möchte.
\begin{longtable}{|p{5cm}|p{10cm}|}
\hline
\textbf{Testfall -- ID und Bezeichnung} & \textnormal{/T223/ Ausleihe zurückgeben} \\
\hline
\textbf{Testlauf Nr.} & \textnormal{1} \\
\hline
\textbf{Eingaben} & 
\textnormal{Erforderliche Eingaben für das Zurückgeben eines Dokumentes ist der
        Datensatz eines Buches, welches an den eingeloggten User ausgeliehen 
        ist, und ein Klick auf den Button \uline{Rückgabe}.}
\\
\hline
\textbf{Soll - Reaktion} & 
\textnormal{Der \glq return\_lend \grq -Wert des letzten Eintrages in \glqq 
        doc\_status \grqq zu diesem Dokument wird auf TRUE gesetzt, ein neuer 
        Eintrag mit dem neuen Status sollte in selbiger Tabelle erstellt werden 
        und der Button \uline{Ausleihen} müsste anstelle von \uline{Rückgabe} 
        und \uline{Übertragen} auf der Webseite dargestellt sein.} 
\\
\hline
\textbf{Ist -- Reaktion} & 
\textnormal{Der Test gibt die gewünschten Soll-Reaktionen zurück.} 
\\
\hline
\textbf{Ergebnis} & 
\textnormal{Der Test ist erfolgreich abgelaufen.} \\
\hline
 \end{longtable}
 
Die folgende Tabelle zeigt den Testfall, wenn man ein Buch zurückgeben möchte,
obwohl man dieses bereits hat (z.B. durch Seitenaktualisierung). 
\begin{longtable}{|p{5cm}|p{10cm}|}
\hline
\textbf{Testfall -- ID und Bezeichnung} & \textnormal{/T223/ Ausleihe zurückgeben} \\
\hline
\textbf{Testlauf Nr.} & \textnormal{1} \\
\hline
\textbf{Eingaben} & 
\textnormal{Erforderliche Eingaben für das Zurückgeben eines Dokumentes ist der
        Datensatz eines Buches, welches an den eingeloggten User ausgeliehen 
        ist, und ein Klick auf den Button \uline{Rückgabe}.}
\\
\hline
\textbf{Soll - Reaktion} & 
\textnormal{Die Seite wird genauso zurückgegeben, wie sie bereits ist. Es wird der
        letzte \glqq doc\_status \grqq -Eintrag nicht verändert und auch kein 
        neuer hinzugefügt.} 
\\
\hline
\textbf{Ist -- Reaktion} & 
\textnormal{Der Test gibt die gewünschten Soll-Reaktionen zurück.} 
\\
\hline
\textbf{Ergebnis} & 
\textnormal{Der Test ist erfolgreich abgelaufen.} \\
\hline
 \end{longtable}


\subsection{Testfall --/T224/: Ausleihe vermisst melden}
\begin{longtable}{|p{5cm}|p{10cm}|}
\hline
\textbf{Testfall -- ID und Bezeichnung} &  \textnormal{/F224/: Ausleihe vermisst melden} \\
\hline
\textbf{Zu testende Objekte und Methoden} &  \textnormal{\begin{itemize}
    \item die Webseite \uline{doc\_detail.html},
    \item in Komponente \textit{Models} die Funktion \lstinline{document.missing()}, 
    \item in Komponente \textit{Models} die Funktion \lstinline{document.set_status()},
    \item in Komponente \textit{Views} die Funktion \lstinline{doc_detail()},
    \end{itemize}}
\\
\hline
\textbf{Kriterien f\"ur erfolgreiche bzw. fehlgeschlagene Testf\"alle} &
\textnormal{Erfolgreich: Dokument wird als \glqq vermisst\grqq in der DB gespeichert, 
die Webseite reflektiert das Dokument als vermisst, jeder registrierte Benutzer
bekommt eine Vermisst Meldung per E-Mail zugesendet. } \\
\hline
\textbf{Einzelschritte} &  \textnormal{Nach Überprüfung, ob man angemeldet ist, öffnet
man die \uline{doc\_detail.html} von einem Dokument. Nun wird auf den Button
\glqq Als Vermisst melden\grqq geklickt.} \\
\hline
\textbf{Beobachtungen / Log} &  \textnormal{Fehler werden über die Konsole ausgegeben, 
der Inhalt der E-Mail wird über \textbf{stdout} ausgegeben, sowie auch alle Adressaten.} \\
\hline
\textbf{Besonderheiten } &  \textnormal{Falls der vorherige Eintrag in \glqq doc\_status \grqq dem neuen bis auf
        den Timestamp gleicht, muss natürlich kein neuer Eintrag eingefügt
        werden. Die Funktionen sollten also komplett ignoriert werden.} \\
\hline


 \end{longtable}

Die folgende Tabelle zeigt den Testfall, wenn man ein Buch vermisst meldet.
\begin{longtable}{|p{5cm}|p{10cm}|}
\hline
\textbf{Testfall -- ID und Bezeichnung} & \textnormal{/T224/: Ausleihe vermisst melden} \\
\hline
\textbf{Testlauf Nr.} & \textnormal{1} \\
\hline
\textbf{Eingaben} & \textnormal{Erforderliche Eingaben für das Vermisst Melden eines Dokumentes ist der
        Datensatz eines Buches, das Recht des Vermisstmeldens beim angemeldeten 
        User und ein Klick auf den Button \uline{als vermisst melden}. Es muss zudem 
        mindestens eine E-Mail in der DB stehen.} \\
\hline
\textbf{Soll - Reaktion} & \textnormal{Der \glq return\_lend \grq -Wert des letzten Eintrages in \glqq 
        doc\_status \grqq zu diesem Dokument wird auf TRUE gesetzt, ein neuer 
        Eintrag mit dem neuen Status sollte in selbiger Tabelle erstellt werden 
        und es sollte nur noch ein Button \uline{Gefunden melden} erscheinen für 
        User mit dem entsprechenden Recht. Außerdem sollten alle registrierten Nutzer
        eine E-Mail mit der Vermisst Meldung bekommen.).
} \\
\hline
\textbf{Ist -- Reaktion} & \textnormal{Der Test gibt die gewünschten Soll-Reaktionen zurück.} \\
\hline
\textbf{Ergebnis} & \textnormal{Der Test war erfolgreich} \\
\hline
 \end{longtable}
 
Die folgende Tabelle zeigt den Testfall, wenn man ein Buch als vermisst melden
möchte, obwohl man dieses bereits getan hat (z.B. durch Seitenaktualisierung).
\begin{longtable}{|p{5cm}|p{10cm}|}
\hline
\textbf{Testfall -- ID und Bezeichnung} & 
\textnormal{/T224/: Ausleihe vermisst melden} 
\\
\hline
\textbf{Testlauf Nr.} & \textnormal{2} \\
\hline
\textbf{Eingaben} & 
\textnormal{Erforderliche Eingaben für das Vermisst Melden eines Dokumentes ist der
        Datensatz eines Buches, das Recht des Vermisstmeldens beim angemeldeten 
        User und ein Klick auf den Button \uline{als vermisst melden}. Es muss zudem 
        mindestens eine E-Mail in der DB stehen. }
\\
\hline
\textbf{Soll - Reaktion} & 
\textnormal{Die Seite wird genauso zurückgegeben, wie sie bereits ist. Es wird der
        letzte \glqq doc\_status \grqq -Eintrag nicht verändert und auch kein 
        neuer hinzugefügt.} 
\\
\hline
\textbf{Ist -- Reaktion} & 
\textnormal{Der Test gibt die gewünschten Soll-Reaktionen zurück.} 
\\
\hline
\textbf{Ergebnis} & 
\textnormal{Der Test ist erfolgreich abgelaufen.} \\
\hline
 \end{longtable}
 
 


\subsection{Testfall -- /T225/ Ausleihe verloren melden}

Die folgende Tabelle beschreibt den Test der Ausleihe verloren melden. \\
\begin{longtable}{|p{5cm}|p{10cm}|}
\hline
\textbf{Testfall -- ID und Bezeichnung} &  
\textnormal{/F225/: Ausleihe verloren melden} 
\\
\hline
\textbf{Zu testende Objekte und Methoden} &  
\textnormal{\begin{itemize}
    \item die Webseite \uline{doc\_detail.html},
    \item in Komponente \textit{Models} die Funktion \lstinline{document.lend()}, 
    \item in Komponente \textit{Models} die Funktion \lstinline{document.set_status()},
    \item in Komponente \textit{Views} die Funktion \lstinline{doc_detail()},
    \end{itemize}}
\\
\hline
\textbf{Kriterien f\"ur erfolgreiche bzw. fehlgeschlagene Testf\"alle} &
\textnormal{Zum Einen muss der letzte Eintrag in der Tabelle \glqq doc\_status \grqq\
        von diesem Dokument aktualisiert werden.
        Außerdem sollte ein neuer Eintrag in dieser Tabelle existieren.
        Auch sollte sich die Webseite aktualisieren.}
\\
\hline
\textbf{Einzelschritte} &  
\textnormal{Zuerst sollte gewährleistet sein, dass man angemeldet ist. Danach muss 
        man die \uline{doc\_detail.html} von einem Dokument öffnen, welches man 
        ausgeliehen hat. Schlussendlich wird nur noch auf den Button 
        \uline{Verloren melden} geklickt, der nur angezeigt wird, wenn man das 
        entsprechende Recht besitzt.}
\\
\hline
\textbf{Beobachtungen / Log} &  
\textnormal{Alle Fehler beziehungsweise Logs werden über die Konsole des Servers 
        ausgegeben.} 
\\
\hline
\textbf{Besonderheiten } &  
\textnormal{Falls der vorherige Eintrag in \glqq doc\_status \grqq dem neuen bis auf
        den Timestamp gleicht, muss natürlich kein neuer Eintrag eingefügt
        werden. Die Funktionen sollten also komplett ignoriert werden.} 
\\
\hline

 \end{longtable}

Die folgende Tabelle zeigt den Testfall, wenn man ein Buch verloren meldet.
\begin{longtable}{|p{5cm}|p{10cm}|}
\hline
\textbf{Testfall -- ID und Bezeichnung} & 
\textnormal{/T225/: Ausleihe verloren melden} 
\\
\hline
\textbf{Testlauf Nr.} & \textnormal{1} \\
\hline
\textbf{Eingaben} & 
\textnormal{Erforderliche Eingaben für das Verlorenmelden eines Dokumentes ist der
        Datensatz eines Buches, das Recht des Verlorenmeldens beim angemeldeten 
        User und ein Klick auf den Button \uline{Verloren melden}.}
\\
\hline
\textbf{Soll - Reaktion} & 
\textnormal{Der \glq return\_lend \grq -Wert des letzten Eintrages in \glqq 
        doc\_status \grqq zu diesem Dokument wird auf TRUE gesetzt, ein neuer 
        Eintrag mit dem neuen Status sollte in selbiger Tabelle erstellt werden 
        und es sollte nur noch ein Button \uline{Gefunden melden} erscheinen für 
        User mit dem entsprechenden Recht.}
\\
\hline
\textbf{Ist -- Reaktion} & 
\textnormal{Der Test gibt die gewünschten Soll-Reaktionen zurück.} 
\\
\hline
\textbf{Ergebnis} & 
\textnormal{Der Test ist erfolgreich abgelaufen.} \\
\hline
 \end{longtable}
 
Die folgende Tabelle zeigt den Testfall, wenn man ein Buch verloren melden
möchte, obwohl man dieses bereits hat (z.B. durch Seitenaktualisierung). 
\begin{longtable}{|p{5cm}|p{10cm}|}
\hline
\textbf{Testfall -- ID und Bezeichnung} & 
\textnormal{/T225/: Ausleihe verloren melden} 
\\
\hline
\textbf{Testlauf Nr.} & \textnormal{2} \\
\hline
\textbf{Eingaben} & 
\textnormal{Erforderliche Eingaben für das Verlorenmelden eines Dokumentes ist der
        Datensatz eines Buches, das Recht des Verlorenmeldens beim angemeldeten 
        User und ein Klick auf den Button \uline{Verloren melden}.}
\\
\hline
\textbf{Soll - Reaktion} & 
\textnormal{Die Seite wird genauso zurückgegeben, wie sie bereits ist. Es wird der
        letzte \glqq doc\_status \grqq -Eintrag nicht verändert und auch kein 
        neuer hinzugefügt.} 
\\
\hline
\textbf{Ist -- Reaktion} & 
\textnormal{Der Test gibt die gewünschten Soll-Reaktionen zurück.} 
\\
\hline
\textbf{Ergebnis} & 
\textnormal{Der Test ist erfolgreich abgelaufen.} \\
\hline
 \end{longtable}



\subsection{Testfall -- /T226/: Ausleihfrist abgelaufen}

Die folgende Tabelle beschreibt den Test der Ausleihe verloren melden. \\
\begin{longtable}{|p{5cm}|p{10cm}|}
\hline
\textbf{Testfall -- ID und Bezeichnung} &  
\textit{/T226/: Ausleihfrist abgelaufen} \\
\hline
\textbf{Zu testende Objekte und Methoden} &  
\textit{\begin{itemize}
    \item in Komponente \emph{Views} die Funktion \lstinline{lending_expired}
    \end{itemize}}
\\
\hline
\textbf{Kriterien f\"ur erfolgreiche bzw. fehlgeschlagene Testf\"alle} &
\textit{Es sollte eine E-Mail an die angegebene E-Mail-Adresse vom Externen und 
        vom Bürgen gesendet, wenn die Ausleihfrist abgelaufen ist.} 
\\
\hline
\textbf{Einzelschritte} &  
\textit{Man starte den Scheduler, der die entsprechende Funktion öffnet, die 
        Ausleihzeiten überprüft und gegebenenfalls die E-Mails versendet. } 
\\
\hline
\textbf{Beobachtungen / Log} &  
\textit{Alle Fehler beziehungsweise Logs werden über die Konsole des Servers 
        ausgegeben.} 
\\
\hline

 \end{longtable}

Die folgende Tabelle beschreibt den Test, falls Fristen abgelaufen sind.
\begin{longtable}{|p{5cm}|p{10cm}|}
\hline
\textbf{Testfall -- ID und Bezeichnung} & \textit{/T226/: Ausleihfrist abgelaufen} \\
\hline
\textbf{Testlauf Nr.} & \textit{1} \\
\hline
\textbf{Eingaben} & 
\textit{Die Eingabedaten sind das aktuelle Datum, die Werte der Tabelle \glqq
        doc\_status\grqq, sämtliche E-Mail-Adressen aus \glqq Auth.Users\grqq\ 
        und \glqq non\_user\grqq\ und der eigentliche E-Mail-Text aus \glqq
        emails\grqq.}
\\
\hline
\textbf{Soll - Reaktion} & 
\textit{Es soll an alle Externe, deren Ausleihfrist abgelaufen ist, und deren 
        Bürgen jeweils eine für den entsprechenden Nutzer entsprechende E-Mail 
        versendet werden.
} \\
\hline
\textbf{Ist -- Reaktion} & 
\textit{Die Funktion reagiert so, wie sie soll.} \\
\hline
\textbf{Ergebnis} & 
\textit{Der Test war erfolgreich.} \\
\hline
 \end{longtable}


\subsection{Testfall -- /T227/ Ausleihhistorie}
In der folgenden Tabelle wird der Test der in \lstinline{doc_detail}
eingebundenen Funktion \lstinline{__filter_history()} beschrieben.
\begin{longtable}{|p{5cm}|p{10cm}|}
\hline
\textbf{Testfall -- ID und Bezeichnung} &  \textnormal{/T227/ Ausleihhistorie} \\
\hline
\textbf{Zu testende Objekte und Methoden} & 
\textnormal{
\begin{itemize}
  \item In Komponente \textit{views} die Funktion
	\lstinline{__filter_history()}
  \item In Komponente \textit{views} die Funktion \lstinline{doc_detail()}
  \item In Komponente \textit{Template} die Datei \textit{doc\_detail.html}
\end{itemize}}
\\
\hline
\textbf{Kriterien f\"ur erfolgreiche bzw. fehlgeschlagene Testf\"alle} &
\textnormal{Wenn in dem \textit{View} doc\_detail die vorhandene Ausleihhistorie angezeigt wird,
so ist der Test geglückt. Andernfalls ist der Test fehlgeschlagen. } \\
\hline
\textbf{Einzelschritte} &  \textnormal{Zuerst wird die Webseite in einem Browser
aufgerufen. Falls der eingeloggte Benutzer nicht über ausreichend Rechte
besitzt, so werden ausschließlich die Attribute des Dokumentes angezeigt.
Andernfalls wird die Funktion \lstinline{__filter_history()} aufgerufen und die
Historie auf der Webseite ausgegeben. } \\
\hline
\textbf{Beobachtungen / Log} &  \textnormal{Der Server loggt in der Konsole, dass
doc\_detail angefodert wurde. Informationen über \lstinline{__filter_history()}
werden nicht ausgegeben. } \\
\hline

 \end{longtable}

Im Folgenden werden die beiden möglichen erfolgreichen Testfälle für die Ausleihhistorie beschrieben. \\


\begin{longtable}{|p{5cm}|p{10cm}|}
\hline
\textbf{Testfall -- ID und Bezeichnung} & \textnormal{/T227/ Ausleihhistorie} \\
\hline
\textbf{Testlauf Nr.} & \textnormal{1} \\
\hline
\textbf{Eingaben} & \textnormal{Eingaben sind vorhandene Einträge in der Datenbank
bezüglich des angeforderten Dokumentes. Weiterhin hat der eingeloggte Nutzer
ausreichend Rechte, um die Historie des Dokumentes einzusehen. } \\
\hline
\textbf{Soll - Reaktion} & \textnormal{Der Benutzer bekommt die Ausleihhistorie des
Dokumentes angezeigt. } \\
\hline
\textbf{Ist -- Reaktion} & \textnormal{Wie Soll - Reaktion} \\
\hline
\textbf{Ergebnis} & \textnormal{Der Test verlief erfolgreich. } \\
\hline
 \end{longtable}

 Im folgenden wird geprüft ob es Möglich ist die History eines Dokumentes ohne
 ausreichende Rechte einzusehen.
\begin{longtable}{|p{5cm}|p{10cm}|}
\hline
\textbf{Testfall -- ID und Bezeichnung} & \textnormal{/T227/ Ausleihhistorie} \\
\hline
\textbf{Testlauf Nr.} & \textnormal{2} \\
\hline
\textbf{Eingaben} & \textnormal{Eingaben sind vorhandene Einträge in der Datenbank
bezüglich des angeforderten Dokumentes. Weiterhin hat der eingeloggte Nutzer
nicht ausreichend Rechte, um die Historie des Dokumentes einzusehen. } \\
\hline
\textbf{Soll - Reaktion} & \textnormal{Der Benutzer bekommt die Ausleihhistorie des
Dokumentes nicht angezeigt. } \\
\hline
\textbf{Ist -- Reaktion} & \textnormal{Wie Soll - Reaktion} \\
\hline
\textbf{Ergebnis} & \textnormal{Der Test verlief erfolgreich. } \\
\hline
 \end{longtable}

\subsection{T228 Derzeitiger Leihender}

Dieser Testfall testet die Profilseite eines User der nicht der angemeldete
ist. Auf diese Weise werden die Informationen eines derzeitig Leihenden
dargestellt.

\begin{longtable}{|p{5cm}|p{10cm}|}
\hline
\textbf{Testfall -- ID und Bezeichnung} &  \textit{T228 Derzeitiger Leihender} \\
\hline
\textbf{Zu testende Objekte und Methoden} &  \textnormal{
\begin{itemize}
\item In Komponente \emph{Views} die Datei
  \lstinline{stranger_profile.html}
\item In Komponente \emph{Templates} die Funktion
  \lstinline{profile}
\end{itemize}
}
\\
\hline
\textbf{Kriterien f\"ur erfolgreiche bzw. fehlgeschlagene Testf\"alle} &
\textit{Der Test ist erfolgreich, falls das Profil richtig angezeigt wird.} \\
\hline
\textbf{Einzelschritte} &  \textit{Zuerst logt man sich mit einem User
ein und ruft dann die Seite eines anderen Users auf. Dies wird mit
jedem im Testplan aufgeführten Browser durchgeführt.
} \\
\hline
\textbf{Beobachtungen / Log} &  \textit{Beobachtungen werden über das Terminal
und den Browser vorgenommen.} \\
\hline

\end{longtable}

Der User \emph{admin} für diesen Testlauf hat alle vorhandenen Rechte. 

\begin{longtable}{|p{5cm}|p{10cm}|}
\hline
\textbf{Testfall -- ID und Bezeichnung} & \textit{/T228/ Derzeitiger
Leihender} \\
\hline
\textbf{Testlauf Nr.} & \textit{1} \\
\hline
\textbf{Eingaben} & \textit{Es wird der User \emph{admin} güt den Login verwendet} \\
\hline
\textbf{Soll - Reaktion} & \textit{Das komplette Profil mit Informationen über
Gruppenzugehörigkeit des Profilinhabers werden dargestellt.
} \\
\hline
\textbf{Ist -- Reaktion} & \textit{Entspricht der Soll-Reaktion} \\
\hline
\textbf{Ergebnis} & \textit{Der Test war erfolgreich} \\
\hline
 \end{longtable}

Der User \emph{foo} für diesen Testlauf hat die Rechte eines normalen Anwenders
des Programmes. 

\begin{longtable}{|p{5cm}|p{10cm}|}
\hline
\textbf{Testfall -- ID und Bezeichnung} & \textit{/T228/ Derzeitiger
Leihender} \\
\hline
\textbf{Testlauf Nr.} & \textit{2} \\
\hline
\textbf{Eingaben} & \textit{Es wird der User \emph{foo} für den Login verwendet} \\
\hline
\textbf{Soll - Reaktion} & \textit{Das komplette Profil ohne Informationen über
Gruppenzugehörigkeit des Profilinhabers werden dargestellt.
} \\
\hline
\textbf{Ist -- Reaktion} & \textit{Entspricht der Soll-Reaktion} \\
\hline
\textbf{Ergebnis} & \textit{Der Test war erfolgreich} \\
\hline
 \end{longtable}

Der User \emph{gast} für diesen Testlauf modelliert einen unregistrierten
Benutzer und deshalb keinerlei Rechte. 

\begin{longtable}{|p{5cm}|p{10cm}|}
\hline
\textbf{Testfall -- ID und Bezeichnung} & \textit{/T228/ Derzeitiger
Leihender} \\
\hline
\textbf{Testlauf Nr.} & \textit{3} \\
\hline
\textbf{Eingaben} & \textit{Es wird der User \emph{gast} für den Login verwendet} \\
\hline
\textbf{Soll - Reaktion} & \textit{Der Zugriff auf die Seite wird verweigert
} \\
\hline
\textbf{Ist -- Reaktion} & \textit{Entspricht der Soll-Reaktion} \\
\hline
\textbf{Ergebnis} & \textit{Der Test war erfolgreich} \\
\hline
 \end{longtable}

\subsection{Testfall -- T229 Entleihliste}
In der folgenden Tabelle wird der Test der Entleihliste beschrieben. 
\begin{longtable}{|p{5cm}|p{10cm}|}
\hline
\textbf{Testfall -- T229} &  \textnormal{ /T229/ Entleihliste} \\
\hline
\textbf{Zu testende Objekte und Methoden} &  
\textnormal{ 
\begin{itemize}
\item In Komponente \textit{views.py} die Funktion \lstinline {doc_rent()}
\item In Komponente \textit{models.py} die Funktion \lstinline {doc_status()}
\end{itemize} }\\
\hline
\textbf{Kriterien f\"ur erfolgreiche bzw. fehlgeschlagene Testf\"alle} &
\textnormal{Es liegt ein erfolgreicher Test vor, wenn eine Auflistung der tatsächlich
ausgeliehenen Dokumente, im richtigen Format und in genauer Anzahl angezeigt wird.
Es wird vorrausgesetzt, dass der angemeldete Benutzer über die Rechte besitzt, 
um Bücher auszuleihen.
Der Test ist fehlgeschlagen, wenn diese nicht auftreten. } \\
\hline
\textbf{Einzelschritte} &  
\textnormal{Die Webseite wird im Browser aufgerufen. 
Nach dem Anmelden (Seitenbutton \uline{Anmelden}) auf einem Testaccount mit den 
dazu benötigten Rechten, soll der User einige Dokumente aus dem 
Literaturverzeichnis ausleihen. Durch Navigieren der Seite 
(\uline{Profil} -> \uline{Ausleihliste}) soll der User eine Liste seiner aktuell ausgeliehenen 
Dokumente vorfinden.
Sollten unvorhergesehene Ereignisse auftreten, kann man den laufenden Testprozess
durch Interaktion mit der Konsole mit dem Django-internen Befehl Ctrl-D terminieren
und gegebenenfalls den Server für einen erneuten Test neustarten.      
} \\
\hline
\textbf{Beobachtungen / Log} &  \textnormal{Ergebnisse und Fehler des Testlaufs 
werden innerhalb vom Django-Projekt im Browser und im Termimal angezeigt.}\\ 
\hline
\end{longtable}

Die folgende Tabelle dokumentiert den erfolgreichen Testlauf der Entleihliste


\begin{longtable}{|p{5cm}|p{10cm}|}
\hline
\textbf{Testfall -- ID und Bezeichnung} & \textnormal{T229} \\
\hline
\textbf{Testlauf Nr.} & \textnormal{1} \\
\hline
\textbf{Eingaben} & \textnormal{Für das Anzeigen der Ausleihliste sind folgende 
Eingaben von Nöten: Einträge (Dokumente, User, etc) in der Datenbank, 
Erfolgreiches Anmelden eines registrierten Benutzers, Bestätigen des Links 
\uline{Literaturverzeichnis}, Auswählen eines oder mehreren Dokumentes 
(Klick auf den Titel des Dokumentes), Bestätigen des Buttons \uline{Ausleihen},   
Aufrufen des \uline{Profil}-Links und der \uline{Ausleihliste}.   
   } \\
\hline
\textbf{Soll - Reaktion} & \textnormal{Die Ausleihliste soll die gewünschten
Dokumente anzeigen und im korrekten Format darstellen (Anordnung von Titel, 
Autor, Jahr, etc). } \\
\hline
\textbf{Ist -- Reaktion} & \textnormal{Dokumente werden in Ausleihliste angezeigt.} \\
\hline
\textbf{Ergebnis} & \textnormal{Der Test ist erfolgreich abgelaufen} \\
\hline
\textbf{Nacharbeiten } & \textnormal{Kein Bugfix notwendig.} \\
\hline
\end{longtable}


\subsection{Testfall -- ID und Bezeichnung}
Jeder Testfall erh\"alt eine eindeutige Identifikation mit Kurzbezeichnung.\\
Beispiel: /T100/ Lager anlegen\\
Die folgende Tabelle beschreibt den Testfall. \\
\begin{longtable}{|p{5cm}|p{10cm}|}
\hline
\textbf{Testfall -- ID und Bezeichnung} &  \textit{Beispiel:
                                                        /T100/ Lager anlegen} \\
\hline
\textbf{Zu testende Objekte und Methoden} &  \textit{Hier sind alle Testobjekte
und Methoden zu beschreiben, die von diesem Testfall ausgef\"uhrt werden.
Testobjekte k\"onnen dabei z.B. auch Komponenten oder einzelne Webseiten sein.}
\\
\hline
\textbf{Kriterien f\"ur erfolgreiche bzw. fehlgeschlagene Testf\"alle} &
\textit{Es sind die Kriterien anzugeben, mit denen man feststellt, dass der
Testfall erfolgreich bzw. fehlgeschlagen ist. } \\
\hline
\textbf{Einzelschritte} &  \textit{Es ist zu beschreiben, was zu tun ist, um
einen Testlauf vorzubereiten und ihn zu starten.
Ggf. sind erforderliche Schritte w\"ahrend seiner Ausf\"uhrung anzugeben (z.B.
Benutzerinteraktion \"uber ein User-Interface). Ferner ist zu beschreiben, was zu
tun ist, um den Testlauf ordnungsgem\"aß oder im Falle unvorhergesehener
Ereignisse anzuhalten (falls er nicht von selbst terminiert).
Ggf. sind Aufr\"aumarbeiten zu beschreiben, um nach den Tests den urspr\"unglichen
Zustand wiederherzustellen (falls der Testlauf nicht seiteneffektfrei ist)
} \\
\hline
\textbf{Beobachtungen / Log} &  \textit{Es sind alle speziellen Methoden oder
Formate zu beschreiben, mit denen die Ergebnisse der Testl\"aufe, die
Zwischenf\"alle und sonstige wichtige Ereignisse aufgenommen werden sollen.
Beispiel: Logdatei eines Servers, Messung der Antwortzeit eines Remote
Terminals mittels Netzwerk Simulator, \ldots} \\
\hline
\textbf{Besonderheiten } &  \textit{optional; auszuf\"ullen, falls es
Besonderheiten in diesem Testfall gibt.
Testfallspezifische Besonderheiten, z.B. Ausf\"uhrungsvorschriften oder
Abweichungen von der Testumgebung (siehe 2.5)  werden hier aufgelistet.} \\
\hline
\textbf{Abh\"angigkeiten} &  \textit{optional; auszuf\"ullen, falls es
Abh\"angigkeiten in diesem Testfall gibt
Ist dieser Testfall von der Ausf\"uhrung anderer Testf\"alle abh\"angig, so werden
diese Testf\"alle hier aufgelistet und kurz beschrieben, worin die Abh\"angigkeit
besteht.} \\
\hline

 \end{longtable}

Die folgenden Tabellen beschreiben, wie der Testfall ausgef\"uhrt wurde und
welches Ergebnis er geliefert hat. Da es bei Korrektur von Softwarefehlern oder
anderen Gegebenheiten notwendig ist, einen Test mehrfach durchzuf\"uhren
(Testl\"aufe), ist jede Testdurchf\"uhrung zu dokumentieren. Daher ist diese
Tabelle f\"ur \textbf{jeden Testlauf }  zu erstellen und \textbf{ fortlaufend zu
nummerieren}. \\


\begin{longtable}{|p{5cm}|p{10cm}|}
\hline
\textbf{Testfall -- ID und Bezeichnung} & \textit{Beispiel: /T100/ Lager
anlegen} \\
\hline
\textbf{Testlauf Nr.} & \textit{Beispiel: 1} \\
\hline
\textbf{Eingaben} & \textit{Es sind alle Eingabedaten bzw. andere Aktionen
aufzuf\"uh-ren, die f\"ur die Ausf\"uhrung des Testfalls notwendig sind.
Diese k\"onnen sowohl als Wert angegeben werden (ggf. mit Toleranzen) als auch
als Name, falls es sich um konstante Tabellen oder um Dateien handelt. Außerdem
sind alle betroffenen Datenbanken, Dateien, Terminal Meldungen, etc. anzugeben.
Hinweis: Es sind nicht noch mal die Einzelschritte aus 3.1.3 zu wiederholen.
W\"ahrend jene allgemeiner sind (z.B. "`Ein-loggen \"uber das Login-Formular"')
sind hier die konkreten eingegebenen Testdaten aufzuf\"uhren (z.B. "`Loginname:
test; Passwort: xxxtest"'`). } \\
\hline
\textbf{Soll - Reaktion} & \textit{Hier ist anzugeben, welches Ergebnis bzw.
Ausgabe der Test haben soll.
Hinweis: Es sind nicht noch mal die Erfolgskriterien aus 3.1.2 zu wiederholen.
W\"ahrend jene allgemeiner sind (z.B. "`Testnachricht wird \"uber Netzwerkkanal
empfangen"') sind hier die konkreten erhaltenen Testdaten aufzuf\"uhren (z.B.
Konsole zeigt Meldung: "`Testnachricht 123 erhalten"').
} \\
\hline
\textbf{Ist -- Reaktion} & \textit{Hier ist anzugeben, welches Ergebnisdaten
bzw. Ausgaben dieser Testlauf geliefert hat.} \\
\hline
\textbf{Ergebnis} & \textit{F\"ur jeden Testlauf ist zu vermerken, ob der Test
erfolgreich durchgef\"uhrt werden konnte oder nicht. Einen missgl\"uckten Test
bitte begr\"unden, sofern der Grund des Fehlschlags bekannt oder offensichtlich
ist.} \\
\hline
\textbf{Unvorhergesehene Ereignisse w\"ahrend des Test-laufs } &
\textit{optional; nur anzugeben, falls es unvorhergesehene Ereig-nisse gab} \\
\hline
\textbf{Nacharbeiten } & \textit{Ist ein Testlauf nicht erfolgreich
durchgef\"uhrt worden, so werden hier die erforderlichen Nacharbeiten aufgef\"uhrt
(z.B. Bugfixes).} \\
\hline
 \end{longtable}


\subsection{Testfall -- /T231/ Universitätsbibliotheks - Export}
In der folgenden Tabelle wird der Test des Exports in das
Universitätsbibliotheks-Format ADT beschrieben.
\begin{longtable}{|p{5cm}|p{10cm}|}
\hline
\textbf{Testfall -- ID und Bezeichnung} &  \textit{/T102/ Universitätsbibliotheks - Export} \\
\hline
\textbf{Zu testende Objekte und Funktionen} & 
\textit{
\begin{itemize}
  \item In Komponente \emph{Server (App: Documents)} die Funktion
	\lstinline{extras_allegro.export_allegro()}
\end{itemize} } \\
\hline
\textbf{Kriterien f\"ur erfolgreiche bzw. fehlgeschlagene Testf\"alle} &
\textit{Der Test ist erfolgreich, wenn eine Datei im ADT-Format ausgegeben
wird, ausser es existieren keine Dokumente in der Datenbank, die exportiert
werden müssten. Weiterhin müssen alle Einträge in der Datei auf Korrektheit
überprüft werden. Vorbedingung ist, dass der eingeloggte Benutzer über
ausreichend Rechte verfügt, um den Export durchzuführen. Falls zu exportierende
Dokumente in der Datenbank vorhanden sind, aber keine neue .ADT-Datei angelegt
wurde, oder der eingeloggte Benutzer verfügt nicht über ausreichend Rechte, so
ist der Test fehlgeschlagen. } \\
\hline
\textbf{Einzelschritte} & 
\textit{Zuerst wird die Webseite in einem Browser aufgerufen. Falls der
eingeloggte Benutzer nicht ausreichend Rechte besitzt, so wird der Export
abgebrochen. Durch den Klick auf \uline{Allegro-Export starten} wird die Datei angelegt und
zum Download angeboten. Es wird eine Meldung ausgegeben, falls es aktuell keine
Dokumente zu exportieren gibt. Anschließend wird auf Serverseite die Datei auf
Korrektheit überprüft. } \\
\hline
\textbf{Beobachtungen / Log} &  \textit{Alle Fehler beziehungsweise Logs werden
über die Konsole des Servers ausgegeben. } \\
\hline

 \end{longtable}

Im Folgenden werden die beiden möglichen erfolgreichen Testläufe für den
Universitätsbibliotheks - Export beschrieben. Angenommen wird dabei, dass
Django, der Server, sowie alle benötigten Komponenten korrekt konfiguriert
sind, da somit Fehler durch die Umgebung ausgeschlossen sind.

\begin{longtable}{|p{5cm}|p{10cm}|}
\hline
\textbf{Testfall -- ID und Bezeichnung} &
\textit{/T231/ Universitätsbibliotheks - Export} \\
\hline 
\textbf{Testlauf Nr.} & \textit{1} \\
\hline
\textbf{Eingaben} & \textit{Erforderliche Eingaben für den Export sind Einträge in der
Datenbank, ein eingeloggter Benutzer und ein Klick auf \uline{Allegro-Export
starten}. Weitere Eingaben sind nicht notwendig. } \\
\hline
\textbf{Soll - Reaktion} & \textit{Die neu erstellte .ADT-Datei wird erstellt
und über den Browser zum Download angeboten, oder es wird die Meldung gegeben,
dass keine Dokumente exportiert werden müssen. } \\
\hline
\textbf{Ist -- Reaktion} & \textit{Die .ADT-Datei wird erstellt und zum
Download angeboten. } \\
\hline
\textbf{Ergebnis} & \textit{Der Test ist erfolgreich abgelaufen. } \\
\hline
\end{longtable}

\begin{longtable}{|p{5cm}|p{10cm}|}
\hline
\textbf{Testfall -- ID und Bezeichnung} &
\textit{/T231/ Universitätsbibliotheks - Export} \\
\hline 
\textbf{Testlauf Nr.} & \textit{2} \\
\hline
\textbf{Eingaben} & \textit{Erforderliche Eingaben für den Export sind Einträge in der
Datenbank, ein eingeloggter Benutzer und ein Klick auf \uline{Allegro-Export
starten}. Weitere Eingaben sind nicht notwendig. } \\
\hline
\textbf{Soll - Reaktion} & \textit{Die neu erstellte .ADT-Datei wird erstellt
und über den Browser zum Download angeboten, oder es wird die Meldung gegeben,
dass keine Dokumente exportiert werden müssen. } \\
\hline
\textbf{Ist -- Reaktion} & \textit{Es wird die Meldung gegeben, dass keine
Dokumente exportiert werden müssen. } \\
\hline
\textbf{Ergebnis} & \textit{Der Test ist erfolgreich abgelaufen. } \\
\hline
\end{longtable}


