% Kapitel 3 mit den entsprechenden Unterkapiteln
% Die Unterkapitel k\"onnen auch in separaten Dateien stehen,
% die dann mit dem \include-Befehl eingebunden werden.
%----------------------------------------------------------------------------

\chapter{Testdurchf\"uhrung}

%In diesem Abschnitt werden die einzelnen Testf\"alle beschrieben und deren
%Durchf\"uhrungen (=Testl\"aufe) protokolliert.\\
%Ein Testfall ist eine Kombination von Eingabedaten, Bedingungen und erwarteten
%Ausgaben, die einen bestimmten Zweck erf\"ullen. Man pr\"uft z.B., ob Vorgaben in
%einem Spezifikationsdokument eingehalten werden oder ob der Programmablauf
%tats\"achlich dem erwarteten Pfad entspricht.\\
%Dieses Kapitel enth\"alt drei Arten von Tabellen:\\
%1.  Die \"Ubersichtstabelle zeigt an, welche Testf\"alle es gibt und welcher
%Testfall welche Objekte, Methoden oder Anforderungen testet. So hat man den
%\"Uberblick, Verfolgbarkeit zwischen der Testdokumentation und anderen
%Dokumenten, und man kann sehen, ob die Testf\"alle vollst\"andig sind.\\
%2.  Der Testfall beschreibt jeden einzelnen Testfall im Detail.\\
%3.  Der Testlauf beschreibt eine Durchf\"uhrung eines Testfalls. Derselbe
%Testfall kann mit verschiedenen Eingabedaten oder auch mit verschiedenen
%Softwareversionen mehrmals durchgef\"uhrt werden.\\

\section{\"Ubersichtstabelle}
  %In der folgenden Tabelle sind entweder f\"ur alle Testf\"alle die zu testende
  %Komponente oder die zu testende Funktion (oder beides) anzugeben. Die
  %Bezeichnungen der Komponenten m\"ussen konsistent sein mit denen in Fein- und
  %Grobkonzept, um die Verfolgbarkeit zum Konzept sicherzustellen. Die IDs und
  %Bezeichnungen der Funktionen m\"ussen denen im Pflichtenheft entsprechen, um
  %die Verfolgbarkeit zu den Anforderungen sicherzustellen. \\
\begin{tabular}{|c|c|c|}
\hline
\textbf{Testfall ID und Bezeichnung} &  \textbf {Zu testende Komponente} &
\textbf {Zu testende Funktion}\\
\hline
/T212/ Erweiterte Suche &  Views und Template  & /F212/ Erweiterte Suche \\
\hline
/T228/ Derzeitiger Leihender &  Views und Template  &
/F228/ Derzeitiger Leihender\\
\hline
&&
\end{tabular}

% Im Folgenden sind so viele Unterkapitel einzuf\"ugen, wie es Testf\"alle gibt.\\

\section{Testfall -- ID und Bezeichnung}
Jeder Testfall erh\"alt eine eindeutige Identifikation mit Kurzbezeichnung.\\
Beispiel: /T100/ Lager anlegen\\
Die folgende Tabelle beschreibt den Testfall. \\
\begin{longtable}{|p{7cm}|p{10cm}|}
\hline
\textbf{Testfall -- ID und Bezeichnung} &  \textit{Beispiel:
                                                        /T100/ Lager anlegen} \\
\hline
\textbf{Zu testende Objekte und Methoden} &  \textit{Hier sind alle Testobjekte
und Methoden zu beschreiben, die von diesem Testfall ausgef\"uhrt werden.
Testobjekte k\"onnen dabei z.B. auch Komponenten oder einzelne Webseiten sein.}
\\
\hline
\textbf{Kriterien f\"ur erfolgreiche bzw. fehlgeschlagene Testf\"alle} &
\textit{Es sind die Kriterien anzugeben, mit denen man feststellt, dass der
Testfall erfolgreich bzw. fehlgeschlagen ist. } \\
\hline
\textbf{Einzelschritte} &  \textit{Es ist zu beschreiben, was zu tun ist, um
einen Testlauf vorzubereiten und ihn zu starten.
Ggf. sind erforderliche Schritte w\"ahrend seiner Ausf\"uhrung anzugeben (z.B.
Benutzerinteraktion \"uber ein User-Interface). Ferner ist zu beschreiben, was zu
tun ist, um den Testlauf ordnungsgem\"aß oder im Falle unvorhergesehener
Ereignisse anzuhalten (falls er nicht von selbst terminiert).
Ggf. sind Aufr\"aumarbeiten zu beschreiben, um nach den Tests den urspr\"unglichen
Zustand wiederherzustellen (falls der Testlauf nicht seiteneffektfrei ist)
} \\
\hline
\textbf{Beobachtungen / Log} &  \textit{Es sind alle speziellen Methoden oder
Formate zu beschreiben, mit denen die Ergebnisse der Testl\"aufe, die
Zwischenf\"alle und sonstige wichtige Ereignisse aufgenommen werden sollen.
Beispiel: Logdatei eines Servers, Messung der Antwortzeit eines Remote
Terminals mittels Netzwerk Simulator, \ldots} \\
\hline
\textbf{Besonderheiten } &  \textit{optional; auszuf\"ullen, falls es
Besonderheiten in diesem Testfall gibt.
Testfallspezifische Besonderheiten, z.B. Ausf\"uhrungsvorschriften oder
Abweichungen von der Testumgebung (siehe 2.5)  werden hier aufgelistet.} \\
\hline
\textbf{Abh\"angigkeiten} &  \textit{optional; auszuf\"ullen, falls es
Abh\"angigkeiten in diesem Testfall gibt
Ist dieser Testfall von der Ausf\"uhrung anderer Testf\"alle abh\"angig, so werden
diese Testf\"alle hier aufgelistet und kurz beschrieben, worin die Abh\"angigkeit
besteht.} \\
\hline

 \end{longtable}

Die folgenden Tabellen beschreiben, wie der Testfall ausgef\"uhrt wurde und
welches Ergebnis er geliefert hat. Da es bei Korrektur von Softwarefehlern oder
anderen Gegebenheiten notwendig ist, einen Test mehrfach durchzuf\"uhren
(Testl\"aufe), ist jede Testdurchf\"uhrung zu dokumentieren. Daher ist diese
Tabelle f\"ur \textbf{jeden Testlauf }  zu erstellen und \textbf{ fortlaufend zu
nummerieren}. \\


\begin{longtable}{|p{7cm}|p{10cm}|}
\hline
\textbf{Testfall -- ID und Bezeichnung} & \textit{Beispiel: /T100/ Lager
anlegen} \\
\hline
\textbf{Testlauf Nr.} & \textit{Beispiel: 1} \\
\hline
\textbf{Eingaben} & \textit{Es sind alle Eingabedaten bzw. andere Aktionen
aufzuf\"uh-ren, die f\"ur die Ausf\"uhrung des Testfalls notwendig sind.
Diese k\"onnen sowohl als Wert angegeben werden (ggf. mit Toleranzen) als auch
als Name, falls es sich um konstante Tabellen oder um Dateien handelt. Außerdem
sind alle betroffenen Datenbanken, Dateien, Terminal Meldungen, etc. anzugeben.
Hinweis: Es sind nicht noch mal die Einzelschritte aus 3.1.3 zu wiederholen.
W\"ahrend jene allgemeiner sind (z.B. "`Ein-loggen \"uber das Login-Formular"')
sind hier die konkreten eingegebenen Testdaten aufzuf\"uhren (z.B. "`Loginname:
test; Passwort: xxxtest"'`). } \\
\hline
\textbf{Soll - Reaktion} & \textit{Hier ist anzugeben, welches Ergebnis bzw.
Ausgabe der Test haben soll.
Hinweis: Es sind nicht noch mal die Erfolgskriterien aus 3.1.2 zu wiederholen.
W\"ahrend jene allgemeiner sind (z.B. "`Testnachricht wird \"uber Netzwerkkanal
empfangen"') sind hier die konkreten erhaltenen Testdaten aufzuf\"uhren (z.B.
Konsole zeigt Meldung: "`Testnachricht 123 erhalten"').
} \\
\hline
\textbf{Ist -- Reaktion} & \textit{Hier ist anzugeben, welches Ergebnisdaten
bzw. Ausgaben dieser Testlauf geliefert hat.} \\
\hline
\textbf{Ergebnis} & \textit{F\"ur jeden Testlauf ist zu vermerken, ob der Test
erfolgreich durchgef\"uhrt werden konnte oder nicht. Einen missgl\"uckten Test
bitte begr\"unden, sofern der Grund des Fehlschlags bekannt oder offensichtlich
ist.} \\
\hline
\textbf{Unvorhergesehene Ereignisse w\"ahrend des Test-laufs } &
\textit{optional; nur anzugeben, falls es unvorhergesehene Ereig-nisse gab} \\
\hline
\textbf{Nacharbeiten } & \textit{Ist ein Testlauf nicht erfolgreich
durchgef\"uhrt worden, so werden hier die erforderlichen Nacharbeiten aufgef\"uhrt
(z.B. Bugfixes).} \\
\hline
 \end{longtable}
