% Kapitel 4
%-------------------------------------------------------------------------------


\chapter{Zusammenfassung}
Alle Testergebnisse liegen in einem zufriedenstellenden Bereich vor.  Die
meisten, aber inzwischen gelösten Probleme und Fehler, lagen vor allem an dem
geschriebenen Code aufgrund der anfänglichen Schwierigkeiten mit der Nutzung
der neuen Programmiersprache (Python) und des für uns unbekannten
Web-Frameworks (Django).
Doch mit der Zeit waren wir besser in der Lage, Datenbanken mithilfe von
Python/Django zu konstruieren und sie in ein Web-Interface einzubinden.  Der
eindeutige Vorteil beim Testen unseres Django-Projekts bestand darin, dass die
von uns eingebundenen Built-In Funktionen (vgl. Liste von 2.3), schon
ausreichend und professionell von den verantwortlichen Teams des Web-Frameworks
getestet wurden und somit unsererseits keiner weiteren Überprüfung bedurften,
so dass der tatsächliche Umfang des Testverlaufes wesentlich geringer war als
der ursprünglich gedachte Testplan. 
Während der Implementierungsphase hat der \BibTeX -Parser einiger Tests bedurft
und benötigte in vielen Phasen auch Anpassungen aufgrund fehlgeschlagener
Tests, diese konnten jedoch zu großen Teilen noch in der
Implementierungsphase gelöst werden konnten.
