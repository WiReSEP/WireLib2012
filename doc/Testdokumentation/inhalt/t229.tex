\subsection{Testfall -- T229 Entleihliste}
In der folgenden Tabelle wird der Test der Entleihliste beschrieben. 
\begin{longtable}{|p{5cm}|p{10cm}|}
\hline
\textbf{Testfall -- T229} &  \textnormal{ /T229/ Entleihliste} \\
\hline
\textbf{Zu testende Objekte und Methoden} &  
\textnormal{ 
\begin{itemize}
\item In Komponente \emph{views.py} die Funktion \lstinline {doc_rent()}
\item In Komponente \emph{models.py} die Funktion \lstinline {doc_status()}
\end{itemize} }\\
\hline
\textbf{Kriterien f\"ur erfolgreiche bzw. fehlgeschlagene Testf\"alle} &
\textnormal{Es liegt ein erfolgreicher Test vor, wenn eine Auflistung der tatsächlich
ausgeliehenen Dokumente, im richtigen Format und in genauer Anzahl angezeigt wird.
Es wird vorrausgesetzt, dass der angemeldete Benutzer über die Rechte besitzt, 
um Bücher auszuleihen.
Der Test ist fehlgeschlagen, wenn diese nicht auftreten. } \\
\hline
\textbf{Einzelschritte} &  
\textnormal{Die Webseite wird im Browser aufgerufen. 
Nach dem Anmelden (Seitenbutton \uline{Anmelden}) auf einem Testaccount mit den 
dazu benötigten Rechten, soll der User einige Dokumente aus dem 
Literaturverzeichnis ausleihen. Durch Navigieren der Seite 
(\uline{Profil} -> \uline{Ausleihliste}) soll der User eine Liste seiner aktuell ausgeliehenen 
Dokumente vorfinden.
Sollten unvorhergesehene Ereignisse auftreten, kann man den laufenden Testprozess
durch Interaktion mit der Konsole mit dem Django-internen Befehl Ctrl-D terminieren
und gegebenenfalls den Server für einen erneuten Test neustarten.      
} \\
\hline
\textbf{Beobachtungen / Log} &  \textnormal{Ergebnisse und Fehler des Testlaufs 
werden innerhalb vom Django-Projekt im Browser und im Termimal angezeigt.}\\ 
\hline
\end{longtable}

Die folgende Tabelle dokumentiert den erfolgreichen Testlauf der Entleihliste


\begin{longtable}{|p{5cm}|p{10cm}|}
\hline
\textbf{Testfall -- ID und Bezeichnung} & \textnormal{T229} \\
\hline
\textbf{Testlauf Nr.} & \textnormal{1} \\
\hline
\textbf{Eingaben} & \textnormal{Für das Anzeigen der Ausleihliste sind folgende 
Eingaben von Nöten: Einträge (Dokumente, User, etc) in der Datenbank, 
Erfolgreiches Anmelden eines registrierten Benutzers, Bestätigen des Links 
\uline{Literaturverzeichnis}, Auswählen eines oder mehreren Dokumentes 
(Klick auf den Titel des Dokumentes), Bestätigen des Buttons \uline{Ausleihen},   
Aufrufen des \uline{Profil}-Links und der \uline{Ausleihliste}.   
   } \\
\hline
\textbf{Soll - Reaktion} & \textnormal{Die Ausleihliste soll die gewünschten
Dokumente anzeigen und im korrekten Format darstellen (Anordnung von Titel, 
Autor, Jahr, etc). } \\
\hline
\textbf{Ist -- Reaktion} & \textnormal{Dokumente werden in Ausleihliste angezeigt.} \\
\hline
\textbf{Ergebnis} & \textnormal{Der Test ist erfolgreich abgelaufen} \\
\hline
\textbf{Nacharbeiten } & \textnormal{Kein Bugfix notwendig.} \\
\hline
\end{longtable}

