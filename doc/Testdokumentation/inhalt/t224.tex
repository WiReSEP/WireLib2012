\subsection{Testfall -- /T224/: Ausleihe vermisst melden}
\begin{longtable}{|p{5cm}|p{10cm}|}
\hline
\textbf{Testfall -- ID und Bezeichnung} &  \textnormal{/F224/: Ausleihe vermisst melden} \\
\hline
\textbf{Zu testende Objekte und Methoden} &  \textnormal{\begin{itemize}
    \item die Webseite \uline{doc\_detail.html},
    \item in Komponente \textit{Models} die Funktion \lstinline{document.missing()}, 
    \item in Komponente \textit{Models} die Funktion \lstinline{document.set_status()},
    \item in Komponente \textit{Views} die Funktion \lstinline{doc_detail()},
    \end{itemize}}
\\
\hline
\textbf{Kriterien f\"ur erfolgreiche bzw. fehlgeschlagene Testf\"alle} &
\textnormal{Erfolgreich: Dokument wird als \glqq vermisst\grqq in der DB gespeichert, 
die Webseite reflektiert das Dokument als vermisst, jeder registrierte Benutzer
bekommt eine Vermisst Meldung per E-Mail zugesendet. } \\
\hline
\textbf{Einzelschritte} &  \textnormal{Nach Überprüfung, ob man angemeldet ist, öffnet
man die \uline{doc\_detail.html} von einem Dokument. Nun wird auf den Button
\glqq Als Vermisst melden\grqq geklickt.} \\
\hline
\textbf{Beobachtungen / Log} &  \textnormal{Fehler werden über die Konsole ausgegeben, 
der Inhalt der E-Mail wird über \textbf{stdout} ausgegeben, sowie auch alle Adressaten.} \\
\hline
\textbf{Besonderheiten } &  \textnormal{Falls der vorherige Eintrag in \glqq doc\_status \grqq dem neuen bis auf
        den Timestamp gleicht, muss natürlich kein neuer Eintrag eingefügt
        werden. Die Funktionen sollten also komplett ignoriert werden.} \\
\hline


 \end{longtable}

Die folgende Tabelle zeigt den Testfall, wenn man ein Buch vermisst meldet.
\begin{longtable}{|p{5cm}|p{10cm}|}
\hline
\textbf{Testfall -- ID und Bezeichnung} & \textnormal{/T224/: Ausleihe vermisst melden} \\
\hline
\textbf{Testlauf Nr.} & \textnormal{1} \\
\hline
\textbf{Eingaben} & \textnormal{Erforderliche Eingaben für das Vermisst Melden eines Dokumentes ist der
        Datensatz eines Buches, das Recht des Vermisstmeldens beim angemeldeten 
        User und ein Klick auf den Button \uline{als vermisst melden}. Es muss zudem 
        mindestens eine E-Mail in der DB stehen.} \\
\hline
\textbf{Soll - Reaktion} & \textnormal{Der \glq return\_lend \grq -Wert des letzten Eintrages in \glqq 
        doc\_status \grqq zu diesem Dokument wird auf TRUE gesetzt, ein neuer 
        Eintrag mit dem neuen Status sollte in selbiger Tabelle erstellt werden 
        und es sollte nur noch ein Button \uline{Gefunden melden} erscheinen für 
        User mit dem entsprechenden Recht. Außerdem sollten alle registrierten Nutzer
        eine E-Mail mit der Vermisst Meldung bekommen.).
} \\
\hline
\textbf{Ist -- Reaktion} & \textnormal{Der Test gibt die gewünschten Soll-Reaktionen zurück.} \\
\hline
\textbf{Ergebnis} & \textnormal{Der Test war erfolgreich} \\
\hline
 \end{longtable}
 
Die folgende Tabelle zeigt den Testfall, wenn man ein Buch als vermisst melden
möchte, obwohl man dieses bereits getan hat (z.B. durch Seitenaktualisierung).
\begin{longtable}{|p{5cm}|p{10cm}|}
\hline
\textbf{Testfall -- ID und Bezeichnung} & 
\textnormal{/T224/: Ausleihe vermisst melden} 
\\
\hline
\textbf{Testlauf Nr.} & \textnormal{2} \\
\hline
\textbf{Eingaben} & 
\textnormal{Erforderliche Eingaben für das Vermisst Melden eines Dokumentes ist der
        Datensatz eines Buches, das Recht des Vermisstmeldens beim angemeldeten 
        User und ein Klick auf den Button \uline{als vermisst melden}. Es muss zudem 
        mindestens eine E-Mail in der DB stehen. }
\\
\hline
\textbf{Soll - Reaktion} & 
\textnormal{Die Seite wird genauso zurückgegeben, wie sie bereits ist. Es wird der
        letzte \glqq doc\_status \grqq -Eintrag nicht verändert und auch kein 
        neuer hinzugefügt.} 
\\
\hline
\textbf{Ist -- Reaktion} & 
\textnormal{Der Test gibt die gewünschten Soll-Reaktionen zurück.} 
\\
\hline
\textbf{Ergebnis} & 
\textnormal{Der Test ist erfolgreich abgelaufen.} \\
\hline
 \end{longtable}
 
 

