\subsection{Testfall -- /T230/ Bib\TeX -Export}
Jeder Benutzer kann in der Detailseite eines Dokumentes einen \BibTeX -Eintrag
zu dem aktuellen Dokument sehen und zur eigenen Verwendung kopieren.
\begin{longtable}{|p{5cm}|p{10cm}|}
\hline
\textbf{Testfall -- ID und Bezeichnung} &  T230 -- \BibTeX -Export \\
\hline
\textbf{Zu testende Objekte und Methoden} &  
\textnormal{
\begin{itemize}
  \item In Komponente \textit{views} die Funktion \lstinline{doc_detail()}
  \item In Komponente \textit{Server (App: Documents)} die Funktion
	\lstinline{Bibtex.export_doc()}
\end{itemize}
}
\\
\hline
\textbf{Kriterien f\"ur erfolgreiche bzw. fehlgeschlagene Testf\"alle} &
\textnormal{Die Funktion \lstinline{doc_detail()} liefert im \lstinline{Context} in 
der Variablen \lstinline{Bibtex.bibtex_string()} einen \BibTeX -String, der die Funktion
\lstinline{is_valid()} besteht.}\\
\hline
\textbf{Einzelschritte} &  Der Test läuft ohne Benutzerinteraktion, also
vollständig selbständig. Unvorhergesehene Exceptions brechen den Test mit einer
entsprechenden Fehlermeldung ab. Änderungen an Daten werden nicht durchgeführt.
\\
\hline
\textbf{Beobachtungen / Log} &  Der Test läuft leise ab und gibt nur im
Fehlerfall die entsprechende Fehlerzeile aus.\\
\hline
\textbf{Abh\"angigkeiten} &  
Dieser Test ist abhängig von dem Test \nameref{t200}.\\
\hline

 \end{longtable}

%Die folgenden Tabellen beschreiben, wie der Testfall ausgef\"uhrt wurde und
%welches Ergebnis er geliefert hat. Da es bei Korrektur von Softwarefehlern oder
%anderen Gegebenheiten notwendig ist, einen Test mehrfach durchzuf\"uhren
%(Testl\"aufe), ist jede Testdurchf\"uhrung zu dokumentieren. Daher ist diese
%Tabelle f\"ur \textbf{jeden Testlauf }  zu erstellen und \textbf{ fortlaufend zu
%nummerieren}. \\
Die folgende Tabelle beschreibt den Test mit dem Dokument \textit{K006003} aus
der Datenbank.


\begin{longtable}{|p{5cm}|p{10cm}|}
\hline
\textbf{Testfall -- ID und Bezeichnung} & T230 -- \BibTeX -Export \\
\hline
\textbf{Testlauf Nr.} & 1 \\
\hline
\textbf{Eingaben} & \textnormal{Der Funktion \lstinline{doc_detail()} wird eine
\lstinline{document} Instanz von \textit{K006003} übergeben, welches der Funktion
den Zugriff auf alle Attribute des Dokumentes liefert. Nachdem die
\lstinline{doc_detail()} die Funktion \lstinline{Bibtex.bibtex_string()} mit
dem Dokument aufruft generiert diese den entsprechenden String, der im
\lstinline{Context} der \lstinline{doc_detail()} zurückgegeben wird. Diese
\lstinline{Context}-Variable wird vom Test entnommen und der Methode
\lstinline{is_valid()} übergeben, die einen \BibTeX -Eintrag auf validität
prüft und einen Boolean zurück gibt. Wenn dieser \lstinline{True} ist, ist der
Test erfolgreich.}\\ 
\hline
\textbf{Soll - Reaktion} & Die Testgumgebung gibt nichts zurück und führt nur
den nachfolgenden Test durch.
\\
\hline
\textbf{Ist -- Reaktion} & Die Testumgebung gibt nichts zurück und führt den
nachfolgenden Test aus. \\
\hline
\textbf{Ergebnis} & Der Test läuft erfolgreich durch. \\
\hline
 \end{longtable}

