\subsection{Testfall -- /T226/: Ausleihfrist abgelaufen}

Die folgende Tabelle beschreibt den Test der Ausleihe verloren melden. \\
\begin{longtable}{|p{5cm}|p{10cm}|}
\hline
\textbf{Testfall -- ID und Bezeichnung} &  
\textnormal{/T226/: Ausleihfrist abgelaufen} \\
\hline
\textbf{Zu testende Objekte und Methoden} &  
\textnormal{\begin{itemize}
    \item in Komponente \textit{Views} die Funktion \lstinline{lending_expired}
    \end{itemize}}
\\
\hline
\textbf{Kriterien f\"ur erfolgreiche bzw. fehlgeschlagene Testf\"alle} &
\textnormal{Es sollte eine E-Mail an die angegebene E-Mail-Adresse vom Externen und 
        vom Bürgen gesendet, wenn die Ausleihfrist abgelaufen ist.} 
\\
\hline
\textbf{Einzelschritte} &  
\textnormal{Man starte den Scheduler, der die entsprechende Funktion öffnet, die 
        Ausleihzeiten überprüft und gegebenenfalls die E-Mails versendet. } 
\\
\hline
\textbf{Beobachtungen / Log} &  
\textnormal{Alle Fehler beziehungsweise Logs werden über die Konsole des Servers 
        ausgegeben.} 
\\
\hline

 \end{longtable}

Die folgende Tabelle beschreibt den Test, falls Fristen abgelaufen sind.
\begin{longtable}{|p{5cm}|p{10cm}|}
\hline
\textbf{Testfall -- ID und Bezeichnung} & \textnormal{/T226/: Ausleihfrist abgelaufen} \\
\hline
\textbf{Testlauf Nr.} & \textnormal{1} \\
\hline
\textbf{Eingaben} & 
\textnormal{Die Eingabedaten sind das aktuelle Datum, die Werte der Tabelle \glqq
        doc\_status\grqq, sämtliche E-Mail-Adressen aus \glqq Auth.Users\grqq\ 
        und \glqq non\_user\grqq\ und der eigentliche E-Mail-Text aus \glqq
        emails\grqq.}
\\
\hline
\textbf{Soll - Reaktion} & 
\textnormal{Es soll an alle Externe, deren Ausleihfrist abgelaufen ist, und deren 
        Bürgen jeweils eine für den entsprechenden Nutzer entsprechende E-Mail 
        versendet werden.
} \\
\hline
\textbf{Ist -- Reaktion} & 
\textnormal{Die Funktion reagiert so, wie sie soll.} \\
\hline
\textbf{Ergebnis} & 
\textnormal{Der Test war erfolgreich.} \\
\hline
 \end{longtable}

