\subsection{Testfall -- /T200/ Bib\TeX -Import}
\label{t200}
Im Pflichtenheft wurde bestimmt, dass der hier beschriebene Vorgang dem
Benutzer ermöglicht über eine Seite eine \BibTeX -Datei hochzuladen. Diese wird
dann an das System weiter gegeben und durch einen \BibTeX -Parser zerlegt und
in die Datenbank übertragen.

\begin{longtable}{|p{5cm}|p{10cm}|}
  \hline
  \textbf{Testfall -- ID und Bezeichnung} &  T200 -- Bib\TeX -Import \\
  \hline
  \textbf{Zu testende Objekte und Methoden} & 
  \textnormal{
  \begin{itemize}
	\item In Komponente \textit{views} die Funktion
	  \lstinline{import_bibtex()}
	\item In Komponente \textit{Server (App: Documents)} die Funktion
	  \lstinline{Bibtex.do_import()}, \lstinline{insert_doc()} und
	  \lstinline{is_valid()} 
  \end{itemize} }
  \\
  %\hline
  \textbf{Kriterien f\"ur erfolgreiche bzw. fehlgeschlagene Testf\"alle} &
  Alle in der Datei enthaltenen validen Dokumente sind nach Abschluss
  des Testes in der Datenbank vorhanden bzw.\ im Fehlerfall nicht vorhanden.\\
  \hline
  \textbf{Einzelschritte} &  Über das Testing Framework wird zuerst
  getestet, dass bei einer fehlenden Anmeldung ein \textbf{HTTP Status Code 404}
  zurück gegeben wird und ein angemeldeter Benutzer mit entsprechneden Rechten
  bekommt einen \textbf{HTTP Status Code 200} zurück. Danach wird die Datei über
  die Seite hochgeladen und damit getestet ob die weiteren Funktionen durch
  diesen Upload ausgelöst werden, ob diese Erfolgreich laufen und am Ende die
  gewünschten Daten in der Datenbank sind. Der Test kommt vollständig ohne
  Benutzerinteraktion aus, muss aber für den Fall von Dead-Locks vom Benutzer
  überwacht und ggf.\ abgebrochen werden.\\
  \hline
  \textbf{Beobachtungen / Log} & 
  Fehler die im Test passieren sorgen dafür, dass der Test abbricht und die
  entsprechnde Bedingung ausgegeben wird. Bei Fehlern in der Verarbeitung der
  \BibTeX -Datei werden diese in eine entsprechende Fehler-Datei geschrieben.
  Sonst läuft der Test still ab.
  \\
  \hline
  \textbf{Besonderheiten } &  Für diesen Test wird als Eingabe Satz die
  \textit{bib2000.bib} mit entsprechenden Daten verwendet und dafür nur eine
  Grundlage Datenbank mit Benutzern, Rechten und E-Mails aber ohne eingetragene
  Dokumente. Weiter wird dieser Test auf den zwei Datenbanksystemen
  \Gls{glos:mysql} und \Gls{glos:sqlite} gefahren\\
  \hline
\end{longtable}

Im folgenden wird das Testszenario mit der \textit{bib2000.bib}-Datei
beschrieben. Das Szenario ist erfolgreich, wenn auch alle Daten aus der Datei
in die Datenbank übernommen wurden. Einmal wird das Szenario mit einer
\Gls{glos:mysql} und einmal mit einer \Gls{glos:sqlite} durchgeführt.

\begin{longtable}{|p{5cm}|p{10cm}|}
  \hline
  \textbf{Testfall -- ID und Bezeichnung} & T200 -- Bib\TeX -Import \\
  \hline
  \textbf{Testlauf Nr.} & 1 (SQLite) \\
  \hline
  \textbf{Eingaben} &  Die Datei \textit{bib2000.bib} und die Userdaten des
  \textit{admin}-Benutzers mit dem Password \textit{sep2012} im Request für den
  \textit{HTTP Status Code 200}.  Der Test läuft automatisch ab und bedarf keiner
  Interaktion, nur die Error-Datei der \textit{bib2000.bib} muss nach Abschluss
  geprüft werden.\\
  \hline
  \textbf{Soll - Reaktion} & Die Datei \textit{bib2000.bib.err} ist nach
  Abschluss des Importes vorhanden aber leer und das Django Testing Framework
  meldet auf der Konsole „0 Errors“ auch als Folge der leeren Fehlerdatei.
  \\
  \hline
  \textbf{Ist -- Reaktion} & Der Test läuft entsprechend der Soll-Reaktion durch.\\
  \hline
  \textbf{Ergebnis} & Erfolgreich \\
  \hline
\end{longtable}

Im folgenden nun der Import der Datei in eine \Gls{glos:mysql}-Datenbank.
\begin{longtable}{|p{5cm}|p{10cm}|}
  \hline
  \textbf{Testfall -- ID und Bezeichnung} & T200 -- Bib\TeX -Import \\
  \hline
  \textbf{Testlauf Nr.} & 2 (MySQL) \\
  \hline
  \textbf{Eingaben} &  Die Datei \textit{bib2000.bib} und die Userdaten des
  \textit{admin}-Benutzers mit dem Password \textit{sep2012} im Request für den
  \textit{HTTP Status Code 200}.  Der Test läuft automatisch ab und bedarf keiner
  Interaktion, nur die Error-Datei der \textit{bib2000.bib} muss nach Abschluss
  geprüft werden.\\
  \hline
  \textbf{Soll - Reaktion} & Die Datei \textit{bib2000.bib.err} ist nach
  Abschluss des Importes vorhanden aber leer und das Django Testing Framework
  meldet auf der Konsole „0 Errors“ auch als Folge der leeren Fehlerdatei.
  \\
  \hline
  \textbf{Ist -- Reaktion} & Der Test wird von einer Exception abgebrochen, die
  von \Gls{glos:mysql} wegen eines „Truncate“ geworfen wird. Ein Keyword-Feld ist
  größer als die definierte Tabellenspalte.\\
  \hline
  \textbf{Ergebnis} & Nicht Erfolgreich: Die „Truncate“-Exception wird durch
  einen Fehler in der {\sffamily import\_bibtex()} ausgelöst, die den
  Keyword-Eintrag nicht korrekt ausplitten kann. Entsprechend muss dieses
  Splitten der Keywords überarbeitet werden.\\ \hline
\end{longtable}
