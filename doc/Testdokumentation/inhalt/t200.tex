\section{T200 -- Bib\TeX -Import}
Im Pflichtenheft wurde bestimmt, dass der hier beschriebene Vorgang dem
Benutzer ermöglicht über eine Seite eine \BibTeX -Datei hochzuladen. Diese wird
dann an das System weiter gegeben und durch einen \BibTeX -Parser zerlegt und
in die Datenbank übertragen.

\begin{longtable}{|p{5cm}|p{10cm}|}
\hline
\textbf{Testfall -- ID und Bezeichnung} &  \textit{T200 -- Bib\TeX -Import} \\
\hline
\textbf{Zu testende Objekte und Methoden} &  \textit{
\begin{itemize}
  \item In Komponente \emph{views} die Funktion
	\lstinline{import_bibtex()}
  \item In Komponente \emph{Server (App: Documents)} die Funktion
	\lstinline{Bibtex.do_import()}, \lstinline{insert_doc()} und
	\lstinline{is_valid()}
\end{itemize}
}
\\
\hline
\textbf{Kriterien f\"ur erfolgreiche bzw. fehlgeschlagene Testf\"alle} &
\textit{Alle in der Datei enthaltenen validen Dokumente sind nach Abschluss
des Testes in der Datenbank vorhanden bzw.\ im Fehlerfall nicht vorhanden.} \\
\hline
\textbf{Einzelschritte} &  \textit{Über das Testing Framework wird zuerst
getestet, dass bei einer fehlenden Anmeldung ein \textbf{HTTP Status Code 404}
zurück gegeben wird und ein angemeldeter Benutzer mit entsprechneden Rechten
bekommt einen \textbf{HTTP Status Code 200} zurück. Danach wird die Datei über
die Seite hochgeladen und damit getestet ob die weiteren Funktionen durch
diesen Upload ausgelöst werden, ob diese Erfolgreich laufen und am Ende die
gewünschten Daten in der Datenbank sind. Der Test kommt vollständig ohne
Benutzerinteraktion aus, muss aber für den Fall von Dead-Locks vom Benutzer
überwacht und ggf.\ abgebrochen werden.} \\
\hline
\textbf{Beobachtungen / Log} &  \textit{Es sind alle speziellen Methoden oder
Formate zu beschreiben, mit denen die Ergebnisse der Testl\"aufe, die
Zwischenf\"alle und sonstige wichtige Ereignisse aufgenommen werden sollen.
Beispiel: Logdatei eines Servers, Messung der Antwortzeit eines Remote
Terminals mittels Netzwerk Simulator, \ldots} \\
\hline
\textbf{Besonderheiten } &  \textit{optional; auszuf\"ullen, falls es
Besonderheiten in diesem Testfall gibt.
Testfallspezifische Besonderheiten, z.B. Ausf\"uhrungsvorschriften oder
Abweichungen von der Testumgebung (siehe 2.5)  werden hier aufgelistet.} \\
\hline
\textbf{Abh\"angigkeiten} &  \textit{optional; auszuf\"ullen, falls es
Abh\"angigkeiten in diesem Testfall gibt
Ist dieser Testfall von der Ausf\"uhrung anderer Testf\"alle abh\"angig, so werden
diese Testf\"alle hier aufgelistet und kurz beschrieben, worin die Abh\"angigkeit
besteht.} \\
\hline

 \end{longtable}

Die folgenden Tabellen beschreiben, wie der Testfall ausgef\"uhrt wurde und
welches Ergebnis er geliefert hat. Da es bei Korrektur von Softwarefehlern oder
anderen Gegebenheiten notwendig ist, einen Test mehrfach durchzuf\"uhren
(Testl\"aufe), ist jede Testdurchf\"uhrung zu dokumentieren. Daher ist diese
Tabelle f\"ur \textbf{jeden Testlauf }  zu erstellen und \textbf{ fortlaufend zu
nummerieren}. \\


\begin{longtable}{|p{5cm}|p{10cm}|}
\hline
\textbf{Testfall -- ID und Bezeichnung} & \textit{Beispiel: /T100/ Lager
anlegen} \\
\hline
\textbf{Testlauf Nr.} & \textit{Beispiel: 1} \\
\hline
\textbf{Eingaben} & \textit{Es sind alle Eingabedaten bzw. andere Aktionen
aufzuf\"uh-ren, die f\"ur die Ausf\"uhrung des Testfalls notwendig sind.
Diese k\"onnen sowohl als Wert angegeben werden (ggf. mit Toleranzen) als auch
als Name, falls es sich um konstante Tabellen oder um Dateien handelt. Außerdem
sind alle betroffenen Datenbanken, Dateien, Terminal Meldungen, etc. anzugeben.
Hinweis: Es sind nicht noch mal die Einzelschritte aus 3.1.3 zu wiederholen.
W\"ahrend jene allgemeiner sind (z.B. "`Ein-loggen \"uber das Login-Formular"')
sind hier die konkreten eingegebenen Testdaten aufzuf\"uhren (z.B. "`Loginname:
test; Passwort: xxxtest"'`). } \\
\hline
\textbf{Soll - Reaktion} & \textit{Hier ist anzugeben, welches Ergebnis bzw.
Ausgabe der Test haben soll.
Hinweis: Es sind nicht noch mal die Erfolgskriterien aus 3.1.2 zu wiederholen.
W\"ahrend jene allgemeiner sind (z.B. "`Testnachricht wird \"uber Netzwerkkanal
empfangen"') sind hier die konkreten erhaltenen Testdaten aufzuf\"uhren (z.B.
Konsole zeigt Meldung: "`Testnachricht 123 erhalten"').
} \\
\hline
\textbf{Ist -- Reaktion} & \textit{Hier ist anzugeben, welches Ergebnisdaten
bzw. Ausgaben dieser Testlauf geliefert hat.} \\
\hline
\textbf{Ergebnis} & \textit{F\"ur jeden Testlauf ist zu vermerken, ob der Test
erfolgreich durchgef\"uhrt werden konnte oder nicht. Einen missgl\"uckten Test
bitte begr\"unden, sofern der Grund des Fehlschlags bekannt oder offensichtlich
ist.} \\
\hline
\textbf{Unvorhergesehene Ereignisse w\"ahrend des Test-laufs } &
\textit{optional; nur anzugeben, falls es unvorhergesehene Ereig-nisse gab} \\
\hline
\textbf{Nacharbeiten } & \textit{Ist ein Testlauf nicht erfolgreich
durchgef\"uhrt worden, so werden hier die erforderlichen Nacharbeiten aufgef\"uhrt
(z.B. Bugfixes).} \\
\hline
 \end{longtable}

