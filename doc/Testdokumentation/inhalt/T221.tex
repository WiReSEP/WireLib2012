\subsection{Testfall -- T221 Ausleihe an Externe}
Die folgende Tabelle beschreibt den Testfall: T221 Ausleihe an Externe. \\
\begin{longtable}{|p{5cm}|p{10cm}|}
\hline
\textbf{Testfall -- ID und Bezeichnung} &  \textit{/T221/ Ausleihe an Externe} \\
\hline
\textbf{Zu testende Objekte und Methoden} & \textnormal{
\begin{itemize}
  \item In Komponente \emph{Models} die Datei \lstinline{doc\_assign.html}
  \item In Komponente \emph{Models} die Datei \lstinline{views.py}
\end{itemize}
} \\
\hline
\textbf{Kriterien f\"ur erfolgreiche bzw. fehlgeschlagene Testf\"alle} &
\textit{Der Test ist erfolgreich, wenn ein neuer Eintrag im Table non\_user
erstellt wird und das Dokument diesem zugewiesen wird, oder einem bereits
existierendem dieses zugewiesen wird.} \\
\hline
\textbf{Einzelschritte} &  \textit{
Über das Userinterface wird auf den Testserver zugegriffen.
Es muss ein Dokument auf den derzeit angemeldeten User ausgeliehen sein.
Über die Ausgeliehenen Dokumente des Users kann man die Dokumentinformationen
einsehen wo der Button \uline{Übertragen} zu der benötigten Form führt.
Dort kann ein Externer ausgewählt werden, oder durch Eingabe der benötigten
Informationen neu hinzugefügt werden.} \\
\hline
\textbf{Beobachtungen / Log} &  \textit{
Über den Browser wird die Veränderung der Website beobachtet.
Über das Terminal werden Ladezeiten sowie Schreib-/Lesevorgänge verfolgt.
}\\
\hline
\end{longtable}

In der folgenden Tabelle werden die Testergebnisse zum Testfall: T221 Ausleihe an
Externe beschrieben.

\begin{longtable}{|p{5cm}|p{10cm}|}
\hline
\textbf{Testfall -- ID und Bezeichnung} & \textit{/T221/ Ausleihe an Externe} \\
\hline
\textbf{Testlauf Nr.} & \textit{1} \\
\hline
\textbf{Eingaben} & \textit{
Es wird ein Externer User ausgewählt oder Daten über die Textfelder eingegeben
um einen neuen Externen Nutzer zu erstellen.
} \\
\hline
\textbf{Soll - Reaktion} & \textit{
Der User wird auf die Seite mit den Dokumentinformationen zurückgeleitet,
welches nun an den Externen entliehen ist.
Im Table non\_user wurde der Externe erstellt und/oder ihm das Dokument
zugewiesen.
} \\
\hline
\textbf{Ist -- Reaktion} & \textit{
Seite wird nur neu geladen und nicht auf die Dokumentinformationen geleitet.
Es wurden keine neuen Einträge erstellt oder aktualisiert.
} \\
\hline
\textbf{Ergebnis} & \textit{
Test war nicht erfolgreich aufgrund fehlender Einträge und fehlender
weiterleitung.
} \\
\hline
\textbf{Unvorhergesehene Ereignisse w\"ahrend des Test-laufs } &
\textit{Kein Zurückleiten auf die Dokumentseite} \\
\hline
\textbf{Nacharbeiten } & \textit{Ändern des Links wenn auf den Button \uline{An
Externen übertragen} geklickt wurde} \\
\hline
 \end{longtable}

