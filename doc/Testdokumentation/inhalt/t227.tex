\subsection{Testfall -- /T227/ Ausleihhistorie}
In der folgenden Tabelle wird der Test der in \lstinline{doc_detail}
eingebundenen Funktion \lstinline{__filter_history()}
\begin{longtable}{|p{5cm}|p{10cm}|}
\hline
\textbf{Testfall -- ID und Bezeichnung} &  \textit{/T227/ Ausleihhistorie} \\
\hline
\textbf{Zu testende Objekte und Methoden} & 
\textit{
\begin{itemize}
  \item In Komponente \emph{views} die Funktion
	\lstinline{__filter_history()}
  \item In Komponente \emph{views} die Funktion \lstinline{doc_detail()}
  \item In Komponente \emph{Template} die Datei \emph{doc\_detail.html}
\end{itemize}}
\\
\hline
\textbf{Kriterien f\"ur erfolgreiche bzw. fehlgeschlagene Testf\"alle} &
\textit{Wenn in dem \emph{View} doc\_detail die vorhandene Ausleihhistorie angezeigt wird,
so ist der Test geglückt. Andernfalls ist der Test fehlgeschlagen. } \\
\hline
\textbf{Einzelschritte} &  \textit{Zuerst wird die Webseite in einem Browser
aufgerufen. Falls der eingeloggte Benutzer nicht über ausreichend Rechte
besitzt, so werden ausschließlich die Attribute des Dokumentes angezeigt.
Andernfalls wird die Funktion \lstinline{__filter_history()} aufgerufen und die
Historie auf der Webseite ausgegeben. } \\
\hline
\textbf{Beobachtungen / Log} &  \textit{Der Server loggt in der Konsole, dass
doc\_detail angefodert wurde. Informationen über \lstinline{__filter_history()} werden nicht
ausgegeben. } \\
\hline

 \end{longtable}

Im Folgenden werden die beiden möglichen erfolgreichen Testfälle für die Ausleihhistorie beschrieben. \\


\begin{longtable}{|p{5cm}|p{10cm}|}
\hline
\textbf{Testfall -- ID und Bezeichnung} & \textit{/T227/ Ausleihhistorie} \\
\hline
\textbf{Testlauf Nr.} & \textit{1} \\
\hline
\textbf{Eingaben} & \textit{Eingaben sind vorhandene Einträge in der Datenbank
bezüglich des angeforderten Dokumentes. Weiterhin hat der eingeloggte Nutzer
ausreichend Rechte, um die Historie des Dokumentes einzusehen. } \\
\hline
\textbf{Soll - Reaktion} & \textit{Der Benutzer bekommt die Ausleihhistorie des
Dokumentes angezeigt. } \\
\hline
\textbf{Ist -- Reaktion} & \textit{Wie Soll - Reaktion} \\
\hline
\textbf{Ergebnis} & \textit{Der Test verlief erfolgreich. } \\
\hline
 \end{longtable}

\begin{longtable}{|p{5cm}|p{10cm}|}
\hline
\textbf{Testfall -- ID und Bezeichnung} & \textit{/T227/ Ausleihhistorie} \\
\hline
\textbf{Testlauf Nr.} & \textit{2} \\
\hline
\textbf{Eingaben} & \textit{Eingaben sind vorhandene Einträge in der Datenbank
bezüglich des angeforderten Dokumentes. Weiterhin hat der eingeloggte Nutzer
nicht ausreichend Rechte, um die Historie des Dokumentes einzusehen. } \\
\hline
\textbf{Soll - Reaktion} & \textit{Der Benutzer bekommt die Ausleihhistorie des
Dokumentes nicht angezeigt. } \\
\hline
\textbf{Ist -- Reaktion} & \textit{Wie Soll - Reaktion} \\
\hline
\textbf{Ergebnis} & \textit{Der Test verlief erfolgreich. } \\
\hline
 \end{longtable}
