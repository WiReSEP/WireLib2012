\subsection{Testfall -- /T225/: Ausleihe verloren melden}

Die folgende Tabelle beschreibt den Test der Ausleihe verloren melden. \\
\begin{longtable}{|p{5cm}|p{10cm}|}
\hline
\textbf{Testfall -- ID und Bezeichnung} &  
\textnormal{/F225/: Ausleihe verloren melden} 
\\
\hline
\textbf{Zu testende Objekte und Methoden} &  
\textnormal{\begin{itemize}
    \item die Webseite \uline{doc\_detail.html},
    \item in Komponente \textit{Models} die Funktion \lstinline{document.lend()}, 
    \item in Komponente \textit{Models} die Funktion \lstinline{document.set_status()},
    \item in Komponente \textit{Views} die Funktion \lstinline{doc_detail()},
    \end{itemize}}
\\
\hline
\textbf{Kriterien f\"ur erfolgreiche bzw. fehlgeschlagene Testf\"alle} &
\textnormal{Zum Einen muss der letzte Eintrag in der Tabelle \glqq doc\_status \grqq\
        von diesem Dokument aktualisiert werden.
        Außerdem sollte ein neuer Eintrag in dieser Tabelle existieren.
        Auch sollte sich die Webseite aktualisieren.}
\\
\hline
\textbf{Einzelschritte} &  
\textnormal{Zuerst sollte gewährleistet sein, dass man angemeldet ist. Danach muss 
        man die \uline{doc\_detail.html} von einem Dokument öffnen, welches man 
        ausgeliehen hat. Schlussendlich wird nur noch auf den Button 
        \uline{Verloren melden} geklickt, der nur angezeigt wird, wenn man das 
        entsprechende Recht besitzt.}
\\
\hline
\textbf{Beobachtungen / Log} &  
\textnormal{Alle Fehler beziehungsweise Logs werden über die Konsole des Servers 
        ausgegeben.} 
\\
\hline
\textbf{Besonderheiten } &  
\textnormal{Falls der vorherige Eintrag in \glqq doc\_status \grqq dem neuen bis auf
        den Timestamp gleicht, muss natürlich kein neuer Eintrag eingefügt
        werden. Die Funktionen sollten also komplett ignoriert werden.} 
\\
\hline

 \end{longtable}

Die folgende Tabelle zeigt den Testfall, wenn man ein Buch verloren meldet.
\begin{longtable}{|p{5cm}|p{10cm}|}
\hline
\textbf{Testfall -- ID und Bezeichnung} & 
\textnormal{/T225/: Ausleihe verloren melden} 
\\
\hline
\textbf{Testlauf Nr.} & \textnormal{1} \\
\hline
\textbf{Eingaben} & 
\textnormal{Erforderliche Eingaben für das Verlorenmelden eines Dokumentes ist der
        Datensatz eines Buches, das Recht des Verlorenmeldens beim angemeldeten 
        User und ein Klick auf den Button \uline{Verloren melden}.}
\\
\hline
\textbf{Soll - Reaktion} & 
\textnormal{Der \glq return\_lend \grq -Wert des letzten Eintrages in \glqq 
        doc\_status \grqq zu diesem Dokument wird auf TRUE gesetzt, ein neuer 
        Eintrag mit dem neuen Status sollte in selbiger Tabelle erstellt werden 
        und es sollte nur noch ein Button \uline{Gefunden melden} erscheinen für 
        User mit dem entsprechenden Recht.}
\\
\hline
\textbf{Ist -- Reaktion} & 
\textnormal{Der Test gibt die gewünschten Soll-Reaktionen zurück.} 
\\
\hline
\textbf{Ergebnis} & 
\textnormal{Der Test ist erfolgreich abgelaufen.} \\
\hline
 \end{longtable}
 
Die folgende Tabelle zeigt den Testfall, wenn man ein Buch verloren melden
möchte, obwohl man dieses bereits hat (z.B. durch Seitenaktualisierung). 
\begin{longtable}{|p{5cm}|p{10cm}|}
\hline
\textbf{Testfall -- ID und Bezeichnung} & 
\textnormal{/T225/: Ausleihe verloren melden} 
\\
\hline
\textbf{Testlauf Nr.} & \textnormal{2} \\
\hline
\textbf{Eingaben} & 
\textnormal{Erforderliche Eingaben für das Verlorenmelden eines Dokumentes ist der
        Datensatz eines Buches, das Recht des Verlorenmeldens beim angemeldeten 
        User und ein Klick auf den Button \uline{Verloren melden}.}
\\
\hline
\textbf{Soll - Reaktion} & 
\textnormal{Die Seite wird genauso zurückgegeben, wie sie bereits ist. Es wird der
        letzte \glqq doc\_status \grqq -Eintrag nicht verändert und auch kein 
        neuer hinzugefügt.} 
\\
\hline
\textbf{Ist -- Reaktion} & 
\textnormal{Der Test gibt die gewünschten Soll-Reaktionen zurück.} 
\\
\hline
\textbf{Ergebnis} & 
\textnormal{Der Test ist erfolgreich abgelaufen.} \\
\hline
 \end{longtable}


