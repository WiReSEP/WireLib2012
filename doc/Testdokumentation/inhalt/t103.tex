\subsection{Testfall -- ID und Bezeichnung}
%Jeder Testfall erh\"alt eine eindeutige Identifikation mit Kurzbezeichnung.\\
%Beispiel: /T100/ Lager anlegen\\
%Die folgende Tabelle beschreibt den Testfall. \\
In der folgenden Tabelle wird der Test zur Funktion E-Mail Änderung protokolliert.
\begin{longtable}{|p{5cm}|p{10cm}|}
\hline
\textbf{Testfall -- ID und Bezeichnung} &  \textnormal{/F103/ Mailtexte ändern} \\
\hline
\textbf{Zu testende Objekte und Methoden} &  \textnormal{
\begin{itemize}
\item In Komponente \textit{Admin} der Bereich E-Mail 
\end{itemize} }
\\
\hline
\textbf{Kriterien f\"ur erfolgreiche bzw. fehlgeschlagene Testf\"alle} &
\textnormal{Erfolgreich: E-Mail Inhalt steht, wie gewünscht, in der Datenbank ;
        Fehlgeschlagen: E-Mail Inhalt steht nicht, oder nicht wie gewünscht in der DB   } \\
\hline
\textbf{Einzelschritte} &  \textnormal{Zunächst wird über das Django-Testing-Framework getestet,
ob der Benutzer angemeldet ist (angemeldet wirft einen Http-Status-Code 200 zurück, sonst 
einen Http-Status-Code 400). Falls dieser angemeldete Benutzer genügend Rechte besitzt,
wiederum zu testen über eine spezielle Django-Testing-Authentification Methode, wird der interne View
für das E-Mail ändern angesteuert (Http-Code=200 erfolgreich, Http-Code=404 gescheitert.).
Falls dies alles erfolgreich sein sollte, kann man nun E-Mail Texte verändern.
} \\
\hline
\textbf{Beobachtungen / Log} &  \textnormal{Ergebnisse werden haptsächlich über 
die Kommandozeile per \textbf{stdout}(gibt Text der E-Mail aus) und \textbf{Django-Testing-
Framework} kontrolliert  } \\
\hline


 \end{longtable}

%Die folgenden Tabellen beschreiben, wie der Testfall ausgef\"uhrt wurde und
%welches Ergebnis er geliefert hat. Da es bei Korrektur von Softwarefehlern oder
%anderen Gegebenheiten notwendig ist, einen Test mehrfach durchzuf\"uhren
%(Testl\"aufe), ist jede Testdurchf\"uhrung zu dokumentieren. Daher ist diese
%Tabelle f\"ur \textbf{jeden Testlauf }  zu erstellen und \textbf{ fortlaufend zu
%nummerieren}. \\

Folgend werden die benötigten Testdurchläufe näher beschrieben
\begin{longtable}{|p{5cm}|p{10cm}|}
\hline
\textbf{Testfall -- ID und Bezeichnung} & \textnormal{/F103/ Mailtexte ändern} \\
\hline
\textbf{Testlauf Nr.} & \textnormal{1} \\
\hline
\textbf{Eingaben} & \textnormal{Ein eingeloggter Nutzer mit entsprechenden Rechten, E-Mail 
Einträge in der Datenbank, Klick auf E-Mail ändern in dem Admin Backend } \\
\hline
\textbf{Soll - Reaktion} & \textnormal{Ausgabe sollte eingegebenen E-Mail
Text enthalten: "`Sehr geehrte Damen und Herren, "').
} \\
\hline
\textbf{Ist -- Reaktion} & \textnormal{stdout lieferte: "`Sehr geehrte Damen und Herren, "'} \\
\hline
\textbf{Ergebnis} & \textnormal{Testlauf Nr.1 gab den erwarteten E-Mail Text zurück, somit erfolgreich.} \\
\hline
 \end{longtable}
 
\begin{longtable}{|p{5cm}|p{10cm}|}
\hline
\textbf{Testfall -- ID und Bezeichnung} & \textnormal{/F103/ Mailtexte ändern} \\
\hline
\textbf{Testlauf Nr.} & \textnormal{2} \\
\hline
\textbf{Eingaben} & \textnormal{Ein eingeloggter Nutzer mit entsprechenden Rechten, E-Mail 
Einträge in der Datenbank, Klick auf E-Mail ändern in dem Admin Backend } \\
\hline
\textbf{Soll - Reaktion} & \textnormal{Ausgabe sollte eingegebenen E-Mail
Text enthalten: "`Sehr geehrte Damen und Herren,///!123 
                    ..--\_\_??==))88\&/\& "').
} \\
\hline
\textbf{Ist -- Reaktion} & \textnormal{stdout lieferte: "`Sehr geehrte Damen und Herren,///!123 
                    ..--\_\_??==))88\&/\& "'} \\
\hline
\textbf{Ergebnis} & \textnormal{Testlauf Nr.2 gab den erwarteten E-Mail Text zurück, somit erfolgreich.} \\
\hline
 \end{longtable} 
 

