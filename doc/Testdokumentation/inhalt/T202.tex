\subsection{Testfall -- /T202/ Editieren von Dokumenten}
Die folgende Tabelle beschreibt den Testfall: /T202/ Editieren von Dokumenten \\
\begin{longtable}{|p{5cm}|p{10cm}|}
\hline
\textbf{Testfall -- ID und Bezeichnung} &  \textit{/T202/ Editieren von
Dokumenten} \\
\hline
\textbf{Zu testende Objekte und Methoden} &  \textnormal{
\begin{itemize}
  \item In Komponente \emph{Models} die Datei \lstinline{doc\_add.html}
  \item In Komponente \emph{Models} die Datei \lstinline{forms.py}
\end{itemize}
} \\
\hline
\textbf{Kriterien f\"ur erfolgreiche bzw. fehlgeschlagene Testf\"alle} &
\textit{Test erfolgreich: Dokument wurde erstellt und im Table documents
erzeugt, oder eine Meldung wirft das die Importdatei fehlerhaft ist.
Test fehlgeschlagen: Dokument ist nicht im Table documents vorhanden.} \\
\hline
\textbf{Einzelschritte} &  \textit{
Über das Userinterface können Bibliothekare und Administratoren Dokumente,
über den Link \uline{Import}, der Datenbank hinzufügen.
Dabei gelangen sie auf eine Form mit der Möglichkeit des Imports einer
\BibTeX Datei oder der direkten Eingabe der Informationen.
} \\
\hline
\textbf{Beobachtungen / Log} &  \textit{
Es werden über das Terminal Ladezeiten sowie Schreibzugriffe beobachtet.} \\
\hline

 \end{longtable}

Die folgenden Tabellen beschreiben, wie der Testfall /T202/ Editieren von
Dokumenten ausgef\"uhrt wurde und welches Ergebnis er geliefert hat.

\begin{longtable}{|p{5cm}|p{10cm}|}
\hline
\textbf{Testfall -- ID und Bezeichnung} & \textit{/T202/ Editieren von
Dokumenten} \\
\hline
\textbf{Testlauf Nr.} & \textit{1} \\
\hline
\textbf{Eingaben} & \textit{
Es wird eine \BibTex Datei eingegeben} \\
\hline
\textbf{Soll - Reaktion} & \textit{
Datei wird eingelesen und ausgelesen. Die Informationen werden in der Datenbank
gespeichert.
} \\
\hline
\textbf{Ist -- Reaktion} & \textit{Datei wurde eingelesen und Informationen
gespeichert.} \\
\hline
\textbf{Ergebnis} & \textit{Test war erfolgreich.} \\
\end{longtable}

\begin{longtable}{|p{5cm}|p{10cm}|}
\hline
\textbf{Testfall -- ID und Bezeichnung} & \textit{/T202/ Editieren von
Dokumenten} \\
\hline
\textbf{Testlauf Nr.} & \textit{2} \\
\hline
\textbf{Eingaben} & \textit{
Eingabe der Informationen über die Felder.} \\
\hline
\textbf{Soll - Reaktion} & \textit{
Die Felder werden ausgelesen und die Informationen in der Datenbank gespeichert.
Bei notwendigen Feldern wird eine Meldung angezeigt, diese Felder noch zu
füllen.
} \\
\hline
\textbf{Ist -- Reaktion} & \textit{Daten wurden eingelesen und Informationen 
gespeichert.} \\
\hline
\textbf{Ergebnis} & \textit{Test war erfolgreich.} \\
\end{longtable}

