% Kapitel 2 mit den entsprechenden Unterkapiteln
% Die Unterkapitel k\"onnen auch in separaten Dateien stehen,
% die dann mit dem \include-Befehl eingebunden werden.
%----------------------------------------------------------------------------

\chapter{Testplan}
%Der Testplan ist das zentrale Dokument der Qualit\"atssicherung und wird daher
%fr\"uhzeitig erstellt. Hier wird Umfang und Vorgehensweise der Qualit\"atssicherung
%beschrieben. Außerdem werden Testgegenst\"ande und deren zu testenden
%Eigenschaften bzw. Funktionen identifiziert. Ferner werden die
%durchzuf\"uhrenden Maßnahmen und die daf\"ur verantwortlichen Personen definiert.
%Falls erforderlich sollte hier auch auf allgemeine Risiken eingegangen werden.

\section{Zu testende Komponenten}

%Hier sind s\"amtliche zu testenden Objekte einschließlich der Versionsnummer aufzuf\"uhren.
%Ebenso ist anzugeben, auf welchem Medium die Software vorliegt, ob dies einen Einfluss auf Hardwareanforderungen hat
%und ob die Software vor Testbeginn in irgendeiner Weise transformiert werden muss. Außerdem wird auf zum
%Objekt geh\"orende Dokumentation der Komponente (Lasten-, Pflichtenheft, Grob-
%bzw. Feinentwurf) referenziert.

Es müssen die Komponenten Views, Templates und Models getestet werden.
Dabei werden die Komponenten Templates und Models durch die Funktionen der
Komponenten Views und Admin gestestet. Da es sich um eine Webanwendung handelt
ist zu beachten, dass unterschiedliche Clienten (Browser) benutzt werden.

Die Software kann online in der Repository, lokal auf der Festplatte oder einem
anderem Speichermedium vorliegen. Es ist allerdings sicherzustellen, dass die
Software korrekt auf dem Server konfiguriert wird (s. Feinentwurf Kapitel 5)
sowie eine funktionierende (Internet-) Verbindung zwischen Server und Client
besteht.

\section{Zu testende Funktionen}
%Dieser Punkt beinhaltet alle Eigenschaften bzw. Funktionen und deren
%Kombinationen, die zu testen sind.

%S\"amtliche Funktionalit\"aten, die getestet werden sollen, werden hier aufgef\"uhrt.
%Dabei sind auf die vorangegangenen Dokumentationen zu referenzieren
%(Pflichtenheft, Grob- und Fein-entwurf) und die dortigen Funktions-IDs zu
%verwenden!\\ Beispiel: /F100/ : Benutzer Login

\begin{itemize}
\item /F102/: Anbindung an die Universiätsbibliothek
\item /F103/: Mailtexte ändern
\item /F200/: \BibTex - Import %Theo
\item /F201/: Webinterface - Import
\item /F202/: Editieren von Dokumenten
\item /F212/: Erweiterte Suche %Markus
\item /F213/: Suche mit regulären Ausdrücken %Markus
\item /F214/: Sortierung der Suche %Eric
\item /F221/: Ausleihe an Externe %Marco
\item /F222/: Ausleihe übertragen %Marco
\item /F223/: Ausleihe zurückgeben
\item /F224/: Ausleihe vermisst melden
\item /F225/: Ausleihe verloren melden
\item /F226/: Ausleihfrist abgelaufen
\item /F227/: Ausleihhistory
\item /F228/: Derzeitiger Leihender %Markus
\item /F229/: Entleihliste
\item /F230/: \BibTex - Export %Theo
\item /F231/: Universitätsbibliotheks - Export
\end{itemize}

\section{Nicht zu testende Funktionen}

%(optional; auszuf\"ullen, falls es Funktionen gibt, die nicht getestet werden
%sollen)\\

%Hier werden alle Eigenschaften bzw. Funktionen und Funktionskombinationen
%aufgelistet, die nicht getestet werden.
%\textbf{ Es sollte begr\"undet werden, warum diese nicht getestet werden.} Es
%versteht sich von selber, dass alle Muss-Funktionalit\"aten des Pflichtenheftes
%(Abschnitt 1.1) getestet werden m\"ussen.

Die folgenden Funktionen werden nicht getestet, da sie aus
Django-Built-In-Funktionen bestehen. Diese Funktionen wurden bereits bei der
Entwicklung von Django ausführlich und professionell getestet.

\begin{itemize}
\item /F100/: Anmeldung
\item /F203/: Löschen von Dokumenten
\item /F210/: Generelle Suche
\item /F211/: Suche mit erweiterten Ausdrücken
\item /F220/: Ausleihe
\item /F300/: Benutzerverwaltung
\item /F301/: Rechtezuweisung für Rollen
\item /F302/: Benutzer Rolle(n) zuweisen
\end{itemize}

\section{Vorgehen}
%Die allgemeinen Vorgehensweisen f\"ur die einzelnen zu testenden Funktionen und
%Funktionskombinationen werden hier beschrieben. Die Beschreibung sollte
%detailliert genug sein, um die Hauptaktivit\"aten und deren Zeitbedarf absch\"atzen
%zu k\"onnen.

%Es ist zu beachten, dass f\"ur alle wichtigen Funktionalit\"aten das Verfahren
%angegeben wird. Dies gew\"ahrleistet, dass diese Funktionalit\"aten ad\"aquat
%getestet werden.

%Es ist zu dokumentieren, welche Aktivit\"aten, Techniken und Werkzeuge ben\"otigt
%werden, damit die Funktionalit\"aten getestet werden k\"onnen.

%Beispiel f\"ur Vorgehen (unvollst\"andige Liste):\\
%a) Komponenten- und Integrationstests\\
%Klassen werden mit JUnit-Testf\"allen gepr\"uft. Vor Beginn der Implementierung
%werden bereits Blackbox-Testf\"alle erstellt, die dann begleitend zur
%Implementierung genutzt werden ("`Test first"'). Nach Abschluss der
%Implementierung einer Komponente wird diese dann durch Whitebox-Tests
%gepr\"uft.\\
%Der Integrationstest der Klassen und Komponenten erfolgt nach dem
%Bottom-Up-Prinzip. Anfangs muss die Integration der Datenbankanbindung und den
%entsprechenden Data-Access-Objects (DAO) gepr\"uft werden, da das Mapping der
%Datenbank auf Objekte die unterste Schicht des Projektes bildet. Dieser
%Testabschnitt wird durch die Schnittstellentests abgedeckt.
%Die Komponenten werden damit unter Ber\"ucksichtigung ihrer Abh\"angigkeiten
%konkret in folgender Reihenfolge integriert: \ldots\\
%(Hier kommt das konkrete Vorgehen bei der Integration: Welche Klassen werden
%zusammen getestet, welche kommen dann dazu etc. Das kann man z.B. auch sch\"on in
%Form eines Baumes aufzeigen.)
%b) Funktionstests\\
%Die Anwendungsf\"alle aus der Anforderungsspezifikation werden \"uber das
%Web-Interface gepr\"uft. Mindestanforderung hierf\"ur ist es, jeden Fall einmal auf
%seine korrekte Funktionalit\"at zu testen.\\
%c) \ldots

Aufgrund der Struktur des Projektes werden sämtliche Tests als Funktionstests
über das Webinterface getestet. Dabei werden Firefox, Opera, Chrome, IE 9 als
Browser benutzt um zu gewährleisten, dass das Webinterface auf möglichst vielen
Clientsystemen funktioniert. Als weiteres Werkzeug verwenden wir das Django
testing Environment

\section{Testumgebung}
%Die genutzte Testumgebung(en) bitte hier angeben und kurz beschreiben.\\
%Beispiel: JUnit Testsuite, lokal installierter Web Application Server, \ldots

Als Testumgebung auf dem Server wird der lokal von Django bereitgestellte Testserver
verwendet. Als Clienten werden verschiedene Windows- und Linuxsysteme
verwendet.
