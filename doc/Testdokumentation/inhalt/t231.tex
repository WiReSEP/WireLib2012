\subsection{Testfall -- T231 Anbindung an die Universitätsbibliothek}
Jeder Testfall erh\"alt eine eindeutige Identifikation mit Kurzbezeichnung.\\
Beispiel: /T100/ Lager anlegen\\
Die folgende Tabelle beschreibt den Testfall. \\
\begin{longtable}{|p{5cm}|p{10cm}|}
\hline
\textbf{Testfall -- T231 } &  \textit{/T102/ Anbindung an die
Universitätsbibiliothek } \\
\hline
\textbf{Zu testende Objekte und Funktionen} & 
\textit{
\begin{itemize}
  \item In Komponente \emph{Server (App: Documents)} die Funktion
	\lstinline{extras_allegro.export_allegro()}
\end{itemize} } \\
\hline
\textbf{Kriterien f\"ur erfolgreiche bzw. fehlgeschlagene Testf\"alle} &
\textit{Der Test ist erfolgreich, wenn eine Datei im ADT-Format ausgegeben
wird, ausser es existieren keine Dokumente in der Datenbank, die exportiert
werden müssten. Weiterhin müssen alle Einträge in der Datei auf Korrektheit
überprüft werden. Vorbedingung ist, dass der eingeloggte Benutzer über
ausreichend Rechte verfügt, um den Export durchzuführen. Falls zu exportierende
Dokumente in der Datenbank vorhanden sind, aber keine neue .ADT-Datei angelegt
wurde, oder der eingeloggte Benutzer verfügt nicht über ausreichend Rechte, so
ist der Test fehlgeschlagen. } \\
\hline
\textbf{Einzelschritte} & 
\textit{Zuerst wird die Webseite in einem Browser aufgerufen. Falls der
eingeloggte Benutzer nicht ausreichend Rechte besitzt, so wird der Export
abgebrochen. Durch den Klick auf \uline{Allegro-Export starten} wird die Datei angelegt und
zum Download angeboten. Es wird eine Meldung ausgegeben, falls es aktuell keine
Dokumente zu exportieren gibt. Anschließend wird auf Serverseite die Datei auf
Korrektheit überprüft. } \\
\hline
\textbf{Beobachtungen / Log} &  \textit{Alle Fehler beziehungsweise Logs werden
über die Konsole des Servers ausgegeben. } \\
\hline

 \end{longtable}

Die folgenden Tabellen beschreiben, wie der Testfall ausgef\"uhrt wurde und
welches Ergebnis er geliefert hat. Da es bei Korrektur von Softwarefehlern oder
anderen Gegebenheiten notwendig ist, einen Test mehrfach durchzuf\"uhren
(Testl\"aufe), ist jede Testdurchf\"uhrung zu dokumentieren. Daher ist diese
Tabelle f\"ur \textbf{jeden Testlauf }  zu erstellen und \textbf{ fortlaufend zu
nummerieren}. \\


\begin{longtable}{|p{5cm}|p{10cm}|}
\hline
\textbf{Testfall -- T231 Anbindung an die Universitätsbibliothek} &
\textit{/T231/ Anbindung an die Universitätsbibliothek} \\
\hline 
\textbf{Testlauf Nr.} & \textit{1} \\
\hline
\textbf{Eingaben} & \textit{Erforderliche Eingaben für den Export sind Einträge in der
Datenbank, ein eingeloggter Benutzer und ein Klick auf \uline{Allegro-Export
starten}. Weitere Eingaben sind nicht notwendig. } \\
\hline
\textbf{Soll - Reaktion} & \textit{Die neu erstellte .ADT-Datei wird erstellt
und über den Browser zum Download angeboten, oder es wird die Meldung gegeben,
dass keine Dokumente exportiert werden müssen. } \\
\hline
\textbf{Ist -- Reaktion} & \textit{Die .ADT-Datei wird erstellt und zum
Download angeboten. } \\
\hline
\textbf{Ergebnis} & \textit{Der Test ist erfolgreich abgelaufen. } \\
\hline
\textbf{Nacharbeiten } & \textit{Keine Bugfixes notwendig} \\
\hline
\end{longtable}

\begin{longtable}{|p{5cm}|p{10cm}|}
\hline
\textbf{Testfall -- T231 Anbindung an die Universitätsbibliothek} &
\textit{/T231/ Anbindung an die Universitätsbibliothek} \\
\hline 
\textbf{Testlauf Nr.} & \textit{2} \\
\hline
\textbf{Eingaben} & \textit{Erforderliche Eingaben für den Export sind Einträge in der
Datenbank, ein eingeloggter Benutzer und ein Klick auf \uline{Allegro-Export
starten}. Weitere Eingaben sind nicht notwendig. } \\
\hline
\textbf{Soll - Reaktion} & \textit{Die neu erstellte .ADT-Datei wird erstellt
und über den Browser zum Download angeboten, oder es wird die Meldung gegeben,
dass keine Dokumente exportiert werden müssen. } \\
\hline
\textbf{Ist -- Reaktion} & \textit{Es wird die Meldung gegeben, dass keine
Dokumente exportiert werden müssen. } \\
\hline
\textbf{Ergebnis} & \textit{Der Test ist erfolgreich abgelaufen. } \\
\hline
\textbf{Nacharbeiten } & \textit{Keine Bugfixes notwendig} \\
\hline
\end{longtable}
