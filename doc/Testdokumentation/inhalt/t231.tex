\subsection{Testfall -- T231 Universitätsbibliotheks - Export}
In der folgenden Tabelle wird der Test des Exports in das
Universitätsbibliotheks-Format ADT beschrieben.
\begin{longtable}{|p{5cm}|p{10cm}|}
\hline
\textbf{Testfall -- ID und Bezeichnung} &  \textit{/T102/ Universitätsbibliotheks - Export} \\
\hline
\textbf{Zu testende Objekte und Funktionen} & 
\textit{
\begin{itemize}
  \item In Komponente \emph{Server (App: Documents)} die Funktion
	\lstinline{extras_allegro.export_allegro()}
\end{itemize} } \\
\hline
\textbf{Kriterien f\"ur erfolgreiche bzw. fehlgeschlagene Testf\"alle} &
\textit{Der Test ist erfolgreich, wenn eine Datei im ADT-Format ausgegeben
wird, ausser es existieren keine Dokumente in der Datenbank, die exportiert
werden müssten. Weiterhin müssen alle Einträge in der Datei auf Korrektheit
überprüft werden. Vorbedingung ist, dass der eingeloggte Benutzer über
ausreichend Rechte verfügt, um den Export durchzuführen. Falls zu exportierende
Dokumente in der Datenbank vorhanden sind, aber keine neue .ADT-Datei angelegt
wurde, oder der eingeloggte Benutzer verfügt nicht über ausreichend Rechte, so
ist der Test fehlgeschlagen. } \\
\hline
\textbf{Einzelschritte} & 
\textit{Zuerst wird die Webseite in einem Browser aufgerufen. Falls der
eingeloggte Benutzer nicht ausreichend Rechte besitzt, so wird der Export
abgebrochen. Durch den Klick auf \uline{Allegro-Export starten} wird die Datei angelegt und
zum Download angeboten. Es wird eine Meldung ausgegeben, falls es aktuell keine
Dokumente zu exportieren gibt. Anschließend wird auf Serverseite die Datei auf
Korrektheit überprüft. } \\
\hline
\textbf{Beobachtungen / Log} &  \textit{Alle Fehler beziehungsweise Logs werden
über die Konsole des Servers ausgegeben. } \\
\hline

 \end{longtable}

Im Folgenden werden die beiden möglichen erfolgreichen Testläufe für den
Universitätsbibliotheks - Export beschrieben. Angenommen wird dabei, dass
Django, der Server, sowie alle benötigten Komponenten korrekt konfiguriert
sind, da somit Fehler durch die Umgebung ausgeschlossen sind.

\begin{longtable}{|p{5cm}|p{10cm}|}
\hline
\textbf{Testfall -- ID und Bezeichnung} &
\textit{/T231/ Universitätsbibliotheks - Export} \\
\hline 
\textbf{Testlauf Nr.} & \textit{1} \\
\hline
\textbf{Eingaben} & \textit{Erforderliche Eingaben für den Export sind Einträge in der
Datenbank, ein eingeloggter Benutzer und ein Klick auf \uline{Allegro-Export
starten}. Weitere Eingaben sind nicht notwendig. } \\
\hline
\textbf{Soll - Reaktion} & \textit{Die neu erstellte .ADT-Datei wird erstellt
und über den Browser zum Download angeboten, oder es wird die Meldung gegeben,
dass keine Dokumente exportiert werden müssen. } \\
\hline
\textbf{Ist -- Reaktion} & \textit{Die .ADT-Datei wird erstellt und zum
Download angeboten. } \\
\hline
\textbf{Ergebnis} & \textit{Der Test ist erfolgreich abgelaufen. } \\
\hline
\end{longtable}

\begin{longtable}{|p{5cm}|p{10cm}|}
\hline
\textbf{Testfall -- ID und Bezeichnung} &
\textit{/T231/ Universitätsbibliotheks - Export} \\
\hline 
\textbf{Testlauf Nr.} & \textit{2} \\
\hline
\textbf{Eingaben} & \textit{Erforderliche Eingaben für den Export sind Einträge in der
Datenbank, ein eingeloggter Benutzer und ein Klick auf \uline{Allegro-Export
starten}. Weitere Eingaben sind nicht notwendig. } \\
\hline
\textbf{Soll - Reaktion} & \textit{Die neu erstellte .ADT-Datei wird erstellt
und über den Browser zum Download angeboten, oder es wird die Meldung gegeben,
dass keine Dokumente exportiert werden müssen. } \\
\hline
\textbf{Ist -- Reaktion} & \textit{Es wird die Meldung gegeben, dass keine
Dokumente exportiert werden müssen. } \\
\hline
\textbf{Ergebnis} & \textit{Der Test ist erfolgreich abgelaufen. } \\
\hline
\end{longtable}
