\subsection{Testfall -- /T222/ Ausleihe übertragen}
Die folgende Tabelle beschreibt den Testfall: T222 Ausleihe übertragen. \\
\begin{longtable}{|p{5cm}|p{10cm}|}
\hline
\textbf{Testfall -- ID und Bezeichnung} &  \textnormal{/T222/ Ausleihe übertragen} \\
\hline
\textbf{Zu testende Objekte und Methoden} & \textnormal{
\begin{itemize}
  \item In der Komponente \textit{Models} die Datei \lstinline{doc\_assign.html}
  \item In der Komponente \textit{Models} die Datei \lstinline{views.py}
\end{itemize}
} \\
\hline
\textbf{Kriterien f\"ur erfolgreiche bzw. fehlgeschlagene Testf\"alle} &
\textnormal{Der Test ist erfolgreich, wenn das Dokument dem User zugeordent wurde
und nicht mehr in der Liste ausgeliehener Dokumente des aktuellen Users
auftaucht.
Der Test schlägt fehl, wenn das Dokument immernoch dem ursprünglichem User
zugewiesen ist.} \\
\hline
\textbf{Einzelschritte} &  \textnormal{
Über das Userinterface wird auf den Testserver zugegriffen.
Es muss ein Dokument auf einen beliebigen User ausgeliehen sein.
Über die Dokumentinformationen gelangt man über den Button
\uline{Übertragen} zu der benötigten Form.
Dort kann ein User ausgewählt werden, oder das Dokument auf den angemeldeten
Nutzer übertragen werden.} \\
\hline
\textbf{Beobachtungen / Log} &  \textnormal{
Über den Browser wird eine Veränderung der Website beobachtet.
Über das Terminal werden Ladezeiten sowie Schreib-/Lesevorgänge verfolgt.
} \\
\hline
\end{longtable}

In der folgenden Tabelle werden die Testergebnisse zum Testfall: T222
Ausleihe übertragen beschrieben.

\begin{longtable}{|p{5cm}|p{10cm}|}
\hline
\textbf{Testfall -- ID und Bezeichnung} & \textnormal{/T222/ Ausleihe übertragen} \\
\hline
\textbf{Testlauf Nr.} & \textnormal{1} \\
\hline
\textbf{Eingaben} & \textnormal{
Der angemeldete User ist nicht der User, welcher das Dokument
entliehen hat. Eingabe: Angemeldeter User.} \\
\hline
\textbf{Soll - Reaktion} & \textnormal{
Der User wird auf die Seite mit den Dokumentinformationen
zurückgeleitet, welches nun an den bestimmten User entliehen ist.
Das Dokument ist jetzt in der Ausleihliste des Users.
} \\
\hline
\textbf{Ist -- Reaktion} & \textnormal{
Der User befindet sich auf der Seite mit den Dokumentinformationen.
Das Dokument kann von ihm mittels des Buttons \uline{Rückgabe}
zurückgegeben werden. Somit ist es in seiner Ausleihliste.} \\
\hline
\textbf{Ergebnis} & \textnormal{
Test war erfolgreich.} \\
\hline
\end{longtable}

\begin{longtable}{|p{5cm}|p{10cm}|}
\hline
\textbf{Testfall -- ID und Bezeichnung} & \textnormal{/T222/ Ausleihe
übertragen} \\
\hline
\textbf{Testlauf Nr.} & \textnormal{2} \\
\hline
\textbf{Eingaben} & \textnormal{
Der angemeldete User ist der User, welcher das Dokument entliehen hat. Der
User wählt einen anderen aus einer Liste existierender User aus.} \\
\hline
\textbf{Soll - Reaktion} & \textnormal{
Der User wird auf die Seite mit den Dokumentinformationen
zurückgeleitet, welches nun an den bestimmten User entliehen ist.
Das Dokument ist jetzt nicht mehr in der Ausleihliste des Users.
} \\
\hline
\textbf{Ist -- Reaktion} & \textnormal{
Der User befindet sich auf der Seite mit den Dokumentinformationen.
Das Dokument kann nicht mehr von ihm mittels des Buttons \uline{Rückgabe}
zurückgegeben werden. Er kann es aber wieder mittels des Buttons
\uline{Übertragen} auf sich übertragen.} \\
\hline
\textbf{Ergebnis} & \textnormal{
Test war erfolgreich.} \\
\hline
\end{longtable}

