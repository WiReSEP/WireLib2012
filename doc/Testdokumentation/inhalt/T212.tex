\subsection{T212 - Erweiterte Suche}

Dieser Testfall testet die Erweiterte Suche mit verschiedenen mögliche Eingaben.

\begin{longtable}{|p{5cm}|p{10cm}|}
\hline
\textbf{Testfall -- ID und Bezeichnung} &  \textit{/T212 Erweiterte Suche} \\
\hline
\textbf{Zu testende Objekte und Methoden} & \textnormal{ 
\begin{itemize}
\item In Komponente \emph{Template} die Datei \lstinline{search_pro.html}
\item In Komponente \emph{Views} die Datei \lstinline{search_pro}
\end{itemize}
}\\
\hline
\textbf{Kriterien f\"ur erfolgreiche bzw. fehlgeschlagene Testf\"alle} &
\textit{Ein Test ist erfolgreich, wenn als Ausgabe die in der Datenbank
vorhanden auf die Suchanfrage passenden Dokumente ausgegeben werden.} \\
\hline
\textbf{Einzelschritte} &  \textit{Zuerst muss per Browser die Webseite  
der erweiterten Suche aufgerufen werden. Danach werden je nach Testlauf die 
Felder der erweiterten Suche ausgefüllt und die Suche per Klick auf den 
entsprechenden Button gestartet. Dies wird mit allen im Testplan angegebenen
Browsern durchgeführt. Der Prozess sollte in jedem Fall terminieren und entweder
eine Fehlermeldung oder eine Erfolgsmeldung mitteilen. Sollte sich wieder
erwarten der Prozess nicht terminieren ist der Django Testserver zu beenden.
Aufräumschritte sind in keinem Fall nötig.} \\
\hline
\textbf{Beobachtungen / Log} &  \textit{Zur Beobachtung werden die Ausgabe des
Browsers und des Testservers auf der Konsole benutzt.}\\
\hline
 \end{longtable}

%Die folgenden Tabellen beschreiben, wie der Testfall ausgef\"uhrt wurde und
%welches Ergebnis er geliefert hat. Da es bei Korrektur von Softwarefehlern oder
%anderen Gegebenheiten notwendig ist, einen Test mehrfach durchzuf\"uhren
%(Testl\"aufe), ist jede Testdurchf\"uhrung zu dokumentieren. Daher ist diese
%Tabelle f\"ur \textbf{jeden Testlauf }  zu erstellen und \textbf{ fortlaufend zu
%nummerieren}. \\

Dieser Testlauf testet den einfachsten Fall eines einzelnen Wortes als
Suchanfrage.

\begin{longtable}{|p{5cm}|p{10cm}|}
\hline
\textbf{Testfall -- ID und Bezeichnung} & \textit{/T212/ Erweiterte
Suche} \\
\hline
\textbf{Testlauf Nr.} & \textit{1} \\
\hline
\textbf{Eingaben} & \textit{Es wird der String \lstinline{Analysis} in das
das Eingabefeld des Titels eingetragen.} \\
\hline
\textbf{Soll - Reaktion} & \textit{Es werden alle Bücher mit
\lstinline{Analysis} im Titel angzeigt.
} \\
\hline
\textbf{Ist -- Reaktion} & \textit{Entspricht der Soll-Reaktion} \\
\hline
\textbf{Ergebnis} & \textit{Der Test ist erfolgreich verlaufen} \\
\hline
\textbf{Unvorhergesehene Ereignisse w\"ahrend des Test-laufs } &
\textit{Opera zeigt die Buttons nicht mit transparentem Hintergrund an. Dies
behindert allerdings nicht die Funktionalität.} \\
\hline
\end{longtable}

Dieser Testlauf testen den Fall, dass ein Suchwort eingegeben, welches aus
mehreren Wortfragmenten besteht, welche in nicht richtiger Reihenfolge
eingegeben werden.

\begin{longtable}{|p{5cm}|p{10cm}|}
\hline
\textbf{Testfall -- ID und Bezeichnung} & \textit{/T212/ Erweiterte
Suche} \\
\hline
\textbf{Testlauf Nr.} & \textit{2} \\
\hline
\textbf{Eingaben} & \textit{Es wird der String \lstinline{Ana Func} in das
das Eingabefeld des Titels eingetragen.} \\
\hline
\textbf{Soll - Reaktion} & \textit{Es werden alle Bücher mit
\lstinline{Functional Analysis} im Titel angzeigt.
} \\
\hline
\textbf{Ist -- Reaktion} & \textit{Entspricht der Soll-Reaktion} \\
\hline
\textbf{Ergebnis} & \textit{Der Test ist erfolgreich verlaufen} \\
\hline
\textbf{Unvorhergesehene Ereignisse w\"ahrend des Test-laufs } &
\textit{Opera zeigt die Buttons nicht mit transparentem Hintergrund an. Dies
behindert allerdings nicht die Funktionalität.} \\
\hline
\end{longtable}

Dieser Testlauf testet den Fall das viele Eingaben gemacht werden, sodass
nurnoch ein einzelnes Dokumet als Ergebnis ausgegeben wird.

\begin{longtable}{|p{5cm}|p{10cm}|}
\hline
\textbf{Testfall -- ID und Bezeichnung} & \textit{/T212/ Erweiterte
Suche} \\
\hline
\textbf{Testlauf Nr.} & \textit{3} \\
\hline
\textbf{Eingaben} & \textit{Es wird 
der String \lstinline{Analysis} in das das Eingabefeld des Titels eingetragen,
der String \lstinline{Conway} in das das Eingabefeld des Authors eingetragen,
der String \lstinline{1995} in das das Eingabefeld des Erscheinungsjahrs
eingetragen,
der String \lstinline{Springer} in das das Eingabefeld des Herausgebers
eingetragen,
der String \lstinline{K056031} in das das Eingabefeld der Bibliotheksnummer eingetragen
.} \\
\hline
\textbf{Soll - Reaktion} & \textit{Es wird die Detailansicht des Buches \emph{A
Course of Functional Analysis} angezeigt.} \\
\hline
\textbf{Ist -- Reaktion} & \textit{Entspricht der Soll-Reaktion} \\
\hline
\textbf{Ergebnis} & \textit{Der Test ist erfolgreich verlaufen} \\
\hline
\textbf{Unvorhergesehene Ereignisse w\"ahrend des Test-laufs } &
\textit{Opera zeigt die Buttons nicht mit transparentem Hintergrund an. Dies
behindert allerdings nicht die Funktionalität.} \\
\hline
\end{longtable}
