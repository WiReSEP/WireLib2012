\subsection{Testfall -- /T201/ Webinterface - Import}
In der folgenden Tabelle wird der Test des Imports über das Webinterface
beschrieben.
\begin{longtable}{|p{5cm}|p{10cm}|}
\hline
\textbf{Testfall -- ID und Bezeichnung} &  \textit{/T201/ Webinterface - Import} \\
\hline
\textbf{Zu testende Objekte und Methoden} &  
\textit{
\begin{itemize}
  \item In Komponente \emph{views} die Funktion \lstinline{doc_add}
  \item In Komponente \emph{Template} die Datei \emph{doc\_add.html}
\end{itemize}}
\\
\hline
\textbf{Kriterien f\"ur erfolgreiche bzw. fehlgeschlagene Testf\"alle} &
\textit{Wenn die eingegebenen Daten korrekt sind, so wird das Dokument in die
Datenbank übernommen. Bei unvollständigen Daten wird die Webseite mit einer
entprechenden Meldung erneut aufgerufen. } \\
\hline
\textbf{Einzelschritte} &  \textit{Zuerst wird die Webseite in einem Browser
aufgerufen. Falls der eingeloggte Benutzer nicht über ausreichend Rechte
verfügt, so wird das Laden der Webseite abgebrochen. Andernfalls wird die
Eingabeform für ein Dokument angezeigt und beim Absenden in die Datenabnk
übernommen. } \\
\hline
\textbf{Beobachtungen / Log} &  \textit{Der Server loggt in der Konsole, dass
doc\_add angefordert wurde. Informationen über das Hinzufügen des Dokumentes
werden nicht ausgegeben. } \\
\hline
\textbf{Abh\"angigkeiten} &  \textit{Dieser Test ist von dem \BibTeX - Import
abhängig, da es auch anstelle von Texteingabemasken auch ein \BibTeX - Uploader
zur Verfügung steht. } \\
\hline

 \end{longtable}

Im Folgenden werden drei mögliche Testfälle für den Webinterface - Import
beschrieben. \\


\begin{longtable}{|p{5cm}|p{10cm}|}
\hline
\textbf{Testfall -- ID und Bezeichnung} & \textit{/T201/ Webinterface - Import} \\
\hline
\textbf{Testlauf Nr.} & \textit{1} \\
\hline
\textbf{Eingaben} & \textit{Eingaben sind Einträge in der Datenbank, sowie alle
Eingaben in der Eingabemaske. In der Eingabemaske sind alle benötigten Felder
bedient und der \BibTeX - Upload ist leer. } \\
\hline
\textbf{Soll - Reaktion} & \textit{Die eingegebenen Daten werden als Dokument
in die Datenbank übernommen. } \\
\hline
\textbf{Ist -- Reaktion} & \textit{Wie Soll - Reaktion.} \\
\hline
\textbf{Ergebnis} & \textit{Der Test verlief erfolgreich. } \\
\hline
 \end{longtable}

\begin{longtable}{|p{5cm}|p{10cm}|}
\hline
\textbf{Testfall -- ID und Bezeichnung} & \textit{/T201/ Webinterface - Import} \\
\hline
\textbf{Testlauf Nr.} & \textit{2} \\
\hline
\textbf{Eingaben} & \textit{Eingaben sind Einträge in der Datenbank, sowie alle
Eingaben in der Eingabemaske. In der Eingabemaske sind nicht alle benötigten Felder
bedient und der \BibTeX - Upload ist leer. } \\
\hline
\textbf{Soll - Reaktion} & \textit{Die eingegebenen Daten werden nicht als Dokument
in die Datenbank übernommen und die Webseite wird erneut angezeigt. } \\
\hline
\textbf{Ist -- Reaktion} & \textit{Wie Soll - Reaktion.} \\
\hline
\textbf{Ergebnis} & \textit{Der Test verlief erfolgreich. } \\
\hline
 \end{longtable}

\begin{longtable}{|p{5cm}|p{10cm}|}
\hline
\textbf{Testfall -- ID und Bezeichnung} & \textit{/T201/ Webinterface - Import} \\
\hline
\textbf{Testlauf Nr.} & \textit{3} \\
\hline
\textbf{Eingaben} & \textit{Eingaben sind Einträge in der Datenbank, sowie alle
Eingaben in der Eingabemaske. In der Eingabemaske sind nicht alle benötigten Felder
bedient und der \BibTeX - Upload ist befüllt. } \\
\hline
\textbf{Soll - Reaktion} & \textit{Für die hochgeladene Datei wird der \BibTeX
- Importer aufgerufen. Alle Fehler werden dem Nutzer dabei auf der Webseite
angezeigt.  } \\
\hline
\textbf{Ist -- Reaktion} & \textit{Wie Soll - Reaktion.} \\
\hline
\textbf{Ergebnis} & \textit{Der Test verlief erfolgreich. } \\
\hline
 \end{longtable}
