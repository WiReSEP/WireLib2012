%------Ändern von Schriftschnitten - (Muss ganz am Anfang stehen !) ------------
\usepackage{fix-cm}

%------Umlaute -----------------------------------------------------------------
%   Umlaute/Sonderzeichen wie äüöß können direkt im Quelltext verwenden werden.
%    Erlaubt automatische Trennung von Worten mit Umlauten.
\usepackage[T1]{fontenc}
\usepackage[utf8]{inputenc}

%------Anpassung der Landessprache----------------------------------------------
\usepackage{ngerman}

%------Einfache Definition der Zeilenabstände und Seitenränder------------------
\usepackage{geometry}
\usepackage{setspace}

%------Schriftgrößenanpassung von einzelnen Textpassagen------------------------
\usepackage{relsize}

%------Trennlinien in Kopf- und Fusszeile
\usepackage[headsepline, footsepline, ilines]{scrpage2}

%------Grafiken-----------------------------------------------------------------
\usepackage{graphicx}

%------Packet zum Sperren, Unterstreichen und Hervorheben von Texten------------
\usepackage{soul}

%------ergänzende Schriftart----------------------------------------------------
\usepackage{helvet}

%------Lange Tabellen-----------------------------------------------------------
\usepackage{longtable}
\usepackage{array}
\usepackage{ragged2e}
\usepackage{lscape}

%------PDF-Optionen-------------------------------------------------------------
\usepackage[
  bookmarks,
  bookmarksopen=true,
  colorlinks=true,
  linkcolor=black,        % einfache interne Verknüpfungen
  anchorcolor=black,      % Ankertext
  citecolor=black,        % Verweise auf Literaturverzeichniseinträge im Text
  filecolor=black,        % Verknüpfungen, die lokale Dateien öffnen
  menucolor=black,        % Acrobat-Menüpunkte
  urlcolor=black,         % Farbe für URL-Links
  backref,                % Zurücktext nach jedem Bibliografie-Eintrag als
                          % Liste von Überschriftsnummern
  pagebackref,            % Zurücktext nach jedem Bibliografie-Eintrag als
                          % Liste von Seitenzahlen
  plainpages=false,       % zur korrekten Erstellung der Bookmarks
  pdfpagelabels,          % zur korrekten Erstellung der Bookmarks
  hypertexnames=false,    % zur korrekten Erstellung der Bookmarks
  linktocpage             % Seitenzahlen anstatt Text im Inhaltsverzeichnis
                          % verlinken
  ]{hyperref}

\usepackage{translator}
\usepackage[				   %
acronym,		% ein Abkürzungsverzeichnis erzeugen
toc,			% Taucht im Inhaltsverzeichnis auf
]				   
{glossaries}
\makeglossaries

\usepackage{xcolor}
\definecolor{lightergray}{rgb}{0.9,0.9,0.9}

\usepackage{listings}
\lstset{numbers=left, numberstyle=\tiny, numbersep=5pt,
breaklines=true, backgroundcolor=\color{lightergray},
basicstyle=\ttfamily,
}
\usepackage{booktabs}
\usepackage{xspace}
