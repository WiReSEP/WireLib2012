% Kapitel 1
% Die Unterkapitel können auch in separaten Dateien stehen,
% die dann mit dem \include-Befehl eingebunden werden.
%-------------------------------------------------------------------------------

\chapter{Zielbestimmung}

%Hier Einleitungstext einfügen, dabei die Formatierungen selber erstellen
Dieser Abschnitt hat die Aufgabe, als eine Art Einleitung zu dienen. Es soll
ein kurzer Umriss über Ziel und Motivation des Gesamt- und ggf. der
Teilprojekte dargestellt werden. Beschrieben wird die Hauptaufgabe des Systems.
Wichtig ist, den Grund für die Systementwicklung (Probleme oder Geschäftsidee)
und damit ihre Ziele herauszuarbeiten.
\section{Musskriterien}

Hier wird aufgeführt, welche Funktionalitäten/Leistungen das Softwareprodukt in
jedem Fall erfüllen muss, damit es genutzt werden kann.
\section{Wunschkriterien}
Dies sind Kriterien, die für die Lauffähigkeit des Produkts nicht zwingend
erforderlich sind, für die Erreichung der Projektziele aber erfüllt werden
sollten.

\section{Abgrenzungskriterien}
Hier ist zu verdeutlichen, welche Ziele mit dem Produkt bewusst nicht erreicht
werden sollen oder werden können.
