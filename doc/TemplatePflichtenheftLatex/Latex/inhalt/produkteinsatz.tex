% Kapitel 2 mit den entsprechenden Unterkapiteln
% Die Unterkapitel können auch in separaten Dateien stehen,
% die dann mit dem \include-Befehl eingebunden werden.
%------------------------------------------------------------------------------------
\chapter{Produkteinsatz}
Dieser Abschnitt hat die Aufgabe, den Einsatzbereich, die Zielgruppen und die
Betriebsbedingungen des zu entwickelnden Systems klarzustellen.

\section{Anwendungsbereiche}
Hier wird der Bereich beschrieben, in dem das Produkt eingesetzt werden soll,
bzw. Bereiche, für die das Produkt nicht gedacht ist.

\section{Zielgruppen}
Hier wird angegeben, für welche Anwender (z. B. Sekretärin, andere Entwickler)
das Produkt im Wesentlichen gedacht bzw. nicht gedacht ist.

\section{Betriebsbedingungen}
Hier werden die unterschiedlichen Bedürfnisse und Anforderungen an das Produkt
aufgelistet. Dies können folgenden Punkte sein:
-  physikalische Umgebung des Systems (z. B. Büroumgebung, mobiler Einsatz)
-  tägliche Betriebszeit (z. B. Dauerbetrieb)
-  ständige Beobachtung des Systems durch einen Bediener oder unbeaufsichtigter
   Betrieb
