\section{Analyse von Funktion F220: Ausleihe}
Ein oder mehrere Dokumente werden von einem Benutzer ausgeliehen. Dabei wählt der Benutzer die gewünschten Dokumente aus und das System muss in der Datenbank nachprüfen, ob diese derzeit ausleihbar sind. Der Status des Dokumentes müsste also \"0\" sein. Wenn dies der Fall sein sollte, werden dem Benutzer diese Bücher zugewiesen.

\section{Analyse von Funktion F221: Ausleihe an Externe}
Die Ausleihe an Externe ist ähnlich gestaltet wie F220. Es existiert lediglich der Zusatz, dass der Benutzer einen Externen als eigentlichen Ausleiher angibt. Dafür wird beim Entleihen die Möglichkeit gegeben Daten über den Externen anzugeben, dessen Daten dann gespeichert und beim Entleihen vermerkt werden.

\section{Analyse von Funktion F222: Ausleihe übertragen}
Ein Dokument wechselt den Ausleihenden. Dabei muss ein Benutzer ein von ihm ausgeliehenes Dokument auswählen und einem anderen Benutzer übertragen. Das System vermerkt dieses Austausch dann in der Datenbank. 

\section{Analyse von Funktion F223: Ausleihe zurückgeben}
Ein Benutzer braucht ein Dokument nicht mehr und gibt es in den Bestand der Bibliothek zurück. Auch hierfür muss der Benutzer als Entleiher des Dokumentes eingetragen sein. Wenn dies der Fall sein sollte, kann er \emph{zurückgeben} wählen und das Dokument wird auch im Datensatz mit dem Status \"0\" versehen.

\section{Analyse von Funktion F224: (Optional) Ausleihe vermisst melden}
Das Dokument befindet sich nicht mehr an dem laut Datenbank befindlichen Ort. Dann kann ein Benutzer ein Dokument auf der Dokumentenansicht als \emph{vermisst} melden und dieses wird in der Datenbank mittels Statusänderung vermerkt. Dazu wird eine E-Mail an alle Benutzer verschickt oder alternativ eine Meldung auf der Homepage angezeigt.

\section{Analyse von Funktion F225: Ausleihe verloren melden}
Das Dokument ist auch nach einer Vermisstenmeldung nicht wieder aufgetaucht. Dann ist ein Bibliothekar berechtigt, es als \emph{verloren gegangen} einzustufen. Das Dokument bekommt den entsprechenden Status und wird demnächst bei weiteren Suchanfragen ausgeschlossen.