\section{Analyse von Funktion F220: Ausleihe}
\label{f:220}
Ein oder mehrere Dokumente werden von einem Benutzer ausgeliehen. Dabei wählt der Benutzer die gewünschten Dokumente aus und das System muss in der Datenbank nachprüfen, ob diese derzeit ausleihbar sind. Der Status des Dokumentes müsste also \"0\" sein. Wenn dies der Fall sein sollte, werden dem Benutzer diese Bücher zugewiesen. Dies erfolgt durch zwei Built-In-Funktion von Django mit dessen Hilfe die Datenbank erweitert bzw.\ ein Datensatz geupdatet werden kann.

\section{Analyse von Funktion F221: Ausleihe an Externe}
Die Ausleihe an Externe ist ähnlich gestaltet wie in \ref{f:220}\nameref{f:220}. Es existiert lediglich der Zusatz, dass der Benutzer einen Externen als eigentlichen Ausleiher angibt. Dafür wird beim Entleihen die Möglichkeit gegeben Daten über den Externen anzugeben, dessen Daten dann gespeichert und beim Entleihen vermerkt werden. Auch hierbei wird eine Built-In-Funktion zum erweitern der Datenbank genutzt.

\section{Analyse von Funktion F222: Ausleihe übertragen}
Ein Dokument wechselt den Ausleihenden. Dabei muss ein Benutzer $A$ ein von ihm ausgeliehenes Dokument auswählen und einem anderen Benutzer $B$ übertragen. Alternative kann Benutzer $B$ auch eintragen, dass er sich ein Dokument von Benutzer $A$ geholt hat, das Dokument also jetzt bei ihm zu finden ist. Das System vermerkt dieses Austausch dann in der Datenbank mittels einer Built-In-Funktion von Django. 

\section{Analyse von Funktion F223: Ausleihe zurückgeben}
Ein Benutzer braucht ein Dokument nicht mehr und gibt es in den Bestand der Bibliothek zurück. Auch hierfür muss der Benutzer als Entleiher des Dokumentes eingetragen sein. Wenn dies der Fall sein sollte, kann er \emph{zurückgeben} wählen und das Dokument wird auch im Datensatz mittels Built-In-Funktionen von Django mit dem Status \"0\" versehen.

\section{Analyse von Funktion F224: (Optional) Ausleihe vermisst melden}
Das Dokument befindet sich nicht mehr an dem laut Datenbank befindlichen Ort. Dann kann ein Benutzer ein Dokument auf der Dokumentenansicht als \emph{vermisst} melden und dieses wird in der Datenbank mittels Statusänderung vermerkt. Dazu wird eine E-Mail an alle Benutzer verschickt oder alternativ eine Meldung auf der Homepage angezeigt. Hierfür muss eine neue App erstellt werden, die den Status ändert, die E-Mail-Adressen, den -Text und die Dokumentinformationen aufruft, verarbeitet und an alle verschickt oder alternativ einen entsprechenden Beitrag auf der Homepage erstellt.

\section{Analyse von Funktion F225: Ausleihe verloren melden}
Das Dokument ist auch nach einer Vermisstenmeldung nicht wieder aufgetaucht. Dann ist ein Bibliothekar berechtigt, es als \emph{verloren gegangen} einzustufen. Das Dokument bekommt den entsprechenden Status und wird demnächst bei weiteren Suchanfragen ausgeschlossen. Dies erfolgt wieder durch eine Built-In-Funktion von Django.