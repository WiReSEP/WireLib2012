\section{Analyse von Funktion F100 (Anmeldung)}
Der Benutzer meldet sich auf dem System unter seinem Benutzernamen an.

Diese Funktion, die lediglich in der Komponente \textbf{View} vorkommt, ist
eine Built-in-Funktion von Django. Innerhalb von Django werden die Benutzer
in einer eigenen Datenstruktur  dargestellt. Diese Klasse muss erweitert werden,
um die zusätzlichen Daten der Benutzer zu speichern. Da Django eine Datenbank
zugrunde liegt, müssen diese weiteren Daten in einer extra Klasse mit Referenz
auf die Benutzer gesetzt werden.

\section{Analyse von Funktion F101 (Anmeldung über LDAP)}
Der Benutzer meldet sich auf dem System über seinen \Gls{GITZ}-Benutzer an.

Diese Funktion wird in einer extra App dargestellt. Die \gls{LDAP}-Anmeldung muss
dabei die normale Anmeldung erweitern und im View evtl. durch eine Checkbox zur
Verfügung gestellt werden. Die zugrundeliegende Datenstruktur wird von \gls{LDAP}
bereit gestellt.

\section{Analyse von Funktion F102 (Anbindung an die Universitätsbibliothek)}
Ein Benutzer mit hinreichenden Rechten verschickt an die Universitätsbibliothek
den Stand der aktuellen Datensätze.

Der zugrunde liegende Datensatz der Dokumente wird von einem, in den
\textbf{Models} gelegenen Befehlssatz in ein Datenformat übersetzt, welches von
der TU Braunschweig Universitätsbibliothek gelesen werden kann. 

\section{Analyse von Funktion F103 (Mailtexte ändern)}
Die Standard-Mails, die vom System an Benutzer verschickt werden, sollen geändert
werden.

Die Mailtexte sind Bestandteil einer veränderbaren, aber nicht großen Tabelle an
Basisinformationen, wie auch andere Einstellungen für die Software. Um die
Informationen bereit zu stellen, muss eine neue Klasse in den \textbf{Models}
erzeugt werden, die alle Informationen für die App bereit stellt.

\section{Analyse von Funktion F201 (Bib\TeX\ import)}
Ein oder mehrere Dokumente werden zum System hinzugefügt. Dafür wählt der
Benutzer den entsprechenden View aus und lädt über einen Upload-Dialog eine
\BibTeX -Datei hoch. Das System analysiert die Datei auf Validität, erzeugt
Objekte aus den Daten, die in eine Datenbank gespeichert werden.

Für die Funktion wird also ein \textbf{View} benötigt, der den Upload bereit
stellt.  Als Basis werden die Dokument-Objekte aus den \textbf{Models} benötigt,
die auch von vielen weiteren Funktionen verwendet werden und die auch über
Django die Datenbankeinträge verwalten.
