\section{Analyse von Funktion F100 (Anmeldung)}
Diese Funktion ist eine Build-In-Funktion von Django. Innerhalb von Django
werden die Benutzer durch eine eigenständige Klasse dargestellt. Diese Klasse
muss erweitert werden um die zusätzlichen Daten der Benutzer zu speichern. Da
Django eine Datenkbank zugrunde liegt müssen diese weiteren Daten in einer Extra
Klasse mit Referenz auf die Benutzer gesetzt werden.

\section{Analyse von Funktion F101 (Anmeldung über LDAP)}
Diese Funktion muss in einer extra App dargestellt werden. Die LDAP-Anmeldung
muss dabei die normale Anmeldung erweitern und im View evtl. durch eine Checkbox
zur Verfügung gestellt werden.

\section{Analyse von Funktion F102 (Anbindung an die Universitätsbibliothek)}
% TODO: Diese Analyse muss von jmd. geschrieben werden der sich mir der UB
% auseinander gesetzt hat.

\section{Analyse von Funktion F103 (Mailtexte ändern)}
Die Mailtexte sind Bestandteil einer veränderbaren aber nicht großen Tabelle an
Basisinformationen wie auch andere Einstellungen für die Software. Um die
Informationen bereit zu stellen muss eine neue Klasse erzeugt werden, das alle
Informationen für die App bereit stellt.
