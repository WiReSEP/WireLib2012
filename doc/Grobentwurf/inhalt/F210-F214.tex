\section{Analyse von Funktion F210 (Generelle Suche)}
Diese Funktion ist eine Build-In Funktion von Django. Das gesuchte Wort wird einfach mithilfe einer Search-Query innerhalb der Datenbank gesucht und ausgegeben. 

\section{Analyse von Funktion F211 (Suche mit erweiterten Ausdrücken)}
Auch diese Suche wird von Django mithilfe von mehreren Querysets bereitgestellt und ist daher bereits eine Built-In Funktion.

\section{Analyse von Funktion 212 (erweiterte Suche)}
Im FrontEnd werden die bestätigten Parameter (aus der Maske für die erweiterte Suche) den dazu passenden Built-In Querysets im System übergeben. 

\section{Analyse von Funktion 213 (Suche mit regulären Ausdrücken)}
Diese Funktion muss in einer extra App dargestellt werden, wo genau bestimmt wird, welche aus der FrontEnd ermittelten Zeichenfolgen in der Suchanfrage bestimmte Querysets auslösen.

\section{Analyse von Funktion 214 (Sortierung der Suche)}
Auch diese Funktion muss in einer extra App dargestellt werden. Das vorläufige Ergebnis aus einer Suchanfrage wird tabellarisch angezeigt, dessen Spalten aber interaktiv sind: Die Spaltennamen lösen system-intern Funktionen aus, die aus Build-In Querysets bestehen, die genau bestimmen, wie die Ergebnisse dargestellt werden sollen (in diesem Fall alphabetisch aufwärts/abwärts). 



 