\section{Analyse von Funktion F211 (Suche mit erweiterten Ausdrücken)}
Auch hier gibt der Benutzer einen oder mehrere Begriffe in die Suchmaske ein, kann aber nun zusätzlich bestimmte Teilworte (and, or etc.) in das Suchfeld eingeben, um seine Suche zu präzisieren.  

Auch diese Suche wird von Django mithilfe von mehreren Querysets bereitgestellt und ist daher bereits eine BuiltIn Funktion. Diese Funktion gehört zur Komponente \textbf{Model:Suche}. Die Ergebnisse werden durch einen View ausgegeben. 
Da der Funktionsaufbau von Suchanfragen trivial sind (Anfrage an den Server/Datenbank, Response zurück), werden auf die Grafiken zu F211-F214 verzichtet.   

\section{Analyse von Funktion 212 (erweiterte Suche)}
Durch diese Funktion ist der Benutzer in der Lage, mithilfe einer erweiterten Suchmaske bestimmte Felder bzw. Checkbox zu bestätigen, die seine Suche spezifizieren sollen.
    
Diese Funktion liest zuerst die im FrontEnd bestätigten Parameter (aus der Maske für die erweiterte Suche) aus und übergibt sie der Objekt-Suche, deren benötigte Funktionen bereits von Django zur Verfügung gestellt wird. Somit gehört auch sie zur Komponente \textbf{Model:Suche}. Durch einen View werden die Ergebnisse ausgegeben. 

\section{Analyse von Funktion 213 (Suche mit regulären Ausdrücken)}
In der erweiteren Suchmaske ist der Benutzer zusätzlich in der Lage, \gls{glos:regex} in das Suchfeld einzugeben, um seine Suche zu vereinfachen. 

Als Teil der Komponente \textbf{Model:Suche} benutzt die Funktion die \gls{glos:regex} (aus dem FrontEnd View ausgelesen) für die Dokumenten-Suche, die Ergebnisse werden dann durch einen View dargestellt.
 
\section{Analyse von Funktion 214 (Sortierung der Suche)}
Nach einer Suchanfrage werden die Ergebnisse tabellarisch aufgelistet. Der Benutzer ist nun in der Lage durch Interaktion mit gewünschten Spaltennamen, diese Ergebnisse alphabetisch aufwärts/abwärts zu sortieren. 

Die Funktion startet in Abhängigkeit von dem bestätigten Spaltenparameter eine Sortierung des aktuellen Views mithilfe von (in Django enthalten) Querysets und passende Filter. Die neue Ergebnisauflistung wird durch einen neuen View dargestellt. Die Funktion ist Teil der Komponente \textbf{Model}. 




 