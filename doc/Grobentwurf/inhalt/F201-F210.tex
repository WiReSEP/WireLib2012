\section{Analyse von Funktion F201 (Webinterface-Import)}
\label{f:201}
Der Benutzer, der entsprechende Rechte hat, wählt einen View, der ein Interface
bietet um die Daten für einzelne Dokumente in das System ein zu pflegen. Dafür
wird ihm vom System eine Eingabemaske auf dem View geboten.

Diese Funktion stellt für den Benutzer eine weitere Schnittstelle zur internen
\BibTeX -Verwaltung. Es handelt sich also nur um einen weiteren View zu einer
bestehenden Funktion, weswegen auf eine Grafik verzichtet wird


\section{Analyse von Funktion F202 (Editieren von Dokumenten)}
% Grafik?
Der Benutzer, der entsprechnde Rechte hat, nimmt auf der Detailansicht eines
Dokumentes die Möglichkeit wahr, die Daten Dieses nachträglich zu ändern. Dafür
wählt der die entsprechnde Option und wird vom System auf eine Seite ähnlich der
aus \ref{f:201}\nameref{f:201} mit einer ausgefüllten Maske.

Diese Funktion muss zurest das Dokument auslesen um den View zu erzeugen und
dann die Einträge des Dokumentes entsprechend ändern. Eine solche Veränderung
von den Objekten benötigt lediglich entsprechnende set- und get-Methoden für die
Dokumente und einen entsprechenden View für Django.

\section{Analyse von Funktion F203 (Löschen von Dokumenten)}
Der Administrator, als einziger Benutzer mit den entsprechenden Privilegien,
löscht ein Dokument vollständig aus der Datenbank inkl. der Ausleihhistory.
Diese Aktion wird über einen View im Backend erledigt, der dort bereit steht.

Benötigt wird für Django ein entsprechender View im Backend, der die Möglichkeit
bietet die eingetragenen Dokumente im Backend ein zu sehen. Weiter wird die
zugrundeliegende Funktion der Löschung von Objekten aus der Datenbank benötigt,
die in Django bereits build-in ist. Aufgrund des nicht sehr komplexen
Funktionsaufbaus wurde auf eine Funktion verzichtet.

\section{Analyse von Funktion F210 (Generelle Suche)}
Der Benutzer gibt in eine Suchmaske einen oder mehrere Begriffe ein, um seine
Suche zu spezifizieren. Nach der Bestätigung der Eingabe gibt das System
entsprechende Treffer dem Benutzer in einer Liste zurück.

Die zugrundeliegenden Funktionen der Objekt-Suche nach bestimmten Parametern
wird bereits durch Django bereit gestellt. Der Such-View arbeitet auf dabei auf
einer Funktion die für die Dokumentfunktionen arbeitet.

Da diese Suche mit der weitläufig bekannten Google-Suche vergleichbar ist wird
auf eine Grafik verzichtet.
