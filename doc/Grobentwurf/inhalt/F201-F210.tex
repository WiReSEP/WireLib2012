\section{Analyse von Funktion F201: Webinterface-Import}
\label{f:201}
Der Benutzer, der entsprechende Rechte hat, wählt einen View, der ein Interface
bietet, um die Daten für einzelne Dokumente in das System einzupflegen. Dafür
wird ihm vom System eine Eingabemaske auf dem View geboten.

Als Teil der Komponente \textbf{View} stellt diese Funktion für den
Benutzer eine weitere Schnittstelle zur internen \BibTeX -Verwaltung zur Verfügung. Es wird
daher ein Zugriff auf die Komponente der \textbf{Models} benötigt.

\section{Analyse von Funktion F202: Editieren von Dokumenten}
% Grafik?
Der Benutzer, der entsprechende Rechte hat, nimmt auf der Detailansicht eines
Dokumentes die Möglichkeit wahr, die Daten dieses Dokumentes nachträglich zu ändern. Dafür
wählt er die entsprechende Option und wird vom System auf eine Seite ähnlich der
aus \ref{f:201} \nameref{f:201} mit einer ausgefüllten Maske geleitet.

Diese Funktion muss zu erst das Dokument auslesen, um den View zu erzeugen und
dann die Einträge des Dokumentes entsprechend zu ändern. Eine solche Veränderung
von den Objekten benötigt lediglich entsprechende set- und get-Methoden für die
Dokumente und einen entsprechenden View für Django. Der in der
\textbf{DB} zugrundeliegende Datensatz wird also über \textbf{View} mit
den \textbf{Models} manipuliert. Da diese Funktion nur in einem View
repräsentiert wird und eine einfache Eingabe-Ausgabe ist, wird auf eine
beschreibende Grafik verzichtet.

\section{Analyse von Funktion F203: Löschen von Dokumenten}
Der Administrator, als einziger Benutzer mit den entsprechenden Privilegien,
löscht ein Dokument vollständig aus der Datenbank inkl. der Ausleihhistory.
Diese Aktion wird über einen View im Backend erledigt, der dort bereit steht.

Benötigt wird für Django ein entsprechender \textbf{View} im Backend, der die
Möglichkeit bietet, die eingetragenen Dokumente im Backend einzusehen. Weiter
wird die zugrundeliegende Funktion der Löschung von Objekten aus der Datenbank
in den \textbf{Models} benötigt, die zur Basis in Django bereits Built-In ist,
jedoch evtl. erweitert werden muss. Aufgrund des nicht sehr komplexen
Funktionsaufbaus wurde auf eine Grafik verzichtet.

\section{Analyse von Funktion F210: Generelle Suche}
Der Benutzer gibt in eine Suchmaske einen oder mehrere Begriffe ein, um seine
Suche zu spezifizieren. Nach der Bestätigung der Eingabe gibt das System
entsprechende Treffer dem Benutzer in einer Liste zurück.

Die zugrundeliegenden Funktionen der Objekt-Suche nach bestimmten Parametern
wird bereits durch Django bereit gestellt. Der Such-View arbeitet dabei auf
einer Funktion, die für die Dokumentfunktionen arbeitet.

Da diese Suche mit der weitläufig bekannten Google-Suche vergleichbar ist, wird
auf eine Grafik verzichtet.
