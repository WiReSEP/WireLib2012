% Kapitel 9
%-------------------------------------------------------------------------------
%Hier werden Fachbegriffe und Abkürzungen erklährt.
% Verwendet werden diese Begriffe mit \gls{name} oder \Gls{name} wenn der Anfang
% groß geschrieben sein muss.
%
%%	Beispiel unter: http://ewus.de/tipp-1029.html
%%	und natürlich Erkärung mit dem Befehl
%texdoc glossaries

\newacronym{UB}{UB}{Universitätsbibliothek}
\newacronym{GITZ}{GITZ}{Gauß-IT-Zentrum}
\newacronym{LDAP}{LDAP}{Lightweight Directory Access Protocol\protect\glsadd{glos:LDAP}}
\newacronym{CMS}{CMS}{Content-Management-System\protect\glsadd{glos:cms}}

\newglossaryentry{glos:cms}{name=Content-Managment-System,
description={Eine Software zur Verwaltung von Inhalten, in diesem Fall der TU Braunschweig Webseite}
}
% nur über \gls{LDAP} verwenden.
\newglossaryentry{glos:LDAP}{name=Lightweight Directory Access Protocol,
description={LDAP ist ein Verzeichnisdienst, der von der TU Braunschweig für die Benutzerauthentifizierung bereit gestellt wird}
}
\newglossaryentry{glos:https}{name=Hypertext Transfer Protocol Secure,
description={Hypertext Transfer Protocol Secure ist die verschlüsselte Variante
vom Hypertext Transfer Protocol und ist eine zertifikatsbasierende sichere
Übertragungstechnik}
}
\newglossaryentry{glos:sqlite}{name=\mbox{SQLite},
description={SQLite ist eine Relationale Datenbank, die von Python direkt zur Verfügung gestellt wird}
}
\newglossaryentry{glos:copdes}{name=Corporate Design,
description={Das Corporate Design ist das gemeinsame Erscheinungsbild eine Unternehmens. Dies bezieht sich unter anderem auf Kommmunikationsmittel, Werbemittel und Internetauftritte}
}
\newglossaryentry{glos:unicode}{name=Unicode,
description={Der Unicode ist ein Standard, in dem jedes sinntragende Schriftzeichen einem digitalen Code zugeordnet werden soll. Dadurch sollen Kompatibilitätsproblem aufgrund verschiedener Kodierungen in verschiedenen Ländern umgangen werden}
}
\newglossaryentry{glos:thmefi}{name=Thunderbird Message Filter,
description={Der Thunderbird Message Filter ist eine einfache Möglichkeit, E-Mails in dem Programm Mozilla Thunderbird zu durchsuchen}
}
\newglossaryentry{glos:BibTeX}{name=Bib\TeX,
description={Bib\TeX\xspace ist ursprünglich eine Erweiterung für \LaTeX\xspace zur Verwaltung eines Literaturverzeichnisses für wissenschaftliche Publikationen. Inzwischen ist das Format auch unter anderen Textverarbeitungsprogrammen benutzbar}
}
\newglossaryentry{glos:Allegro}{name=Allegro,
description={Allegro ist das Datenbanksystem, welches in der Universitätsbibliothek der TU Braunschweig entwickelt und verwendet wird}
}
\newglossaryentry{glos:regex}{name=Reguläre Ausdrücke,
description={Reguläre Ausdrücke sind ein Mittel der Theoretischen Informatik um Sprachen (dt. Wortmengen) zu beschreiben. In der angewandten Informatik werden diese Ausdrücke noch heute oft verwendet. Mit Regulären Ausdrücken ist eine präzise Beschreibung eines Suchwortes möglich}
}
\newglossaryentry{glos:ext}{name=Externe,
description={Der Begriff \emph{Externe} beschreibt alle Personen, die nicht am System angemeldet sind, also \mbox{z.\,B.}\xspace Mitarbeiter anderer Institute mit einem Ausleihwunsch}
}

