\section{Analyse von Qualitätsmerkmal /Q11/ (SQL-Injections vermeiden)}

SQL-Statements werden ausschließlich durch Übergabe von Parametern an 
Funktionen von Django ausgeführt. SQL-Injections sind in Django durch 
integriertes Escaping von SQL-Syntax nicht möglich. Nur wenn RAW-SQL 
ausgeführt werden soll, oder aber an die Funktion .extra von Django 
übergeben werden soll, ist eigenes Escaping notwendig.

\section{Analyse von Qualitätsmerkmal /Q20/ (Layoutstruktur)}

Die Templates für Django müssen so angepasst werden, dass sie dem
\Gls{glos:copdes}  der TU Braunschweig entsprechen. Dies wird durch Verwendung
ensprechender Grafiken, Cascading-Stylesheets, Webseitenstruktur und weiteren
Mitteln erreicht.


\section{Analyse von Qualitätsmerkmal /Q21/ (Klare Struktur)}

Die Stylesheets für die Templates von Django sowie die Templates selber
müssen so konstruiert werden, dass der auf der Webseite dargestellte 
Inhalt übersichtlich ist. Menüführung soll ebenfalls ohne Verschachtelung
möglich sein. Ein entsprechendes Design muss vorher entworfen werden.


\section{Analyse von Qualitätsmerkmal /Q22/ (Suche)}

Verwendet wird die in Django bereits implementierte Suche. Diese Suche ist ohne
große Anpassungen bereits intuitiv und einfach nutzbar.


\section{Analyse von Qualitätsmerkmal /Q30/ (Zeichenketten)} 

Die Bearbeitung von Zeichenketten als \Gls{glos:unicode} wird bereits von der benutzten
Datenbank \Gls{glos:sqlite} erfordert, welche nur \Gls{glos:unicode}-codierte Zeichenketten benutzt.


\section{Analyse von Qualitätsmerkmal /Q31/ (Löschen von Dokumenten)} 

Löschen von Dokumenten ist mit Django möglich. Intern muss verhindert werden,
dass andere Nutzer als Administratoren löschen dürfen.
