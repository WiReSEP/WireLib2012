\section{Analyse von Qualitätsmerkmal /Q11/ (SQL-Injections vermeiden)}
SQL-Statements werden ausschließlich durch Übergabe von Parametern an Funktionen von Django ausgeführt. SQL-Injections sind in Django nicht möglich.
\section{Analyse von Qualitätsmerkmal /Q20/ (Layoutstruktur)}
Die Templates für Django müssen so angepasst werden, dass sie dem Corporate Design der TU Braunschweig entsprechen. Dies wird durch Verwendung ensprechender Grafiken erreicht.
\section{Analyse von Qualitätsmerkmal /Q21/ (Klare Struktur)}
Die Stylesheets für die Templates von Django müssen auf Übersichtlichkeit ausgelegt werden. Ein entsprechendes Design muss vorher entworfen werden.
\section{Analyse von Qualitätsmerkmal /Q22/ (Suche)}
Verwendet wird die in Django bereits implementierte Suche. Diese Suche ist ohne große Anpassungen bereits intuitiv und einfach nutzbar.
\section{Analyse von Qualitätsmerkmal /Q30/ (Zeichenketten)}
Die Bearbeitung von Zeichenketten als Unicode wird bereits von der benutzten Datenbank SQLite erfordert, welche nur Unicode-codierte Zeichenketten benutzt.
\section{Analyse von Qualitätsmerkmal /Q31/ (Löschen von Dokumenten)}
Löschen von Dokumenten ist mit Django möglich. Intern muss verhindert werden, dass andere Nutzer als Administratoren löschen dürfen.
