%Diese Datei dient der Versionskontrolle. Sie ist vollständig zu bearbeiten.

%----Überschrift------------------------------------------------------------
{\relsize{2}\textbf{Versionsübersicht}}\\[2ex]

%----Start der Tabelle------------------------------------------------------
\begin{longtable}{|m{1.78cm}|m{1.59cm}|m{2.86cm}|m{1.9cm}|m{5.25cm}|}

  \hline                                              % Linie oberhalb

  %----Spaltenüberschriften------------------------------------------------
  \textbf{Version}  &    \textbf{Datum}  &    \textbf{Autor}  &
  \textbf{Status}   &    \textbf{Kommentar}       \\  %Spaltenüberschrift
  \hline                                              % Gitterlinie

  %----die nachfolgeden beiden Zeilen so oft wiederholen und die ... mit den
  %    entsprechenden Daten zu füllen wie erforderlich
  0.1   &    16.05.2012    &    Theodor \mbox{van Nahl}    &   in Bearbeitung     &    Analyse der Produktfunktionen F100-F200\\       % Eintrag in Zeile
  \hline                                              % Gitterlinie unten
  0.2   &    17.05.2012    &    Jörn Hameyer    &   in Bearbeitung     &    Analyse der Produktfunktionen F226-F229\\       % Eintrag in Zeile
  \hline
  0.3   &    18.05.2012    &    Stephan Sobol    &   in Bearbeitung     &    Analyse der Produktfunktionen F220-F225\\       % Eintrag in Zeile
  \hline
  0.4   &    18.05.2012    &    Philipp \mbox{Offensand}    &   in Bearbeitung     &    Qualitätsanalyse\\       % Eintrag in Zeile
  \hline
  0.5   &    18.05.2012    &    Markus Dietrich    &   in Bearbeitung     &    Analyse der Produktfunktionen F230-F302\\       % Eintrag in Zeile
  \hline
  0.6   &    18.05.2012    &    Johann Hong    &   in Bearbeitung     &    Analyse der Produktfunktionen F210-F214\\       % Eintrag in Zeile
  \hline
  0.7   &    18.05.2012    &    Marco Melzer    &   in Bearbeitung     &    Einleitung\\       % Eintrag in Zeile
  \hline
  0.8   &    18.05.2012    &    Philipp \mbox{Offensand}    &   in Bearbeitung     &    Komponentenspezifikation\\       % Eintrag in Zeile
  \hline
  0.9   &    19.05.2012    &    Theodor \mbox{van Nahl}    &   in Bearbeitung     &    Analyse der Produktfunktion F201-F210\\       % Eintrag in Zeile
  \hline
  0.10   &   21.05.2012    &    Marco Melzer    &   in Bearbeitung     &    Schnittstellenspezifikation\\       % Eintrag in Zeile
  \hline
  0.11   &   22.05.2012    &    Johann Hong    &   in Bearbeitung     &    Verteilungsentwurf\\       % Eintrag in Zeile
  \hline
  0.12   &   23.05.2012    &    Jörn Hameyer,\newline Stephan Sobol    &   in Bearbeitung     &    Protokolle für die Benutzung der Komponenten\\    
  \hline
  0.13   &   23.05.2012    &    Eric Anders &   in Bearbeitung  &   Glossar und Rechtschreibkontrolle \\
     % Eintrag in Zeile
  \hline
  1.0   &    23.05.2012    &    sämtliche Auftragnehmer    &   abgenommen     &    \\       % Eintrag in Zeile
  \hline
  

%----Ende der Tabelle------------------------------------------------------
\end{longtable}

%Status: "`in Bearbeitung"' oder "`abgenommen"'
%Kommentar: hier eintragen, was ge\"andert bzw. ergänzt wurde


%Hinweis zum Template:
%Dieses Template enth\"ult Hinweise, die alle kursiv geschrieben sind. Alles
%Kursivgeschriebene ist selbstverst\"andlich bei Abgabe zu entfernen sind.
%Angaben in <\ldots> sind mit dem entsprechendem Text zu f\"ullen.  \"Uberz\"ahlige
%Kapitel, d.h. Kapitel, die nicht bearbeitet werden m\"ussen, da sie nicht der
%Aufgabenstellung entsprechen, bitte entfernen.

%Aufgabe des Grobentwurfs: Aufgabe dieses Dokumentes ist es, die Architektur des
%Systems zu beschreiben und die daraus resultierenden Pakete durch die
%Definition von Schnittstellen zu Komponenten auszubauen.
