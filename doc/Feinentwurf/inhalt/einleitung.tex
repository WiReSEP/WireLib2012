% Kapitel 1
%-------------------------------------------------------------------------------
\chapter{Einleitung}
%Hier Einleitungstext einfügen, dabei die Formatierungen selber erstellen

In diesem Feinentwurf werden die thematisierten Inhalte im Pflichtenheft und im Grobentwurf, die genaue Konstruktion eines Bibliothekmanagementsystems (WireLib), weiter konkretisiert. 
Dies erfolgt mit dem Ziel, jedem außenstehenden Softwareentwickler zu ermöglichen, die Problematik besser zu verstehen und mithilfe des Dokumentes das Projekt technisch umzusetzen.\\
Somit stehen programmspezifische Details im Vordergrund. \\
Am Anfang werden die Grundbedingungen des zu implementierenden Systems, bestehend aus den Musskriterien und den Wunschkriterien, vorgestellt.
Darauf folgt als wichtiges Element der Implementierungsentwurf: \\
Es behandelt die Dokumentation von verwendeten Klassen und Bibliotheken mit ihren entsprechenden Attributen und Methoden, unter anderem mithilfe von Klassendiagrammen sowie anderen UML Diagrammen.\\ 
Das zu verwendende Datenmodell wird im nächsten Kapitel explizit beschrieben: \\
Alle benutzten Entities und ihre Beziehungen zueinander werden näher erläutert, und sollen ein besseres Grundverständnis der Datenbank übermitteln.
Danach folgen die Betriebsanleitungen für die Serverkonfiguration.    
