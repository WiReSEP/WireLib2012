% Kapitel 4 mit den entsprechenden Unterkapiteln
% Die Unterkapitel können auch in separaten Dateien stehen,
% die dann mit dem \include-Befehl eingebunden werden.
%-------------------------------------------------------------------------------
\chapter{Datenmodell}
Falls in der Anwendung bestimmte Daten dauerhaft gespeichert werden, so sind
die entsprechenden Entities und Beziehungen hier darzustellen und zu erl\"autern.
Dies ist insbesondere relevant, falls der Einsatz einer (relationalen)
Datenbank geplant ist.

\section{Diagramm}

Eigenes Klassendiagramm einsetzen
\section{Erl\"auterung}
Versuch 1: \\
\begin{tabular}[ht]{|l||c|c|}
  \hline
  Entit\"at & \multicolumn{2}{c|}{Beziehungen} \\
  \hline\hline\hline
  
  Autoren:  & Name der Beziehung &  Kardinalit\"at\\
  \hline\hline
  documents\_author & hat geschrieben & 1,n \\
  \hline\hline\hline
  
  Autor\_Dokument: & Name der Beziehung & Kardinalität\\
  \hline\hline
  documents\_documents\_authors & wurde geschrieben von & 1\\
  \hline
  documents\_documents\_authors & hat geschrieben & 1\\
  \hline\hline\hline
  
  Dokument: & Name der Beziehung & Kardinalität\\
  \hline\hline
  documents\_document & wurde geschrieben von & 1,n\\
  \hline
  documents\_document & gehört zur Kategorie & 1\\
  \hline
  documents\_document & wurde veröffentlicht von & 1\\
  \hline
  documents\_document & besitzt die Schlüsselwörter & n\\  
  \hline
  documents\_document & besitzt die Extrafelder & n\\
  \hline
  documents\_document & hat den Status/verliehen an & n\\
  \hline\hline\hline
  
  Schlüsselwörter:  & Name der Beziehung &  Kardinalit\"at\\
  \hline\hline
  documents\_keywords & gehört zum Dokument & 1 \\
  \hline\hline\hline
  
  Publisher:  & Name der Beziehung &  Kardinalit\"at\\
  \hline\hline
  documents\_publisher & hat veröffentlicht & 1,n \\
  \hline\hline\hline
  
  Kategorie:  & Name der Beziehung &  Kardinalit\"at\\
  \hline\hline
  documents\_category & enthält & n \\
  \hline\hline\hline
  
  Extrafelder:  & Name der Beziehung &  Kardinalit\"at\\
  \hline\hline
  documents\_doc\_extra & gehört zu & 1 \\
  \hline\hline\hline
  
  Status: & Name der Beziehung & Kardinalität\\
  \hline\hline
  documents\_doc\_status & gehört zum Dokument & 1\\
  \hline
  documents\_doc\_status & wurde geändert von & 1\\
  \hline
  documents\_doc\_status & wurde ausgeliehen von & 0,1\\
  \hline
  documents\_doc\_status & wurde weitergegeben an & 0,1\\
  \hline\hline\hline
  
  Non\_User: & Name der Beziehung & Kardinalität\\
  \hline\hline
  documents\_non\_user & entlieh & 1,n\\
  \hline
  documents\_non\_user & hat die Telefonnummern & 1,n\\
  \hline\hline\hline
  
  Telefonnummer von Non\_Usern:  & Name der Beziehung &  Kardinalit\"at\\
  \hline\hline
  documents\_tel\_non\_user & gehört zu & 1 \\
  \hline
\end{tabular}

\begin{tabular}[ht]{|l||c|c|}
  \hline
  Entit\"at & \multicolumn{2}{c|}{Beziehungen} \\
  \hline\hline\hline
    
  User: & Name der Beziehung & Kardinalität\\
  \hline\hline
  auth\_user & wohnt & 1\\
  \hline
  auth\_user & hat die Telefonnumern & 1,n\\
  \hline
  auth\_user & hat die Rechte & n\\
  \hline
  auth\_user & änderte den Status & n\\  
  \hline
  auth\_user & leiht/bürgt gerade für & n\\
  \hline
  auth\_user & TODO django\_admin\_log & n\\
  \hline
  auth\_user & bekam & n\\
  \hline\hline\hline
  
  Telefonnummer von Usern:  & Name der Beziehung &  Kardinalit\"at\\
  \hline\hline
  documents\_tel\_user & gehört zu & 1 \\
  \hline\hline\hline 
  
  Message: & Name der Beziehung &  Kardinalit\"at\\
  \hline\hline
  auth\_message & gesendet an & 1 \\
  \hline\hline\hline 
  
  Anschrift:  & Name der Beziehung &  Kardinalit\"at\\
  \hline\hline
  documents\_user\_profile & Anschrift von & 1 \\
  \hline\hline\hline
  
  User\_Gruppe:  & Name der Beziehung &  Kardinalit\"at\\
  \hline\hline
  auth\_user\_groups & der User & 1 \\
  \hline
  auth\_user\_groups & gehört zur Gruppe & 1 \\
  \hline\hline\hline
  
  Gruppe: & Name der Beziehung &  Kardinalit\"at\\
  \hline\hline
  auth\_group & besitzt die Mitglieder & n \\
  \hline
  auth\_group & hat die Rechte & n \\
  \hline\hline\hline 

  Gruppe\_Rechte: & Name der Beziehung &  Kardinalit\"at\\
  \hline\hline
  auth\_group\_permissions & die Gruppe & 1 \\
  \hline
  auth\_group\_permissions & hat das Recht & 1 \\
  \hline\hline\hline 
  
  User\_Rechte:  & Name der Beziehung &  Kardinalit\"at\\
  \hline\hline
  auth\_user\_user\_permissions & der User & 1 \\
  \hline
  auth\_user\_user\_permissions & hat das Recht & 1 \\
  \hline\hline\hline 
  
  Rechte:  & Name der Beziehung &  Kardinalit\"at\\
  \hline\hline
  auth\_permission & der User hat  & n \\
  \hline
  auth\_permission & die Gruppe hat & n \\
  \hline
  auth\_permission & TODO django\_content\_type & 1 \\
  \hline
\end{tabular}

\begin{tabular}[ht]{|l||c|c|}
  \hline
  Entit\"at & \multicolumn{2}{c|}{Beziehungen} \\
  \hline\hline\hline
  
  Adminlog:  & Name der Beziehung &  Kardinalit\"at\\
  \hline\hline
  django\_admin\_log & getätigt von  & 1 \\
  \hline
  auth\_admin\_log & TODO django\_content\_type & 1 \\
  \hline\hline\hline
  
  Content\_Type:  & Name der Beziehung &  Kardinalit\"at\\
  \hline\hline
  django\_content\_type & TODO auth\_permission & n \\
  \hline
  django\_content\_type & TODO django\_admin\_log & n \\
  \hline\hline\hline
\end{tabular}


Versuch 2: \\
\begin{tabular}[ht]{|l||c|c|}
  \hline
  Entit\"at & \multicolumn{2}{c|}{Beziehungen} \\
  \hline\hline\hline
  
  documents\_author  & Name der Beziehung &  Kardinalit\"at\\
  \hline\hline
  documents\_documents\_authors & hat geschrieben & 1,n \\
  \hline\hline\hline
  
  documents\_documents\_authors & Name der Beziehung & Kardinalität\\
  \hline\hline
  documents\_author & wurde geschrieben von & 1,n\\
  \hline
  documents\_document & hat geschrieben & 1,n\\
  \hline\hline\hline
  
  documents\_document: & Name der Beziehung & Kardinalität\\
  \hline\hline
  documents\_documents\_authors & wurde geschrieben von & 1,n\\
  \hline
  documents\_category & gehört zur Kategorie & 1\\
  \hline
  documents\_publisher & wurde veröffentlicht von & 1\\
  \hline
  documents\_keywords & besitzt die Schlüsselwörter & n\\  
  \hline
  documents\_doc\_extra & besitzt die Extrafelder & n\\
  \hline
  documents\_doc\_status & hat den Status/verliehen an & n\\
  \hline\hline\hline
  
  documents\_keywords:  & Name der Beziehung &  Kardinalit\"at\\
  \hline\hline
  documents\_document & gehört zum Dokument & 1 \\
  \hline\hline\hline
  
  documents\_publisher:  & Name der Beziehung &  Kardinalit\"at\\
  \hline\hline
  documents\_document & hat veröffentlicht & 1,n \\
  \hline\hline\hline
  
  documents\_category:  & Name der Beziehung &  Kardinalit\"at\\
  \hline\hline
  documents\_document & enthält & n \\
  \hline\hline\hline
  
  documents\_doc\_extra:  & Name der Beziehung &  Kardinalit\"at\\
  \hline\hline
  documents\_document & gehört zu & 1 \\
  \hline\hline\hline
  
  documents\_doc\_status: & Name der Beziehung & Kardinalität\\
  \hline\hline
  documents\_document & gehört zum Dokument & 1\\
  \hline
  auth\_user & wurde geändert von & 1\\
  \hline
  auth\_user & wurde ausgeliehen von & 0,1\\
  \hline
  documents\_non\_user & wurde weitergegeben an & 0,1\\
  \hline\hline\hline
  
  documents\_non\_user: & Name der Beziehung & Kardinalität\\
  \hline\hline
  documents\_doc\_status & entlieh & 1,n\\
  \hline
  documents\_tel\_non\_user & hat die Telefonnummern & 1,n\\
  \hline\hline\hline
  
  documents\_tel\_non\_user:  & Name der Beziehung &  Kardinalit\"at\\
  \hline\hline
  documents\_non\_user & gehört zu & 1 \\
  \hline
\end{tabular}

\begin{tabular}[ht]{|l||c|c|}
  \hline
  Entit\"at & \multicolumn{2}{c|}{Beziehungen} \\
  \hline\hline\hline
    
  auth\_user: & Name der Beziehung & Kardinalität\\
  \hline\hline
  documents\_user\_profile & wohnt & 1\\
  \hline
  documents\_tel\_user & hat die Telefonnumern & 1,n\\
  \hline
  auth\_user\_user\_permissions & hat die Rechte & n\\
  \hline
  documents\_doc\_status & änderte den Status & n\\  
  \hline
  documents\_doc\_status & leiht/bürgt gerade für & n\\
  \hline
  django\_admin\_log & TODO django\_admin\_log & n\\
  \hline
  auth\_message & bekam & n\\
  \hline\hline\hline
  
  documents\_tel\_user:  & Name der Beziehung &  Kardinalit\"at\\
  \hline\hline
  auth\_user & gehört zu & 1 \\
  \hline\hline\hline 
  
  auth\_message: & Name der Beziehung &  Kardinalit\"at\\
  \hline\hline
  auth\_user & gesendet an & 1 \\
  \hline\hline\hline 
  
  documents\_user\_profile:  & Name der Beziehung &  Kardinalit\"at\\
  \hline\hline
  auth\_user & Anschrift von & 1 \\
  \hline\hline\hline
  
  auth\_user\_groups:  & Name der Beziehung &  Kardinalit\"at\\
  \hline\hline
  auth\_user & der User & 1 \\
  \hline
  auth\_group & gehört zur Gruppe & 1 \\
  \hline\hline\hline
  
  auth\_group: & Name der Beziehung &  Kardinalit\"at\\
  \hline\hline
  auth\_user\_groups & besitzt die Mitglieder & n \\
  \hline
  auth\_group\_permissions & hat die Rechte & n \\
  \hline\hline\hline 

  auth\_group\_permissions: & Name der Beziehung &  Kardinalit\"at\\
  \hline\hline
  auth\_group & die Gruppe & 1 \\
  \hline
  auth\_permission & hat das Recht & 1 \\
  \hline\hline\hline 
  
  auth\_user\_user\_permissions:  & Name der Beziehung &  Kardinalit\"at\\
  \hline\hline
  auth\_user & der User & 1 \\
  \hline
  auth\_permission & hat das Recht & 1 \\
  \hline\hline\hline 
  
  auth\_permission:  & Name der Beziehung &  Kardinalit\"at\\
  \hline\hline
  auth\_user\_user\_permissions & der User hat  & n \\
  \hline
  auth\_group\_permissions & die Gruppe hat & n \\
  \hline
  django\_content\_type & TODO & 1 \\
  \hline
\end{tabular}

\begin{tabular}[ht]{|l||c|c|}
  \hline
  Entit\"at & \multicolumn{2}{c|}{Beziehungen} \\
  \hline\hline\hline
  
  django\_admin\_log:  & Name der Beziehung &  Kardinalit\"at\\
  \hline\hline
  auth\_user & getätigt von  & 1 \\
  \hline
  django\_content\_type & TODO & 1 \\
  \hline\hline\hline
  
  django\_content\_type:  & Name der Beziehung &  Kardinalit\"at\\
  \hline\hline
  auth\_permission & TODO  & n \\
  \hline
  django\_admin\_log & TODO & n \\
  \hline\hline\hline
\end{tabular}