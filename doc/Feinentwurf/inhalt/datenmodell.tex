% Kapitel 4 mit den entsprechenden Unterkapiteln
% Die Unterkapitel können auch in separaten Dateien stehen,
% die dann mit dem \include-Befehl eingebunden werden.
%-------------------------------------------------------------------------------
\chapter{Datenmodell}
\label{kap4}
Falls in der Anwendung bestimmte Daten dauerhaft gespeichert werden, so sind
die entsprechenden Entities und Beziehungen hier darzustellen und zu erläutern.
Dies ist insbesondere relevant, falls der Einsatz einer (relationalen)
Datenbank geplant ist.

\section{Diagramm}


\begin{figure}[H]
\includegraphics[width=1.0\linewidth]{bilder/db_wirelib-packages.pdf}
\caption{Datenbankmodell}
\label{fig:DBDiagramm}
\end{figure}

Das Diagramm gibt eine grobe Übersicht über die Struktur der zugrundeliegenden
relationalen Datenbank. Im Folgenden werden die beiden wichtigsten
Teilstrukturen der Datenbank für bessere Übersicht dargestellt.


\begin{figure}[H]
\includegraphics[width=1.0\linewidth]{bilder/database-wirelib_cluster-doc.pdf}
\caption{Datenbankmodell: Bibliothek}
\label{fig:DB_docDiagramm}
\end{figure}

Dieses Diagramm gibt eine Übersicht über alle Tabellen der Datenbank, die im
Zusammenhang mit der Bibliothek stehen. Die wichtigste Tabelle ist document, in
der die vorhandenen Dokumente eingetragen werden. In der Tabelle doc\_status
werden alle Statusänderungen eines Dokumentes festgehalten, einschließlich der
Ausleihe eines Dokumentes. Die Tabelle doc\_extra speichert zusätzliche Inhalte
eines Dokumentes, die aufgrund seltenen Gebrauchs nicht lohnen, in document
direkt aufgenommen zu werden. In der Tabelle emails werden die E-Mail-Templates
gespeichert, die verschickt werden, wenn die Ausleihfrist eines Dokumentes
abläuft oder in ähnlichen Fällen. Die Tabellen auth\_user und non\_user sind
in der Teilstruktur Benutzer und hier nur als Referenz gedacht zur besseren
Übersicht. Die verbleibenden Tabellen speichern trivialerweise den
beschriebenen Inhalt.


\begin{figure}[H]
\includegraphics[width=1.0\linewidth]{bilder/database-wirelib_cluster-user.pdf}
\caption{Datenbankmodell: Benutzer}
\label{fig:DB_UserDiagramm}
\end{figure}

Dieses Diagramm gibt eine Übersicht über alle Tabellen der Datenbank, die im
Zusammenhang mit Benutzerverwaltung stehen. Alle Tabellen, die in diesem Modell
mit dem Präfix \emph{auth\_} beginnen, sind von Django generiert und werden für
eine erleichterte Handhabung von Benutzern und Rechtevergabe genutzt. Tabellen
mit dem Präfix \emph{django\_} werden ebenfalls von Django generiert und dienen
dem Admin-Interface von Django und dem Session-Managment von eingeloggten
Benutzern. Die Tabelle non\_user ist für Nutzer gedacht, die nicht am Institut
für Wissenschaftliches Rechnen arbeiten und trotzdem ein Dokument ausleihen
wollen. Sie haben so keine Möglichkeit, sich auf der Webseite einzuloggen,
können aber trotzdem Dokumente ausleihen. Die Tabellen tel\_user und
tel\_non\_user werden benötigt, um Telefonnumern zu Benutzern zu speichern.
user\_profile ist zur Erweiterung der Django-Tabelle auth\_user, um auch
Adressen speichern zu können.

% Eigenes Klassendiagramm einsetzen
\section{Erläuterung}
Im Folgenden sind alle Relationen von einer Tabelle zu einer anderen 
in der Datenbank angezeigt. Dabei ist jeweils oben die ausgehende Tabelle und 
unter \glqq Relation zu\grqq\ alle zu denen eine Verbindung existiert. Unter 
\glqq Kardinalität\grqq\ steht dann, mit wie vielen Datensätzen der 
Relationstabelle ein Datensatz der Ausgangstabelle verbunden ist. Angegeben wird 
dabei zuerst die Mindestanzahl von Verbindungen und nach dem Komma die 
Maximalanzahl. Bei \glqq 1,n\grqq\ hat Datensatz $A$ also mindestens eine und 
maximal unendlich viele Verbindungen zur Relationstabelle.

\begin{longtable}{@{}ccc@{}}
  \toprule
  \multicolumn{3}{c}{\emph{Tabelle:} author} \\
  Relation zu & Name der Beziehung & Kardinalität \\
  \cmidrule(lr){1-1}\cmidrule(lr){2-2}\cmidrule(lr){3-3}
  document\_authors & hat geschrieben & 1,n \\
  
  \toprule
  \multicolumn{3}{c}{\emph{Tabelle:} document\_authors} \\
  Relation zu & Name der Beziehung & Kardinalität \\
  \cmidrule(lr){1-1}\cmidrule(lr){2-2}\cmidrule(lr){3-3}
  author & wurde geschrieben von & 1,1 \\
  document & hat geschrieben & 1,1\\

  \toprule
  \multicolumn{3}{c}{\emph{Tabelle:} document} \\
  Relation zu & Name der Beziehung & Kardinalität \\
  \cmidrule(lr){1-1}\cmidrule(lr){2-2}\cmidrule(lr){3-3}
  document\_authors & wurde geschrieben von & 1,n\\
  category & gehört zur Kategorie & 1,1\\
  publisher & wurde veröffentlicht von & 1,1\\
  keywords & besitzt die Schlüsselwörter & 0,n\\  
  doc\_extra & besitzt die Extrafelder & 0,n\\
  doc\_status & hat den Status/verliehen an & 0,n\\

  \toprule
  \multicolumn{3}{c}{\emph{Tabelle:} keywords} \\
  Relation zu & Name der Beziehung & Kardinalität \\
  \cmidrule(lr){1-1}\cmidrule(lr){2-2}\cmidrule(lr){3-3}
  document & gehört zum Dokument & 1,1 \\

  \toprule
  \multicolumn{3}{c}{\emph{Tabelle:} publisher} \\
  Relation zu & Name der Beziehung & Kardinalität \\
  \cmidrule(lr){1-1}\cmidrule(lr){2-2}\cmidrule(lr){3-3}
  document & hat veröffentlicht & 1,n \\

  \toprule
  \multicolumn{3}{c}{\emph{Tabelle:} category} \\
  Relation zu & Name der Beziehung & Kardinalität \\
  \cmidrule(lr){1-1}\cmidrule(lr){2-2}\cmidrule(lr){3-3}
  document & enthält & 0,n \\

  \toprule
  \multicolumn{3}{c}{\emph{Tabelle:} doc\_extra} \\
  Relation zu & Name der Beziehung & Kardinalität \\
  \cmidrule(lr){1-1}\cmidrule(lr){2-2}\cmidrule(lr){3-3}
  document & gehört zu & 1,1 \\

  \toprule
  \multicolumn{3}{c}{\emph{Tabelle:} doc\_status} \\
  Relation zu & Name der Beziehung & Kardinalität \\
  \cmidrule(lr){1-1}\cmidrule(lr){2-2}\cmidrule(lr){3-3}
  document & gehört zum Dokument & 1,1\\
  auth\_user & wurde geändert von & 1,1\\
  auth\_user & wurde ausgeliehen von & 0,1\\
  non\_user & wurde weitergegeben an & 0,1\\

  \toprule
  \multicolumn{3}{c}{\emph{Tabelle:} non\_user} \\
  Relation zu & Name der Beziehung & Kardinalität \\
  \cmidrule(lr){1-1}\cmidrule(lr){2-2}\cmidrule(lr){3-3}
  doc\_status & entlieh & 1,n\\
  tel\_non\_user & hat die Telefonnummern & 1,n\\

  \toprule
  \multicolumn{3}{c}{\emph{Tabelle:} tel\_non\_user} \\
  Relation zu & Name der Beziehung & Kardinalität \\
  \cmidrule(lr){1-1}\cmidrule(lr){2-2}\cmidrule(lr){3-3}
  non\_user & gehört zu & 1,1 \\

  \toprule
  \multicolumn{3}{c}{\emph{Tabelle:} auth\_user} \\
  Relation zu & Name der Beziehung & Kardinalität \\
  \cmidrule(lr){1-1}\cmidrule(lr){2-2}\cmidrule(lr){3-3}
  user\_profile & wohnt & 1,1\\
  tel\_user & hat die Telefonnumern & 1,n\\
  auth\_user\_user\_permissions & hat die Rechte & 0,n\\
  doc\_status & änderte den Status & 0,n\\  
  doc\_status & leiht/bürgt gerade für & 0,n\\
  django\_admin\_log & aktualisiert & 0,n\\
  auth\_message & bekam & 0,n\\
  auth\_user\_groups & ist in Gruppe & 0,n\\

  \toprule
  \multicolumn{3}{c}{\emph{Tabelle:} tel\_user} \\
  Relation zu & Name der Beziehung & Kardinalität \\
  \cmidrule(lr){1-1}\cmidrule(lr){2-2}\cmidrule(lr){3-3}
  auth\_user & gehört zu & 1,1 \\

  \toprule
  \multicolumn{3}{c}{\emph{Tabelle:} auth\_message} \\
  Relation zu & Name der Beziehung & Kardinalität \\
  \cmidrule(lr){1-1}\cmidrule(lr){2-2}\cmidrule(lr){3-3}
  auth\_user & gesendet an & 1,1 \\

  \toprule
  \multicolumn{3}{c}{\emph{Tabelle:} user\_profile} \\
  Relation zu & Name der Beziehung & Kardinalität \\
  \cmidrule(lr){1-1}\cmidrule(lr){2-2}\cmidrule(lr){3-3}
  auth\_user & Anschrift von & 1,1 \\

  \toprule
  \multicolumn{3}{c}{\emph{Tabelle:} auth\_user\_groups} \\
  Relation zu & Name der Beziehung & Kardinalität \\
  \cmidrule(lr){1-1}\cmidrule(lr){2-2}\cmidrule(lr){3-3}
  auth\_user & der User & 1,1 \\
  auth\_group & gehört zur Gruppe & 1,1 \\

  \toprule
  \multicolumn{3}{c}{\emph{Tabelle:} auth\_group} \\
  Relation zu & Name der Beziehung & Kardinalität \\
  \cmidrule(lr){1-1}\cmidrule(lr){2-2}\cmidrule(lr){3-3}
  auth\_user\_groups & besitzt die Mitglieder & 0,n \\
  auth\_group\_permissions & hat die Rechte & 0,n \\

  \toprule
  \multicolumn{3}{c}{\emph{Tabelle:} auth\_group\_permissions} \\
  Relation zu & Name der Beziehung & Kardinalität \\
  \cmidrule(lr){1-1}\cmidrule(lr){2-2}\cmidrule(lr){3-3}
  auth\_group & die Gruppe & 1,1 \\
  auth\_permission & hat das Recht & 1,1 \\

  \toprule
  \multicolumn{3}{c}{\emph{Tabelle:} auth\_user\_user\_permissions} \\
  Relation zu & Name der Beziehung & Kardinalität \\
  \cmidrule(lr){1-1}\cmidrule(lr){2-2}\cmidrule(lr){3-3}
  auth\_user & der User & 1,1 \\
  auth\_permission & hat das Recht & 1,1 \\

  \toprule
  \multicolumn{3}{c}{\emph{Tabelle:} auth\_permission} \\
  Relation zu & Name der Beziehung & Kardinalität \\
  \cmidrule(lr){1-1}\cmidrule(lr){2-2}\cmidrule(lr){3-3}
  auth\_user\_user\_permissions & der User hat  & 0,n \\
  auth\_group\_permissions & die Gruppe hat & 0,n \\
  django\_content\_type & referenziert & 1,1 \\

  \toprule
  \multicolumn{3}{c}{\emph{Tabelle:} django\_admin\_log} \\
  Relation zu & Name der Beziehung & Kardinalität \\
  \cmidrule(lr){1-1}\cmidrule(lr){2-2}\cmidrule(lr){3-3}
  auth\_user & getätigt von  & 1,1 \\
  django\_content\_type & referenziert & 1,1 \\

  \toprule
  \multicolumn{3}{c}{\emph{Tabelle:} django\_content\_type} \\
  Relation zu & Name der Beziehung & Kardinalität \\
  \cmidrule(lr){1-1}\cmidrule(lr){2-2}\cmidrule(lr){3-3}
  auth\_permission & stellt Referenz  & 0,n \\
  django\_admin\_log & ändert & 0,n \\
\end{longtable}
