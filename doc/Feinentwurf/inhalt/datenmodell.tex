% Kapitel 4 mit den entsprechenden Unterkapiteln
% Die Unterkapitel können auch in separaten Dateien stehen,
% die dann mit dem \include-Befehl eingebunden werden.
%-------------------------------------------------------------------------------
\chapter{Datenmodell}
Falls in der Anwendung bestimmte Daten dauerhaft gespeichert werden, so sind
die entsprechenden Entities und Beziehungen hier darzustellen und zu erl\"autern.
Dies ist insbesondere relevant, falls der Einsatz einer (relationalen)
Datenbank geplant ist.

\section{Diagramm}

Eigenes Klassendiagramm einsetzen
\section{Erl\"auterung}
Die Tabelle ist um so viele Eintr\"age zu erweitern, wie es Entities im obigen
Klassendiagramm gibt. F\"ur jede Entity sind so viele Eintr\"age in der
Beziehungs-Subtabelle einzuf\"ugen, wie es Beziehungen zu dieser Entity gibt.


\begin{tabular}[ht]{|l|c|}
  \hline
  Entit\"at & Beziehungen\\
  \hline\hline
  <Entity ... ID>:  & Name der Beziehung |  Kardinalit\"at\\
  \hline\hline\hline
  <Bezeichnung> & <Name der Beziehung> | <Kardinalit\"at>\\
  \hline
\end{tabular}
