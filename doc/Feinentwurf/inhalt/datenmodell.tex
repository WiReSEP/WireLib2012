% Kapitel 4 mit den entsprechenden Unterkapiteln
% Die Unterkapitel können auch in separaten Dateien stehen,
% die dann mit dem \include-Befehl eingebunden werden.
%-------------------------------------------------------------------------------
\chapter{Datenmodell}
Falls in der Anwendung bestimmte Daten dauerhaft gespeichert werden, so sind
die entsprechenden Entities und Beziehungen hier darzustellen und zu erl\"autern.
Dies ist insbesondere relevant, falls der Einsatz einer (relationalen)
Datenbank geplant ist.

\section{Diagramm}

Eigenes Klassendiagramm einsetzen
\section{Erl\"auterung}

\begin{tabular}[ht]{|l||c|c|}
  \hline
  Entit\"at & \multicolumn{2}{c|}{Beziehungen} \\
  \hline\hline\hline
  
  author  & Name der Beziehung &  Kardinalit\"at\\
  \hline\hline
  document\_authors & hat geschrieben & 1,n \\
  \hline\hline\hline
  
  document\_authors & Name der Beziehung & Kardinalität\\
  \hline\hline
  author & wurde geschrieben von & 1\\
  \hline
  document & hat geschrieben & 1\\
  \hline\hline\hline
  
  document: & Name der Beziehung & Kardinalität\\
  \hline\hline
  document\_authors & wurde geschrieben von & 1,n\\
  \hline
  category & gehört zur Kategorie & 1\\
  \hline
  publisher & wurde veröffentlicht von & 1\\
  \hline
  keywords & besitzt die Schlüsselwörter & n\\  
  \hline
  doc\_extra & besitzt die Extrafelder & n\\
  \hline
  doc\_status & hat den Status/verliehen an & n\\
  \hline\hline\hline
  
  keywords:  & Name der Beziehung &  Kardinalit\"at\\
  \hline\hline
  document & gehört zum Dokument & 1 \\
  \hline\hline\hline
  
  publisher:  & Name der Beziehung &  Kardinalit\"at\\
  \hline\hline
  document & hat veröffentlicht & 1,n \\
  \hline\hline\hline
  
  category:  & Name der Beziehung &  Kardinalit\"at\\
  \hline\hline
  document & enthält & n \\
  \hline\hline\hline
  
  doc\_extra:  & Name der Beziehung &  Kardinalit\"at\\
  \hline\hline
  document & gehört zu & 1 \\
  \hline\hline\hline
  
  doc\_status: & Name der Beziehung & Kardinalität\\
  \hline\hline
  document & gehört zum Dokument & 1\\
  \hline
  auth\_user & wurde geändert von & 1\\
  \hline
  auth\_user & wurde ausgeliehen von & 0,1\\
  \hline
  non\_user & wurde weitergegeben an & 0,1\\
  \hline\hline\hline
  
  non\_user: & Name der Beziehung & Kardinalität\\
  \hline\hline
  doc\_status & entlieh & 1,n\\
  \hline
  tel\_non\_user & hat die Telefonnummern & 1,n\\
  \hline\hline\hline
  
  tel\_non\_user:  & Name der Beziehung &  Kardinalit\"at\\
  \hline\hline
  non\_user & gehört zu & 1 \\
  \hline
\end{tabular}

\begin{tabular}[ht]{|l||c|c|}
  \hline
  Entit\"at & \multicolumn{2}{c|}{Beziehungen} \\
  \hline\hline\hline
    
  auth\_user: & Name der Beziehung & Kardinalität\\
  \hline\hline
  user\_profile & wohnt & 1\\
  \hline
  tel\_user & hat die Telefonnumern & 1,n\\
  \hline
  auth\_user\_user\_permissions & hat die Rechte & n\\
  \hline
  doc\_status & änderte den Status & n\\  
  \hline
  doc\_status & leiht/bürgt gerade für & n\\
  \hline
  django\_admin\_log & aktualisiert & n\\
  \hline
  auth\_message & bekam & n\\
  \hline\hline\hline
  
  tel\_user:  & Name der Beziehung &  Kardinalit\"at\\
  \hline\hline
  auth\_user & gehört zu & 1 \\
  \hline\hline\hline 
  
  auth\_message: & Name der Beziehung &  Kardinalit\"at\\
  \hline\hline
  auth\_user & gesendet an & 1 \\
  \hline\hline\hline 
  
  user\_profile:  & Name der Beziehung &  Kardinalit\"at\\
  \hline\hline
  auth\_user & Anschrift von & 1 \\
  \hline\hline\hline
  
  auth\_user\_groups:  & Name der Beziehung &  Kardinalit\"at\\
  \hline\hline
  auth\_user & der User & 1 \\
  \hline
  auth\_group & gehört zur Gruppe & 1 \\
  \hline\hline\hline
  
  auth\_group: & Name der Beziehung &  Kardinalit\"at\\
  \hline\hline
  auth\_user\_groups & besitzt die Mitglieder & n \\
  \hline
  auth\_group\_permissions & hat die Rechte & n \\
  \hline\hline\hline 

  auth\_group\_permissions: & Name der Beziehung &  Kardinalit\"at\\
  \hline\hline
  auth\_group & die Gruppe & 1 \\
  \hline
  auth\_permission & hat das Recht & 1 \\
  \hline\hline\hline 
  
  auth\_user\_user\_permissions:  & Name der Beziehung &  Kardinalit\"at\\
  \hline\hline
  auth\_user & der User & 1 \\
  \hline
  auth\_permission & hat das Recht & 1 \\
  \hline\hline\hline 
  
  auth\_permission:  & Name der Beziehung &  Kardinalit\"at\\
  \hline\hline
  auth\_user\_user\_permissions & der User hat  & n \\
  \hline
  auth\_group\_permissions & die Gruppe hat & n \\
  \hline
  django\_content\_type & referenziert & 1 \\
  \hline
\end{tabular}

\begin{tabular}[ht]{|l||c|c|}
  \hline
  Entit\"at & \multicolumn{2}{c|}{Beziehungen} \\
  \hline\hline\hline
  
  django\_admin\_log:  & Name der Beziehung &  Kardinalit\"at\\
  \hline\hline
  auth\_user & getätigt von  & 1 \\
  \hline
  django\_content\_type & referenziert & 1 \\
  \hline\hline\hline
  
  django\_content\_type:  & Name der Beziehung &  Kardinalit\"at\\
  \hline\hline
  auth\_permission & stellt Referenz  & n \\
  \hline
  django\_admin\_log & ändert & n \\
  \hline\hline\hline
\end{tabular}
