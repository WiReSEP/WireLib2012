%Diese Datei dient der Versionskontrolle. Sie ist vollständig zu bearbeiten.

%----Überschrift----------------------------------------------------------------
{\relsize{2}\textbf{Versionsübersicht}}\\[2ex]

%----Start der Tabelle----------------------------------------------------------
\begin{longtable}{|m{1.78cm}|m{1.59cm}|m{2.86cm}|m{1.9cm}|m{5.25cm}|}

  \hline                                              % Linie oberhalb

  %----Spaltenüberschriften-----------------------------------------------------
  \textbf{Version}  &    \textbf{Datum}  &    \textbf{Autor}  &
  \textbf{Status}   &    \textbf{Kommentar}       \\  %Spaltenüberschrift
  \hline                                              % Gitterlinie


  %----die nachfolgenden beiden Zeilen so oft wiederholen und die ... mit den
  %    entsprechenden Daten zu füllen wie erforderlich
  0.1    &    25.06.2012    &    Stephan Sobol,\newline Phiipp \mbox{Offensand}    &    
  in Bearbeitung    &  Kapitel 4   \\      % Eintrag in Zeile
  \hline                                                     % Gitterlinie unten
  0.2    &    25.06.2012    &    Markus Dietrich,\newline Jörn Hameyer    &    
  in Bearbeitung    &    Kapitel 3\\      % Eintrag in Zeile
  \hline                            % Gitterlinie unten
  0.3    &    25.06.2012    &    Johann Hong    &    
  in Bearbeitung    &    Kapitel 1\\      % Eintrag in Zeile
  \hline                                                     % Gitterlinie unten
  0.4    &    25.06.2012    &    Theodor \mbox{van Nahl}    &    
  in Bearbeitung    &    Kapitel 5\\      % Eintrag in Zeile
  \hline                                                     % Gitterlinie unten
  0.5    &    26.06.2012    &    Marco Melzer    &    
  in Bearbeitung    &    Kapitel 2\\      % Eintrag in Zeile
  \hline                                                     % Gitterlinie unten
  0.6    &    27.06.2012    &    Eric Anders    &    
  in Bearbeitung    &    Glossar und Rechtschreibprüfung\\      % Eintrag in Zeile
  \hline                                                     % Gitterlinie unten
  1.0    &    27.06.2012    &    sämtliche Auftragnehmer    &    
  abgenommen    &    \\      % Eintrag in Zeile
  \hline                                                     % Gitterlinie unten
  
  

%----Ende der Tabelle-----------------------------------------------------------
\end{longtable}
Status: "`in Bearbeitung"' oder "`abgenommen"'\\
Kommentar: hier eintragen, was ge\"andert bzw. erg\"anzt wurde\\

Hinweis zum Template:\\
Dieses Template enth\"alt Hinweise, die alle kursiv geschrieben sind. Alles
Kursivgeschriebene ist selbstverst\"andlich bei Abgabe zu entfernen.\\
Angaben in <...> sind mit dem entsprechendem Text zu f\"ullen.\\
\"Uberz\"ahlige Kapitel, d.h. Kapitel, die nicht bearbeitet werden m\"ussen, da
sie nicht der Aufgabenstellung entsprechen, bitte entfernen.\\

Aufgabe des Feinentwurfs:\\
Der Feinentwurf dokumentiert die klassischen Entwurfsentscheidungen wie z.B.
Verwendung bestimmter Bibliotheken oder Entwurfsmuster. Darueber hinaus
bildet der Feinentwurf die Grundlage der Implementierung, d.h. anhand
dieses Dokumentes muss jeder Softwareentwickler in der Lage sein, das Produkt
zu entwickeln. Es ist also auf Vollst\"andigkeit der Dokumentation zu
achten.
