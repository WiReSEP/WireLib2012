% Kapitel 2 mit den entsprechenden Unterkapiteln
% Die Unterkapitel können auch in separaten Dateien stehen,
% die dann mit dem \include-Befehl eingebunden werden.
%-------------------------------------------------------------------------------
\chapter{Erfüllung der Kriterien}

Nachfolgend wird beschrieben, wie die einzelnen Kriterien des Pflichtenheftes
erfüllt werden und worauf geachtet wird.
\section{Musskriterien}

Die folgenden Kriterien sind unabdingbar und werden durch das Produkt erfüllt:
\begin{description}
  \item[/M10/] Es muss eine gut funktionierende Suchfunktion vorhanden sein.
  \item[/M20/] Den Benutzern werden verschiedene Rollen zugewiesen
  \item[/M21/] Gäste können Bücher suchen und sich Informationen zu diesen Anzeigen
			   lassen
  \item[/M22/] Normale Nutzer können auf alle Funktionen, die für den allgemeinen
			   allgemeinen Gebrauch nötig sind, zugreifen. So zum Beispiel das Entleihen von
			   Dokumenten, oder die Einsicht in ihre derzeitig ausgeliehen Dokumente.
  \item[/M23/] Verwaltung
  \item[/M24/] Der Bibliothekar besitzt Verwaltungsrechte für Dokumente.
  \item[/M25/] Der Administrator besitzt Volzugriff auf ddas System.
  \item[/M30/] Es ist für Externe möglich sich ein Dokument auszuleihen indem ihnen
ein registrierter Nutzer dieses Überträgt.
  \item[/M40/] Jedes Dokument besitzt verschiedene Eigenschaften wie eine Kategorie,
Authoren und weitere. Auch kann eine Ausleih-Historie zu jedem Dokument
abgefragt werden.
  \item[/M50/] Das Layout ist an das Corporate Design der Technischen Universität
Braunschweig angepasst und ist klar Strukturiert.
  \item[/M60/] Für die Zeichencodierung wird UTF-8 verwendet.
Ein Import von BibTeX Dateien ist möglich, sowie ein Export in BibTeX, als
Backup, und in ein für Allegro lesbares Datenformat.
  \item[/M70/] Sicheres Protokoll?
\end {description}

\section{Wunschkriterien}
Die Erfüllung folgender Kriterien für das abzugebende Produkt wird angestrebt:
\begin{description}
  \item[/W10/] Die einzelnen Rechte von Rollen können flexibel zu jeder Zeit von der
Verwaltung oder dem Administrator geändert werden.
Dieses Kriterium wird erfüllt.
  \item[/W20/] Ein Filter in der Suche nach dem Vorbild des Thunderbird Message Filters.
  \item[/W30/] Verschicken von Erinnerungsmails sobald die Ausleihfrist abgelaufen ist.
  \item[/W40/] Eine Authentifizierung über Lightweight Directory Access Protocoll (LDAP)
wird nicht realisiert werden.
\end{description}

\section{Abgrenzungskriterien}
Folgende Funktionalitäten werden nicht durch das Produkt, sondern wie folgt
beschrieben anderweitig erfüllt:
\begin{description}
  \item[/A10/] Es findet kein direkter Datentransfer mit der UB statt, dies wird über
den Export einer von Allegro lesbaren Datei realisiert.
  \item[/A20/] Das System wird nicht darauf ausgelegt mehr als 100 Benutzer oder mehr
als 5000 Dokumente zu beinhalten.
  \item[/A30/] Es wird keine Möglichkeit geben sich das Layout in einer anderen Sprache
anzeigen zu lassen außer Deutsch.
  \item[/A40/] Bei Importfehlern wird keine dynamische Fehlerbehebung durchgeführt. Es
wird ein Zeichen gesetzt das der Datensatz fehlerhaft ist und muss dann manuell
behoben werden.
  \item[/A50/] Es wird kein spezielles Layout für mobile Endgeräte geben.
\end{description}
