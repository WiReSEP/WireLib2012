% Kapitel 2 mit den entsprechenden Unterkapiteln
% Die Unterkapitel können auch in separaten Dateien stehen,
% die dann mit dem \include-Befehl eingebunden werden.
%-------------------------------------------------------------------------------
\chapter{Erfüllung der Kriterien}

Nachfolgend wird beschrieben, wie die einzelnen Kriterien des Pflichtenheftes
erfüllt werden und worauf geachtet wird.
\section{Musskriterien}

Die folgenden Kriterien sind unabdingbar und werden durch das Produkt erfüllt:
\begin{description}
  \item[/M10/] Es muss eine gut funktionierende Suchfunktion vorhanden sein.
  \item[/M20/] Den Benutzern werden verschiedene Rollen zugewiesen.
  \item[/M21/] Gäste können Bücher suchen und sich Informationen zu diesen
	anzeigen lassen.
  \item[/M22/] Normale Nutzer können auf alle Funktionen, die für den
	allgemeinen Gebrauch nötig sind, zugreifen. So zum Beispiel
	das Entleihen von Dokumenten oder die Einsicht in ihre derzeitig
	ausgeliehen Dokumente.
  \item[/M23/] Die Benutzerverwaltung wird durch die Rolle User-Admin realisiert.
  \item[/M24/] Der Bibliothekar besitzt Verwaltungsrechte für Dokumente.
  \item[/M25/] Der Administrator besitzt Vollzugriff auf das System.
  \item[/M30/] Es ist für \gls{glos:ext} möglich, sich ein Dokument auszuleihen,
	indem ihnen ein registrierter Nutzer dieses überträgt.
  \item[/M40/] Jedes Dokument besitzt verschiedene Eigenschaften, wie eine
	Kategorie, Autoren und weitere. Auch kann eine Ausleih-Historie zu jedem
	Dokument abgefragt werden.
  \item[/M50/] Das Layout ist an das \gls{glos:copdes} der Technischen
	Universität Braunschweig angepasst und ist klar strukturiert.
  \item[/M60/] Für die Zeichencodierung wird UTF-8, welches eine Form von
	\gls{glos:unicode} ist, verwendet.
	Ein Import von \gls{glos:BibTeX} Dateien ist möglich, sowie ein Export in
	\gls{glos:BibTeX}, als Backup, und eine adt-Datei für \gls{glos:Allegro}.
  \item[/M70/] Für den sicheren Datentransfer wird \gls{glos:ssl}/\gls{glos:https} verwendet.
\end{description}

\section{Wunschkriterien}
Die Erfüllung folgender Kriterien für das abzugebende Produkt wird angestrebt:
\begin{description}
  \item[/W10/] Die einzelnen Rechte von Rollen können flexibel zu jeder Zeit
	von der Verwaltung oder dem Administrator geändert werden. Dieses
	Kriterium wird erfüllt.
  \item[/W20/] Ein Filter in der Suche nach dem Vorbild des \gls{glos:thmefi}s.
  \item[/W30/] Verschicken von Erinnerungsmails, sobald die Ausleihfrist
	abgelaufen ist.
  \item[/W40/] Eine Authentifizierung über \gls{LDAP} wird nicht realisiert werden.
\end{description}

\section{Abgrenzungskriterien}
Folgende Funktionalitäten werden nicht durch das Produkt, sondern wie folgt
beschrieben anderweitig erfüllt:
\begin{description}
\item[/A10/] Es findet kein direkter Datentransfer mit der \gls{UB} statt.
	Dies wird über den Export einer adt-Datei, welche für \gls{glos:Allegro}
	lesbar ist, realisiert.
  \item[/A20/] Das System wird nicht darauf ausgelegt, mehr als 100 Benutzer
	oder mehr als 5000 Dokumente zu beinhalten.
  \item[/A30/] Es wird keine Möglichkeit geben, sich das Layout in einer anderen
	Sprache, außer Deutsch, anzeigen zu lassen.
  \item[/A40/] Bei Importfehlern wird keine dynamische Fehlerbehebung 
	durchgeführt. Es wird ein Zeichen gesetzt, dass der Datensatz fehlerhaft
	ist, und muss dann manuell behoben werden.
  \item[/A50/] Es wird kein spezielles Layout für mobile Endgeräte erstellt.
\end{description}
