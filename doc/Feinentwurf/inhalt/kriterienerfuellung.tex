% Kapitel 2 mit den entsprechenden Unterkapiteln
% Die Unterkapitel können auch in separaten Dateien stehen,
% die dann mit dem \include-Befehl eingebunden werden.
%-------------------------------------------------------------------------------
\chapter{Erf\"ullung der Kriterien}

Nachfolgend wird beschrieben, wie die einzelnen Kriterien des Pflichtenheftes
erf\"ullt werden und worauf geachtet wird.  Es ist dabei explizit auf die
definierten Kriterien des Pflichtenheftes zu verweisen.
\section{Musskriterien}

Die folgenden Kriterien sind unabdingbar und m\"ussen durch das Produkt erf\"ullt
werden:

/M10/ ... \\
/M20/ ... \\

\section{Wunschkriterien}
Die Erf\"ullung folgender Kriterien f\"ur das abzugebende Produkt wird angestrebt:

/W10/ ... \\
/W20/ ... \\


\section{Abgrenzungskriterien}
Folgende Funktionalit\"aten werden nicht durch das Produkt, sondern wie folgt
beschrieben anderweitig erf\"ullt:

/A10/ ... \\
