% Kapitel 5
%-------------------------------------------------------------------------------
\chapter{Serverkonfiguration}
%Sollte ein Server (z.B. Tomcat) f\"ur die Bearbeitung und Nutzung des Produktes
%erforderlich sein, so ist hier dessen Konfiguration zu beschreiben. Dies
%geschieht durch explizite Nennung aller Konfigurationsdateien und notwendiger
%Eintr\"age.
Im folgenden eine vollständige Installationsbeschreibung am Beispiel eines
\Gls{glos:ubuntu} Servers mit \Gls{glos:lighttpd}. Diese
Beschreibung berücksichtigt einige Anforderungen für ein Produktivsystem nicht
und ist für ein solches Setup nur als Hilfsreferenz verwendbar. 

Das im folgenden beschriebene Setup weicht in Bezug auf das Datenbanksystem von
dem Pflichtenheft ab, in dem die Verwendung einer \Gls{glos:sqlite}-Datenbank
angesetzt wurde. Dieser Punkt des Pflichtenheftes kann ohne Probleme bei Bedarf
verwendet werden, ist aber aufgrund von Einschränkungen von \Gls{glos:sqlite}
für ein Produktivsystem nicht zu empfehlen\footnote{Eine SQLite Datenbank wird
bei Schreibzugriffen vollständig gesperrt und arbeitet durch ihre Persistenz
langsamer als eine MySQL-Datenbank}.

\section{Installation notwendiger Programme}
Um das System über einen Webserver verfügbar machen zu können muss erst einmal
die nötige Software installiert werden. In dieser Beschreibung wird auf den
schlanken und flexiblen Webserver \Gls{glos:lighttpd} gesetzt, als
Datenbanksystem wird ein \Gls{glos:mysql} Datenbanksystem verwendet.

Zu installierende Pakete:
\begin{itemize}
  \item lighttpd
  \item mysql-server
  \item python-django
  \item git
  \item python-pip
  \item python-mysqldb
\end{itemize}
Die Installation kann mit \zB \lstinline{apt-get install lighttpd} getätigt
werden.

Sobald diese Pakete installiert sind muss die \Gls{glos:mysql} Datenbank
vorbereitet werden.

\section{MySQL einrichten}
Um die Datenbank ein zu richten, wird in dieser Beschreibung auf das in
\Gls{glos:php} geschriebene Programm \Gls{glos:adminer}. Um dies jedoch
verwenden zu können, müssen die folgenden zwei weitere Pakete installiert
werden.

\begin{itemize}
  \item php5-cgi
  \item php5-mysql
\end{itemize}

Für \Gls{glos:adminer} muss zuerst \Gls{glos:php} auf dem Webserver verfügbar
gemacht werden. Dazu sind folgende Anpassungen an \Gls{glos:lighttpd}
notwendig:
In der Konfigurationsdatei \emph{/etc/lighttpd/lighttpd.conf} muss zu den
\lstinline{server.modules} das Modul \lstinline{mod_fastcgi} hinzugefügt
werden und am Ende der Konfigurationsdatei folgende Zeilen:

\begin{lstlisting}
fastcgi.server = ( ".php" => (( 
     "bin-path" => "/usr/bin/php-cgi",
     "socket" => "/tmp/php.sock",
     )))
\end{lstlisting}

Nun kann \Gls{glos:adminer} herunter geladen werden und unter \emph{/var/www}
gespeichert werden. Nach einem Neustart von \Gls{glos:lighttpd} mit
\lstinline{service lighttpd restart} kann man über Adminer mit einem Webbrowser
ansteuern und damit auf die lokale \Gls{glos:mysql}-Datenbank verbinden. Zuerst
sollten zur Absicherung die leeren Benutzer (über "`Rechte"') von der Datenbank
entfernt werden. Zusätzlich müssen noch die zwei Datensätze, die die leeren
Benutzer betreffen unter "`mysql/db"' gelöscht werden.  Danach wird ein neuer
Benutzer "`wirelib"' mit Rechten auf die Datenbank "`wirelib"' erzeugt. Die
Rechte des Benutzers müssen und sollten nur alle Rechte auf Tabelle und Spalte
haben.

Da nun der zukünftige Benutzer auf der Datenbank existiert, muss nur noch die
Datenbank "`wirelib"' erzeugt werden. Die Datenbank heißt "`wirelib"' und
bekommt den Zeichensatz "`utf8\_unicode\_ci"'.

Nach Abschluss dieser Vorbereitungen kann die adminer.php gelöscht und php in
lighttpd deaktiviert werden.

\section{Django einrichten}

Um das Produkt unter Django zum laufen zu bringen werden zwei weitere
Python-Pakete benötigt die mittels \Gls{glos:pip} installiert werden können.
Mit 
\begin{lstlisting}
  pip install django-pagination flup
\end{lstlisting}
werden die letzten Abhängigkeiten für das Produkt erfüllt.

\subsection{Django Projekt initieren}

Um das Projekt ein zu richten sind die folgenden Schritte notwendig:
\begin{enumerate}
  \item Mit \lstinline{cd /var/www} in den Webordner wechseln
  \item Mit \lstinline{django-admin startproject wire} das Projekt starten
  \item Mit \lstinline{git clone https://github.com/WiReSEP/WireLib2012.git}
	das Produkt herunterladen
  \item \lstinline{cd /var/www/wire}
  \item \lstinline{ln -s /var/www/WireLib2012/src/wirelib/documents documents}
	fügt das Produkt dem \Gls{glos:django}-Projekt hinzu.
\end{enumerate}

\subsection{Django Projekt einrichten}
Um das Produkt fertig einzurichten müssen im Projekt zwei Dateien editiert
werden, deren Setup im folgenden näher beschrieben wird. Evtl. Anpassungen an
spezielle Bedürfnisse inerhalb einer Django-Projektes müssten wahrscheinlich
hier vorgenommen werden.

\subsubsection{settings.py}
Ein \Gls{glos:django}-Projekt wird über die \emph{settings.py} eingerichtet.
Um das Produkt aber zu verwalten, muss die \emph{settings.py} noch editiert
werden. Den Anfang machen dabei die Zugriffsdaten für die Datenbank, die wie
folgt eingerichtet werden:


\begin{lstlisting}
DATABASES = {
    'default': {
        'ENGINE': 'django.db.backends.mysql',
        'NAME': 'wirelib',
        'USER': 'wirelib', 
        'PASSWORD': 'streng geheim',
        'HOST': 'localhost',
        'PORT': '', #Bei default einfach leer lassen.
    }
}
\end{lstlisting}

Weiter unten in der \emph{settings.py} muss die folgende Middelware aktiv sein:
\begin{lstlisting}
MIDDLEWARE_CLASSES = (
    'django.middleware.common.CommonMiddleware',
    'django.contrib.sessions.middleware.SessionMiddleware',
    'django.middleware.csrf.CsrfViewMiddleware',
    'django.middleware.csrf.CsrfResponseMiddleware',
    'django.contrib.auth.middleware.AuthenticationMiddleware',
    'django.contrib.messages.middleware.MessageMiddleware',
    'pagination.middleware.PaginationMiddleware',
)
\end{lstlisting}

Gefolgt von dem folgenden Templateverzeichnis:
\begin{lstlisting}
TEMPLATE_DIRS = (
    '/var/www/WireLib2012/src/templates',
)
\end{lstlisting}

Damit das Produkt auch verfügbar ist, muss es in der \emph{settings.py} unter
den installierten Apps zu finden sein. Auch zwei Abhängigkeiten müssen zu den
Apps ergänzt werden:
\begin{lstlisting}
INSTALLED_APPS(
    'django.contrib.auth',
    'django.contrib.contenttypes',
    'django.contrib.sessions',
    'django.contrib.sites',
    'django.contrib.messages',
    'django.contrib.staticfiles',
    'django.contrib.admin',
    'pagination',
    'documents',
)
\end{lstlisting}

Zu guter letzt folgen einige Anweisungen für den Template Prozessor:
\begin{lstlisting}
  TEMPLATE_CONTEXT_PROCESSORS = (
    'django.contrib.auth.context_processors.auth',
    'django.core.context_processors.debug',
    'django.core.context_processors.i18n',
    'django.core.context_processors.media',
    'django.core.context_processors.static',
    'django.contrib.messages.context_processors.messages',
    'django.core.context_processors.request'
)
\end{lstlisting}

\subsubsection{urls.py}
Die \emph{urls.py} stellt die Weiche für einkommende Verbindungen dar, alle
Wege auf die Apps und die \emph{views.py} einer jeden App  wird hier
festgelegt.  Für den vollen Funktionsumfang des Produktes wird das
Admin-Backend benötigt. Für dieses Backend muss auch eine entsprechende Weiche
bereit stehen. Weiter wird eine Erweiterung für statische Dateien benötigt, die
nicht von Django interpretiert werden sollen. Zu guter letzt ist natürlich auch
noch ein Eintrag für das Produkt notwendig, der im folgenden Beispiel den
gesammten Root-Ordner mit Außnahme von den Ordnern \emph{static/} und
\emph{admin/}.

\begin{lstlisting}
from django.conf.urls.defaults import patterns, include, url
from django.contrib.staticfiles.urls import staticfiles_urlpatterns

from django.contrib import admin
admin.autodiscover(

urlpatterns = patterns('',
    url(r'^admin/', include(admin.site.urls)),
    url(r'^', include('documents.urls')),
)

urlpatterns += staticfiles_urlpatterns()
\end{lstlisting}

 \subsection{Django Setup abschließen}
 Der Projektordner ist nun bereit für einen Webserver. Was dem Projekt noch
 fehlt, ist die Datenbank, um seine Daten ablegen zu können.

 Die Datenbank wird wie folgt initialisiert:
\begin{lstlisting}
python manage.py syncdb
\end{lstlisting}

\section{Django in Lighttpd verfügbar machen}
In der Datei \emph{/etc/lighttpd/lighttpd.conf} müssen zuerst die nötigen
Module aktiviert werden. Das sind die folgenden:
\begin{lstlisting}
 'mod_fastcgi'
 'mod_rewrite'
\end{lstlisting}

Weiter muss das \emph{document-root} angepasst werden,
\begin{lstlisting}
server.document-root        = "/var/www/wire"
\end{lstlisting}
sowie die folgenden Zeilen der Konfiguration hinzugefügt:
\begin{lstlisting}
fastcgi.server = ( "/wire.fcgi" => (
                "main" => (
                        "socket" => "/var/www/wire/wire.sock",
                        "check-local" => "disable",
                )
        ),
)
alias.url = (
        "/media" => "/usr/lib/python2.7/dist-packages/django/contrib/admin/media/",
        "/static" => "/var/www/static"
)
url.rewrite-once = (
        "^(/wire\.fcgi.*)$" => "$1",
        "^(/media.*)$" => "$1",
        "^(/.*)$" => "/wire.fcgi$1",
)
\end{lstlisting}

 \subsection{Django FCGI-Skript}
Das folgende Skript kann als Grundlage für ein automatisches FCGI-Start-Skript
verwendet werden. Wichtig ist dafür allein, dass dem Benutzer Webserver
Benutzer der Ordner
\lstinline{$PROJECTDIR} gehört:
\begin{lstlisting}
chown www-data:www-data /var/www/wire
\end{lstlisting}

Nun das \Gls{cgi}-Skript:
\lstinputlisting{inhalt/django-fcgi}
