% Kapitel 5
%-------------------------------------------------------------------------------
\chapter{Serverkonfiguration}
%Sollte ein Server (z.B. Tomcat) f\"ur die Bearbeitung und Nutzung des Produktes
%erforderlich sein, so ist hier dessen Konfiguration zu beschreiben. Dies
%geschieht durch explizite Nennung aller Konfigurationsdateien und notwendiger
%Eintr\"age.
Im folgenden eine vollständige Installationsbeschreibung am Beispiel eines
\Gls{glos:ubuntu} Servers mit \Gls{glos:lighttpd} und \Gls{glos:fastcgi}.

\section{Installation notwendiger Programme}
Für die Installation müssen zuerst die notwendigen Pakete installiert werden,
dies sind die folgenden:

#Packete
##Webserver: lighttpd
Erstmal deaktivieren: service lighttpd stop
Für adminer noch installieren: php5-mysql php5-cgi

##Mysql: mysql-server
noch zusätzliches Paket: python-mysqldb
##Django: python-django
##Git: git
##pip: python-pip

#pagination: pip install django-pagination 

# Mysql-Server einrichten

#Django-Seite initieren:
  cd /var/www
  django-admin startproject wire

  cd /var/www
  git clone https://github.com/WiReSEP/WireLib2012.git
  ln -s WireLib2012/src/wirelib/documents wire/documents

#Webseite einrichten:
Zuerst die settings.py
 0. DEBUG= False
 1. Datenbank zugriff eintragen.
% 2. Autocommit ausschalten: 'OPTIONS'\ldots
 2. Templatedir eintragen: /var/www/WireLib2012/src/templates
 3. installed_apps eintragen: 
  'pagination'
  'documents'
  'django.contrib.admin' (auskommentieren)

dann urls.py
 1. import nachtragen: 
 from django.contrib.staticfiles.urls import staticfiles_urlpatterns
 2. zwei Zeilen für admin auskommentieren.
 3. urlpatterns erweitern:
 zuerst admin-sektion: url(r'^admin/', include(admin.site.urls)),
 dann url(r'^', include('documents.urls'))
 4. urlpatterns += staticfiles_urlpatterns()
