%Dies ist die Hauptseite des Dokumentes. Es werden u. a. alle Kapitel,
%Einstellung im Header eingebunden.  Veränderungen müssen in folgenden Dateien
%vorgenommen werden:
      %- Layout.tex
      %- newComments.tex
      %- Titelseite
      %- Versionsübersicht
      %- einzelne Kapitel (evtl. erweitern)

\input{einstellungen/newComments.tex}
\input{einstellungen/layout.tex}          % Diese Datei enthält alle
                                          % Layouteinstellungen
\usepackage{float}

% Dieses Befehle sortieren die Einträge in den einzelnen Listen:
%makeindex -s datei.ist -t datei.alg -o datei.acr datei.acn
%makeindex -s datei.ist -t datei.glg -o datei.gls datei.glo
%makeindex -s datei.ist -t datei.slg -o datei.syi datei.syg

% Kapitel 9
%-------------------------------------------------------------------------------

\chapter{Glossar}
Hier werden Fachbegriffe erklärt.
%% TODO: 	glossaries einpflegen.
%%		Erklährung unter: http://ewus.de/tipp-1029.html
\begin{itemize}
  \item https
  \item AFS
  \item GITZ
  \item UB
  \item sqlite
  \item \BibTeX
  \item Allegro
  \item reguläre Ausdrücke
\end{itemize}


\newcommand{\BibTeX}{\gls{glos:BibTeX}\xspace}
\newcommand{\zB}{\mbox{z.\,B.}\xspace}
%------Beginn des Gesamtdokumentes----------------------------------------------
\begin{document}

%------Eingebundene Seiten, Verzeichnisse bzw. Kapitel--------------------------
% Dies ist die Titelseite des Pflichtenhefts.
% Die in "<...>" sind zu ersetzen
% Die Ausgabe darf 1 Seite nicht überschreiten, also ggf. Abstände anpassen
% Die Angabe in [...] gibt den Abstand nach der entsprechenden Zeile an.


%----Stil dieser Seite----------------------------------------------------------
\thispagestyle{plain}      % Kopfzeile bleibt leer

%----Beginn der Titelseite------------------------------------------------------
\begin{titlepage}

%----zentrierte Ausrichtung über die gesamte Seite----------------------------
\begin{center}

%----Titel des Praktikum (\praktikumTitel in newComments zu verändern)--------
{\relsize{4}{\textbf{\textsc{\praktikumTitel}}}}\\[3ex]%[5ex]

%----Titel des Teilprojektes (\projektTitel in newComments verändern)---------
{\relsize{3}{\textbf{\textsc{\projektTitel}}}}\\[3ex]%[5ex]

Software-Entwicklungspraktikum (SEP)\\
Sommersemester 2012\\[4ex]%[6ex]

{\relsize{3}\so{\textbf{Pflichtenheft}}}\\[4ex]%[5ex]

%----eingebundenes Logo der TU--------------------------------------------------
\includegraphics[scale=0.8]{bilder/carolo.jpg}\\[4ex]%[5ex]

%----Daten des Auftraggebers
Auftraggeber\\
Technische Universität Braunschweig\\
Wissenschaftliches Rechnen\\
Prof. Hermann G. Matthies\\
Hans-Sommer-Straße 65\\
D-38092 Braunschweig\\[1ex]%[2ex]
Betreuer: Elmar Zander\\[4ex]%[5ex]

% ----Tabelle der Praktikumsteilnehmer------------------------------------------
Auftragnehmer:
\begin{tabular}{l<{\hspace{20mm}} l<{\hspace{30mm}}}\\
  Name                   &   E-Mail-Adresse\\      % Zeilenüberschift
  \hline                    % Linie unterhalb der Zeilenüberschrift
  %----Nachfolgend alle Namen und E-Mail-Adressen der Teilnehmer einfügen
  Eric Anders 		& eric.anders89@web.de\\
  Johann Hong 		& johann.hong@googlemail.com\\
  Jörn Hameyer 		& j.hameyer@tu-bs.de\\
  Marco Melzer 		& marco.melzer@tu-braunschweig.de\\
  Markus Dietrich 	& markus.dietrich@tu-bs.de\\
  Philipp Offensand & PhilippOffensand@gmx.de\\
  Stephan Sobol 	& stephan.sobol@web.de\\
  Theodor van Nahl 	& t.nahl@tu-bs.de
\end{tabular}\\[1ex]%[2ex]

Braunschweig, 24.04.2012

\end{center}
\end{titlepage}
                      % Titelseite
%Diese Datei dient der Versionskontrolle. Sie ist vollständig zu bearbeiten.

%----Überschrift----------------------------------------------------------------
{\relsize{2}\textbf{Versionsübersicht}}\\[2ex]

%----Start der Tabelle----------------------------------------------------------
\begin{longtable}{|m{1.78cm}|m{1.59cm}|m{2.86cm}|m{1.9cm}|m{5.25cm}|}

  \hline                                              % Linie oberhalb

  %----Spaltenüberschriften-----------------------------------------------------
  \textbf{Version}  &    \textbf{Datum}  &    \textbf{Autor}  &
  \textbf{Status}   &    \textbf{Kommentar}       \\  %Spaltenüberschrift
  \hline                                              % Gitterlinie


  %----die nachfolgenden beiden Zeilen so oft wiederholen und die ... mit den
  %    entsprechenden Daten zu füllen wie erforderlich
  0.1    &    25.06.2012    &    Stephan Sobol,\newline Phiipp \mbox{Offensand}    &    
  in Bearbeitung    &  Kapitel 4   \\      % Eintrag in Zeile
  \hline                                                     % Gitterlinie unten
  0.2    &    25.06.2012    &    Markus Dietrich,\newline Jörn Hameyer    &    
  in Bearbeitung    &    Kapitel 3\\      % Eintrag in Zeile
  \hline                            % Gitterlinie unten
  0.3    &    25.06.2012    &    Johann Hong    &    
  in Bearbeitung    &    Kapitel 1\\      % Eintrag in Zeile
  \hline                                                     % Gitterlinie unten
  0.4    &    25.06.2012    &    Theodor \mbox{van Nahl}    &    
  in Bearbeitung    &    Kapitel 5\\      % Eintrag in Zeile
  \hline                                                     % Gitterlinie unten
  0.5    &    26.06.2012    &    Marco Melzer    &    
  in Bearbeitung    &    Kapitel 2\\      % Eintrag in Zeile
  \hline                                                     % Gitterlinie unten
  0.6    &    27.06.2012    &    Eric Anders    &    
  in Bearbeitung    &    Glossar und Rechtschreibprüfung\\      % Eintrag in Zeile
  \hline                                                     % Gitterlinie unten
  1.0    &    27.06.2012    &    sämtliche Auftragnehmer    &    
  abgenommen    &    \\      % Eintrag in Zeile
  \hline                                                     % Gitterlinie unten
  
  

%----Ende der Tabelle-----------------------------------------------------------
\end{longtable}
Status: "`in Bearbeitung"' oder "`abgenommen"'\\
Kommentar: hier eintragen, was ge\"andert bzw. erg\"anzt wurde\\

Hinweis zum Template:\\
Dieses Template enth\"alt Hinweise, die alle kursiv geschrieben sind. Alles
Kursivgeschriebene ist selbstverst\"andlich bei Abgabe zu entfernen.\\
Angaben in <...> sind mit dem entsprechendem Text zu f\"ullen.\\
\"Uberz\"ahlige Kapitel, d.h. Kapitel, die nicht bearbeitet werden m\"ussen, da
sie nicht der Aufgabenstellung entsprechen, bitte entfernen.\\

Aufgabe des Feinentwurfs:\\
Der Feinentwurf dokumentiert die klassischen Entwurfsentscheidungen wie z.B.
Verwendung bestimmter Bibliotheken oder Entwurfsmuster. Darueber hinaus
bildet der Feinentwurf die Grundlage der Implementierung, d.h. anhand
dieses Dokumentes muss jeder Softwareentwickler in der Lage sein, das Produkt
zu entwickeln. Es ist also auf Vollst\"andigkeit der Dokumentation zu
achten.
        % Versionsübersicht

\tableofcontents                          % Inhaltsverzeichnis wird automatisch
                                          % generiert
\listoffigures                            % ebenso das Abbildungsverzeichnis

Abbildung 1: Komponentendiagramm                                         7\\
Abbildung 2: Implementierung von Komponente <Name>                       7\\
Abbildung 3: Datenmodell                                                 9\\

%----Kapitel des Feinentwurfs, die mit Inhalt zu füllen sind--------------------
% Kapitel 1
%-------------------------------------------------------------------------------
\chapter{Einleitung}
%Hier Einleitungstext einfügen, dabei die Formatierungen selber erstellen

In diesem Feinentwurf werden die thematisierten Inhalte im Pflichtenheft und im Grobentwurf, die genaue Konstruktion eines Bibliothekmanagementsystems (WireLib), weiter konkretisiert. 
Dies erfolgt mit dem Ziel, jedem außenstehenden Softwareentwickler zu ermöglichen, die Problematik besser zu verstehen und mithilfe des Dokumentes das Projekt technisch umzusetzen.\\
Somit stehen programmspezifische Details im Vordergrund. \\
Am Anfang werden die Grundbedingungen des zu implementierenden Systems, bestehend aus den Musskriterien und den Wunschkriterien, vorgestellt.
Darauf folgt als wichtiges Element der Implementierungsentwurf: \\
Es behandelt die Dokumentation von verwendeten Klassen und Bibliotheken mit ihren entsprechenden Attributen und Methoden, unter anderem mithilfe von Klassendiagrammen sowie anderen UML Diagrammen.\\ 
Das zu verwendende Datenmodell wird im nächsten Kapitel explizit beschrieben: \\
Alle benutzten Entities und ihre Beziehungen zueinander werden näher erläutert, und sollen ein besseres Grundverständnis der Datenbank übermitteln.
Danach folgen die Betriebsanleitungen für die Serverkonfiguration.    
               % Kapitel 1
% Kapitel 2 mit den entsprechenden Unterkapiteln
% Die Unterkapitel können auch in separaten Dateien stehen,
% die dann mit dem \include-Befehl eingebunden werden.
%-------------------------------------------------------------------------------
\chapter{Erf\"ullung der Kriterien}

Nachfolgend wird beschrieben, wie die einzelnen Kriterien des Pflichtenheftes
erf\"ullt werden und worauf geachtet wird.  Es ist dabei explizit auf die
definierten Kriterien des Pflichtenheftes zu verweisen.
\section{Musskriterien}

Die folgenden Kriterien sind unabdingbar und m\"ussen durch das Produkt erf\"ullt
werden:

/M10/ ... \\
/M20/ ... \\

\section{Wunschkriterien}
Die Erf\"ullung folgender Kriterien f\"ur das abzugebende Produkt wird angestrebt:

/W10/ ... \\
/W20/ ... \\


\section{Abgrenzungskriterien}
Folgende Funktionalit\"aten werden nicht durch das Produkt, sondern wie folgt
beschrieben anderweitig erf\"ullt:

/A10/ ... \\
      % Kapitel 2
% Kapitel 3 mit den entsprechenden Unterkapiteln
% Die Unterkapitel können auch in separaten Dateien stehen,
% die dann mit dem \include-Befehl eingebunden werden.
%-------------------------------------------------------------------------------
\chapter{Implementierungsentwurf}
Dieser Abschnitt hat die Aufgabe, alle verwendeten Klassen und Bibliotheken zu
dokumentieren. Dabei wird jede Komponente aus dem Grobentwurf gesondert
betrachtet. F\"ur Entwurfsentscheidungen, die mehr als eine Komponente betreffen,
wird mit Verweisen zwischen den Dokumentationen der Komponente gearbeitet.  Es
sind dabei so viele Unterabschnitte einzuf\"ugen, wie Komponenten vorhanden sind.


\section{Gesamtsystem}
F\"ugen Sie hier bitte das Komponentendiagramm aus dem Grobentwurf ein und
erl\"autern Sie kurz die Funktionen der Komponenten.

\section{Implementierung von Komponente
         <ID aus Grobentwurf>: <Komponentenname>:}

Beschreiben Sie hier bitte die Implementierung der Komponente. Erl\"autern Sie
bitte dabei, welche Entwurfsmuster und Bibliotheken Sie verwenden. Die
Implementierung wird dabei durch Klassendiagramme dokumentiert.

\subsection{Paketdiagramm}
\subsection{Erl\"auterung}

Die verwendeten Attribute, Aufgaben und Kommunikationspartner sind f\"ur jede
Klasse kurz zu erl\"autern. Die ankommenden Nachrichten beziehen sich dabei auf
die Sequenzdiagramme der Feinanalyse im Grobentwurf und stellen meist
aufzurufende Methoden der Klasse dar.  Reine get- / set-Methoden oder
Bibliotheksfunktionen brauchen nicht aufgef\"uhrt zu werden.  % Kapitel 3
% Kapitel 4 mit den entsprechenden Unterkapiteln
% Die Unterkapitel können auch in separaten Dateien stehen,
% die dann mit dem \include-Befehl eingebunden werden.
%-------------------------------------------------------------------------------
\chapter{Datenmodell}
\label{kap4}
Falls in der Anwendung bestimmte Daten dauerhaft gespeichert werden, so sind
die entsprechenden Entities und Beziehungen hier darzustellen und zu erläutern.
Dies ist insbesondere relevant, falls der Einsatz einer (relationalen)
Datenbank geplant ist.

\section{Diagramm}


\begin{figure}[H]
\includegraphics[width=1.0\linewidth]{bilder/db_wirelib-packages.pdf}
\caption{Datenbankmodell}
\label{fig:DBDiagramm}
\end{figure}

Das Diagramm gibt eine grobe Übersicht über die Struktur der zugrundeliegenden
relationalen Datenbank. Im Folgenden werden die beiden wichtigsten
Teilstrukturen der Datenbank für bessere Übersicht dargestellt.


\begin{figure}[H]
\includegraphics[width=1.0\linewidth]{bilder/database-wirelib_cluster-doc.pdf}
\caption{Datenbankmodell: Bibliothek}
\label{fig:DB_docDiagramm}
\end{figure}

Dieses Diagramm gibt eine Übersicht über alle Tabellen der Datenbank, die im
Zusammenhang mit der Bibliothek stehen. Die wichtigste Tabelle ist document, in
der die vorhandenen Dokumente eingetragen werden. In der Tabelle doc\_status
werden alle Statusänderungen eines Dokumentes festgehalten, einschließlich der
Ausleihe eines Dokumentes. Die Tabelle doc\_extra speichert zusätzliche Inhalte
eines Dokumentes, die aufgrund seltenen Gebrauchs nicht lohnen, in document
direkt aufgenommen zu werden. In der Tabelle emails werden die E-Mail-Templates
gespeichert, die verschickt werden, wenn die Ausleihfrist eines Dokumentes
abläuft oder in ähnlichen Fällen. Die Tabellen auth\_user und non\_user sind
in der Teilstruktur Benutzer und hier nur als Referenz gedacht zur besseren
Übersicht. Die verbleibenden Tabellen speichern trivialerweise den
beschriebenen Inhalt.


\begin{figure}[H]
\includegraphics[width=1.0\linewidth]{bilder/database-wirelib_cluster-user.pdf}
\caption{Datenbankmodell: Benutzer}
\label{fig:DB_UserDiagramm}
\end{figure}

Dieses Diagramm gibt eine Übersicht über alle Tabellen der Datenbank, die im
Zusammenhang mit Benutzerverwaltung stehen. Alle Tabellen, die in diesem Modell
mit dem Präfix \emph{auth\_} beginnen, sind von Django generiert und werden für
eine erleichterte Handhabung von Benutzern und Rechtevergabe genutzt. Tabellen
mit dem Präfix \emph{django\_} werden ebenfalls von Django generiert und dienen
dem Admin-Interface von Django und dem Session-Managment von eingeloggten
Benutzern. Die Tabelle non\_user ist für Nutzer gedacht, die nicht am Institut
für Wissenschaftliches Rechnen arbeiten und trotzdem ein Dokument ausleihen
wollen. Sie haben so keine Möglichkeit, sich auf der Webseite einzuloggen,
können aber trotzdem Dokumente ausleihen. Die Tabellen tel\_user und
tel\_non\_user werden benötigt, um Telefonnumern zu Benutzern zu speichern.
user\_profile ist zur Erweiterung der Django-Tabelle auth\_user, um auch
Adressen speichern zu können.

% Eigenes Klassendiagramm einsetzen
\section{Erläuterung}
Im Folgenden sind alle Relationen von einer Tabelle zu einer anderen 
in der Datenbank angezeigt. Dabei ist jeweils oben die ausgehende Tabelle und 
unter \glqq Relation zu\grqq\ alle zu denen eine Verbindung existiert. Unter 
\glqq Kardinalität\grqq\ steht dann, mit wie vielen Datensätzen der 
Relationstabelle ein Datensatz der Ausgangstabelle verbunden ist. Angegeben wird 
dabei zuerst die Mindestanzahl von Verbindungen und nach dem Komma die 
Maximalanzahl. Bei \glqq 1,n\grqq\ hat Datensatz $A$ also mindestens eine und 
maximal unendlich viele Verbindungen zur Relationstabelle.

\begin{longtable}{@{}ccc@{}}
  \toprule
  \multicolumn{3}{c}{\emph{Tabelle:} author} \\
  Relation zu & Name der Beziehung & Kardinalität \\
  \cmidrule(lr){1-1}\cmidrule(lr){2-2}\cmidrule(lr){3-3}
  document\_authors & hat geschrieben & 1,n \\
  
  \toprule
  \multicolumn{3}{c}{\emph{Tabelle:} document\_authors} \\
  Relation zu & Name der Beziehung & Kardinalität \\
  \cmidrule(lr){1-1}\cmidrule(lr){2-2}\cmidrule(lr){3-3}
  author & wurde geschrieben von & 1,1 \\
  document & hat geschrieben & 1,1\\

  \toprule
  \multicolumn{3}{c}{\emph{Tabelle:} document} \\
  Relation zu & Name der Beziehung & Kardinalität \\
  \cmidrule(lr){1-1}\cmidrule(lr){2-2}\cmidrule(lr){3-3}
  document\_authors & wurde geschrieben von & 1,n\\
  category & gehört zur Kategorie & 1,1\\
  publisher & wurde veröffentlicht von & 1,1\\
  keywords & besitzt die Schlüsselwörter & 0,n\\  
  doc\_extra & besitzt die Extrafelder & 0,n\\
  doc\_status & hat den Status/verliehen an & 0,n\\

  \toprule
  \multicolumn{3}{c}{\emph{Tabelle:} keywords} \\
  Relation zu & Name der Beziehung & Kardinalität \\
  \cmidrule(lr){1-1}\cmidrule(lr){2-2}\cmidrule(lr){3-3}
  document & gehört zum Dokument & 1,1 \\

  \toprule
  \multicolumn{3}{c}{\emph{Tabelle:} publisher} \\
  Relation zu & Name der Beziehung & Kardinalität \\
  \cmidrule(lr){1-1}\cmidrule(lr){2-2}\cmidrule(lr){3-3}
  document & hat veröffentlicht & 1,n \\

  \toprule
  \multicolumn{3}{c}{\emph{Tabelle:} category} \\
  Relation zu & Name der Beziehung & Kardinalität \\
  \cmidrule(lr){1-1}\cmidrule(lr){2-2}\cmidrule(lr){3-3}
  document & enthält & 0,n \\

  \toprule
  \multicolumn{3}{c}{\emph{Tabelle:} doc\_extra} \\
  Relation zu & Name der Beziehung & Kardinalität \\
  \cmidrule(lr){1-1}\cmidrule(lr){2-2}\cmidrule(lr){3-3}
  document & gehört zu & 1,1 \\

  \toprule
  \multicolumn{3}{c}{\emph{Tabelle:} doc\_status} \\
  Relation zu & Name der Beziehung & Kardinalität \\
  \cmidrule(lr){1-1}\cmidrule(lr){2-2}\cmidrule(lr){3-3}
  document & gehört zum Dokument & 1,1\\
  auth\_user & wurde geändert von & 1,1\\
  auth\_user & wurde ausgeliehen von & 0,1\\
  non\_user & wurde weitergegeben an & 0,1\\

  \toprule
  \multicolumn{3}{c}{\emph{Tabelle:} non\_user} \\
  Relation zu & Name der Beziehung & Kardinalität \\
  \cmidrule(lr){1-1}\cmidrule(lr){2-2}\cmidrule(lr){3-3}
  doc\_status & entlieh & 1,n\\
  tel\_non\_user & hat die Telefonnummern & 1,n\\

  \toprule
  \multicolumn{3}{c}{\emph{Tabelle:} tel\_non\_user} \\
  Relation zu & Name der Beziehung & Kardinalität \\
  \cmidrule(lr){1-1}\cmidrule(lr){2-2}\cmidrule(lr){3-3}
  non\_user & gehört zu & 1,1 \\

  \toprule
  \multicolumn{3}{c}{\emph{Tabelle:} auth\_user} \\
  Relation zu & Name der Beziehung & Kardinalität \\
  \cmidrule(lr){1-1}\cmidrule(lr){2-2}\cmidrule(lr){3-3}
  user\_profile & wohnt & 1,1\\
  tel\_user & hat die Telefonnumern & 1,n\\
  auth\_user\_user\_permissions & hat die Rechte & 0,n\\
  doc\_status & änderte den Status & 0,n\\  
  doc\_status & leiht/bürgt gerade für & 0,n\\
  django\_admin\_log & aktualisiert & 0,n\\
  auth\_message & bekam & 0,n\\
  auth\_user\_groups & ist in Gruppe & 0,n\\

  \toprule
  \multicolumn{3}{c}{\emph{Tabelle:} tel\_user} \\
  Relation zu & Name der Beziehung & Kardinalität \\
  \cmidrule(lr){1-1}\cmidrule(lr){2-2}\cmidrule(lr){3-3}
  auth\_user & gehört zu & 1,1 \\

  \toprule
  \multicolumn{3}{c}{\emph{Tabelle:} auth\_message} \\
  Relation zu & Name der Beziehung & Kardinalität \\
  \cmidrule(lr){1-1}\cmidrule(lr){2-2}\cmidrule(lr){3-3}
  auth\_user & gesendet an & 1,1 \\

  \toprule
  \multicolumn{3}{c}{\emph{Tabelle:} user\_profile} \\
  Relation zu & Name der Beziehung & Kardinalität \\
  \cmidrule(lr){1-1}\cmidrule(lr){2-2}\cmidrule(lr){3-3}
  auth\_user & Anschrift von & 1,1 \\

  \toprule
  \multicolumn{3}{c}{\emph{Tabelle:} auth\_user\_groups} \\
  Relation zu & Name der Beziehung & Kardinalität \\
  \cmidrule(lr){1-1}\cmidrule(lr){2-2}\cmidrule(lr){3-3}
  auth\_user & der User & 1,1 \\
  auth\_group & gehört zur Gruppe & 1,1 \\

  \toprule
  \multicolumn{3}{c}{\emph{Tabelle:} auth\_group} \\
  Relation zu & Name der Beziehung & Kardinalität \\
  \cmidrule(lr){1-1}\cmidrule(lr){2-2}\cmidrule(lr){3-3}
  auth\_user\_groups & besitzt die Mitglieder & 0,n \\
  auth\_group\_permissions & hat die Rechte & 0,n \\

  \toprule
  \multicolumn{3}{c}{\emph{Tabelle:} auth\_group\_permissions} \\
  Relation zu & Name der Beziehung & Kardinalität \\
  \cmidrule(lr){1-1}\cmidrule(lr){2-2}\cmidrule(lr){3-3}
  auth\_group & die Gruppe & 1,1 \\
  auth\_permission & hat das Recht & 1,1 \\

  \toprule
  \multicolumn{3}{c}{\emph{Tabelle:} auth\_user\_user\_permissions} \\
  Relation zu & Name der Beziehung & Kardinalität \\
  \cmidrule(lr){1-1}\cmidrule(lr){2-2}\cmidrule(lr){3-3}
  auth\_user & der User & 1,1 \\
  auth\_permission & hat das Recht & 1,1 \\

  \toprule
  \multicolumn{3}{c}{\emph{Tabelle:} auth\_permission} \\
  Relation zu & Name der Beziehung & Kardinalität \\
  \cmidrule(lr){1-1}\cmidrule(lr){2-2}\cmidrule(lr){3-3}
  auth\_user\_user\_permissions & der User hat  & 0,n \\
  auth\_group\_permissions & die Gruppe hat & 0,n \\
  django\_content\_type & referenziert & 1,1 \\

  \toprule
  \multicolumn{3}{c}{\emph{Tabelle:} django\_admin\_log} \\
  Relation zu & Name der Beziehung & Kardinalität \\
  \cmidrule(lr){1-1}\cmidrule(lr){2-2}\cmidrule(lr){3-3}
  auth\_user & getätigt von  & 1,1 \\
  django\_content\_type & referenziert & 1,1 \\

  \toprule
  \multicolumn{3}{c}{\emph{Tabelle:} django\_content\_type} \\
  Relation zu & Name der Beziehung & Kardinalität \\
  \cmidrule(lr){1-1}\cmidrule(lr){2-2}\cmidrule(lr){3-3}
  auth\_permission & stellt Referenz  & 0,n \\
  django\_admin\_log & ändert & 0,n \\
\end{longtable}
              % Kapitel 4
% Kapitel 5
%-------------------------------------------------------------------------------
\chapter{Serverkonfiguration}
%Sollte ein Server (z.B. Tomcat) f\"ur die Bearbeitung und Nutzung des Produktes
%erforderlich sein, so ist hier dessen Konfiguration zu beschreiben. Dies
%geschieht durch explizite Nennung aller Konfigurationsdateien und notwendiger
%Eintr\"age.
Im folgenden eine vollständige Installationsbeschreibung am Beispiel eines
\Gls{glos:ubuntu} Servers.

\section{Installation notwendiger Programme}
Erforderlich für die Installation 

      % Kapitel 5

%---Hier werden das Glossar und das Abkürzungsverzeichnis gedruckt--------------
\providetranslation{Glossary}{Glossar}
\printglossary[style=altlist,title=Glossar]

\providetranslation{Acronyms}{Akronyme}
\printglossary[type=\acronymtype,style=long]

%------Ende des Dokumentes------------------------------------------------------
\end{document}
