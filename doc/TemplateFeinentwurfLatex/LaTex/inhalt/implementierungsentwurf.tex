% Kapitel 3 mit den entsprechenden Unterkapiteln
% Die Unterkapitel können auch in separaten Dateien stehen,
% die dann mit dem \include-Befehl eingebunden werden.
%-------------------------------------------------------------------------------
\chapter{Implementierungsentwurf}
Dieser Abschnitt hat die Aufgabe, alle verwendeten Klassen und Bibliotheken zu
dokumentieren. Dabei wird jede Komponente aus dem Grobentwurf gesondert
betrachtet. F\"ur Entwurfsentscheidungen, die mehr als eine Komponente betreffen,
wird mit Verweisen zwischen den Dokumentationen der Komponente gearbeitet.  Es
sind dabei so viele Unterabschnitte einzuf\"ugen, wie Komponenten vorhanden sind.


\section{Gesamtsystem}
F\"ugen Sie hier bitte das Komponentendiagramm aus dem Grobentwurf ein und
erl\"autern Sie kurz die Funktionen der Komponenten.

\section{Implementierung von Komponente
         <ID aus Grobentwurf>: <Komponentenname>:}

Beschreiben Sie hier bitte die Implementierung der Komponente. Erl\"autern Sie
bitte dabei, welche Entwurfsmuster und Bibliotheken Sie verwenden. Die
Implementierung wird dabei durch Klassendiagramme dokumentiert.

\subsection{Paketdiagramm}
\subsection{Erl\"auterung}

Die verwendeten Attribute, Aufgaben und Kommunikationspartner sind f\"ur jede
Klasse kurz zu erl\"autern. Die ankommenden Nachrichten beziehen sich dabei auf
die Sequenzdiagramme der Feinanalyse im Grobentwurf und stellen meist
aufzurufende Methoden der Klasse dar.  Reine get- / set-Methoden oder
Bibliotheksfunktionen brauchen nicht aufgef\"uhrt zu werden.